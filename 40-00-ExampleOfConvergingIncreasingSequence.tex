\documentclass[12pt]{article}
\usepackage{pmmeta}
\pmcanonicalname{ExampleOfConvergingIncreasingSequence}
\pmcreated{2013-03-22 17:40:44}
\pmmodified{2013-03-22 17:40:44}
\pmowner{pahio}{2872}
\pmmodifier{pahio}{2872}
\pmtitle{example of converging increasing sequence}
\pmrecord{6}{40118}
\pmprivacy{1}
\pmauthor{pahio}{2872}
\pmtype{Example}
\pmcomment{trigger rebuild}
\pmclassification{msc}{40-00}
\pmrelated{NthRoot}
\pmrelated{BolzanosTheorem}

% this is the default PlanetMath preamble.  as your knowledge
% of TeX increases, you will probably want to edit this, but
% it should be fine as is for beginners.

% almost certainly you want these
\usepackage{amssymb}
\usepackage{amsmath}
\usepackage{amsfonts}

% used for TeXing text within eps files
%\usepackage{psfrag}
% need this for including graphics (\includegraphics)
%\usepackage{graphicx}
% for neatly defining theorems and propositions
 \usepackage{amsthm}
% making logically defined graphics
%%%\usepackage{xypic}

% there are many more packages, add them here as you need them

% define commands here

\theoremstyle{definition}
\newtheorem*{thmplain}{Theorem}

\begin{document}
Let $a$ be a positive real number and $q$ an integer greater than 1.\, Set
$$x_1 := \sqrt[q]{a},$$
$$x_2 := \sqrt[q]{a+x_1} = \sqrt[q]{a+\sqrt[q]{a}},$$
$$x_3 := \sqrt[q]{a+x_2} = \sqrt[q]{a+\sqrt[q]{a+\sqrt[q]{a}}},$$
and generally
\begin{align}
x_n := \sqrt[q]{a+x_{n-1}}.
\end{align}
Since\, $x_1 > 0$,\, the two first above equations imply that\, $x_1 < x_2$.\, By induction on $n$ one can show that
$$x_1 < x_2 < x_3 < \ldots < x_n < \ldots$$
The numbers $x_n$ are all below a finite bound $M$.\, For demonstrating this, we write the inequality \,$x_n < x_{n+1}$\, in the form \,$x_n < \sqrt[q]{a+x_n}$, which implies \, $x_n^q < a+x_n$,\, i.e.
\begin{align}
x_n^q-x_n-a < 0
\end{align}
for all $n$.\, We study the polynomial
$$f(x) := x^q-x-a = x(x^{q-1}-1)-1.$$
From its latter form we see that the function $f$ attains negative values when\, $0 \leqq x \leqq 1$\, and that $f$ increases monotonically and boundlessly when $x$ increases from 1 to $\infty$.\, Because $f$ as a polynomial function is also continuous, we infer that the equation
\begin{align}
x^q-x-a = 0
\end{align}
has exactly one \PMlinkescapetext{positive} \PMlinkname{root}{Equation} \, $x = M > 1$,\, and that $f$ is negative for\, $0 < x < 1$\, and positive for\, $x > M$.\, Thus we can conclude by (2) that\, $x_n < M$\, for all values of $n$. 

The proven facts
$$x_1 < x_2 < x_3 < \ldots < x_n < \ldots < M$$
settle, by the theorem of the \PMlinkname{parent entry}{NondecreasingSequenceWithUpperBound}, that the sequence 
$$x_1,\,x_2,\,x_3,\,\ldots,\,x_n,\,\ldots$$
converges to a limit $x' \leqq M$.

Taking limits of both sides of (1) we see that $x' = \sqrt[q]{a+x'}$,\, i.e.\, $x'^q-x'-a = 0$,\, which means that\, $x' = M$,\, in other words: the limit of the sequence is the only \PMlinkescapetext{positive root} $M$ of the equation (3).\\

\begin{thebibliography}{9}
\bibitem{NP}{\sc E. Lindel\"of:} {\em Johdatus korkeampaan analyysiin}. Nelj\"as painos.\, Werner S\"oderstr\"om Osakeyhti\"o, Porvoo ja Helsinki (1956).
\end{thebibliography}


%%%%%
%%%%%
\end{document}
