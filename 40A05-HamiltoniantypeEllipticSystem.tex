\documentclass[12pt]{article}
\usepackage{pmmeta}
\pmcanonicalname{HamiltoniantypeEllipticSystem}
\pmcreated{2013-03-11 19:28:32}
\pmmodified{2013-03-11 19:28:32}
\pmowner{linor}{11198}
\pmmodifier{}{0}
\pmtitle{Hamiltonian-type elliptic system}
\pmrecord{1}{50088}
\pmprivacy{1}
\pmauthor{linor}{0}
\pmtype{Definition}

%none for now
\begin{document}
\documentclass[12pt,leqno]{article}
\usepackage{amssymb}
\usepackage{color}

\newcommand{\be}{\begin{equation}}
\newcommand{\ee}{\end{equation}}
\newcommand{\dk}{d\sigma_{\xi}}
\newcommand{\dx}{d\sigma_{x}}
\newcommand{\nd}{\frac{ \partial}{ \partial n}}
\newcommand{\ndk}{\disfrac{\textstyle \partial}{\textstyle \partial n_{ \xi}}}
\newcommand{\ndx}{\disfrac{\textstyle \partial}{\textstyle \partial n_{ x}}}
\newcommand{\ik}{\int_{ \Gamma}}
\newcommand{\ts}{\textstyle}


\begin{document}

A hamiltonian-type elliptic system is of the form
    $$ 
       (HS) \;\; \left \{ \begin{array}{ll}
              -\Delta u= G_v(x; u,v),  &\, x\in \Omega \nonumber \\
              -\Delta v= G_u(x; u,v),  &\, x\in \Omega \\
                u=v=0 \; \mbox{or} \;
           \frac{\partial u}{\partial n}= \frac{\partial v}{\partial n}=0
           &\, x\in \partial \Omega
         \end{array}  \right.
   $$
where $\Omega\subset {\mathbb R}^N (N\ge 1)$ is an open bounded domain,
$G(x; u,v)\in \mathcal{C}^{1}(\overline{\Omega} \times
\mathbb{R}^2; \mathbb{R})$ in the variables $(u,v) \in \mathbb{R}^2$ with
$\nabla G =(G_u, G_v)$. 


\end{document}
%%%%%
\end{document}
