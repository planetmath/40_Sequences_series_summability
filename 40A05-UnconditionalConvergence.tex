\documentclass[12pt]{article}
\usepackage{pmmeta}
\pmcanonicalname{UnconditionalConvergence}
\pmcreated{2013-03-22 15:29:57}
\pmmodified{2013-03-22 15:29:57}
\pmowner{kompik}{10588}
\pmmodifier{kompik}{10588}
\pmtitle{unconditional convergence}
\pmrecord{11}{37358}
\pmprivacy{1}
\pmauthor{kompik}{10588}
\pmtype{Definition}
\pmcomment{trigger rebuild}
\pmclassification{msc}{40A05}
\pmsynonym{unconditionally convergent}{UnconditionalConvergence}
\pmrelated{AbsoluteConvergence}
\pmrelated{ConditionallyConvergentSeriesOfRealNumbersCanBeRearrangedToConvergeToAnyNumber}

% this is the default PlanetMath preamble.  as your knowledge
% of TeX increases, you will probably want to edit this, but
% it should be fine as is for beginners.

% almost certainly you want these
\usepackage{amssymb}
\usepackage{amsmath}
\usepackage{amsfonts}

% used for TeXing text within eps files
%\usepackage{psfrag}
% need this for including graphics (\includegraphics)
%\usepackage{graphicx}
% for neatly defining theorems and propositions
%\usepackage{amsthm}
% making logically defined graphics
%%%\usepackage{xypic}

% there are many more packages, add them here as you need them

% define commands here
\newcommand{\sR}[0]{\mathbb{R}}
\newcommand{\sN}[0]{\mathbb{N}}
\begin{document}
A series $\displaystyle{\sum_{n=1}^\infty x_n}$ in a Banach space $X$ is \emph{unconditionally convergent}
if for every permutation $\sigma: \sN \to \sN$ the series $\displaystyle{\sum_{n=1}^\infty x_{\sigma(n)}}$
converges.

Alternatively, for every chain of finite subsets $S_1\subseteq S_2\subseteq\cdots$ of $\mathbb{N}$, the partial sums
$$\sum_{k\in S_1} x_k,\mbox{ }\sum_{k\in S_2} x_k,\mbox{ },\ldots$$
converges.  The trick to see this equivalence is to realize two facts: 1. every subsequence of a convergent sequence is convergent, and 2. every chain $\lbrace S_i\rbrace$ can be enlarged to a maximal chain  $\lbrace T_i\rbrace$, such that $|T_i|=i$.  Then the series indexed by $\{S_i\}$ is a subseries indexed by $\{T_i\}$, which is a subseries of a permutation of the original convergent series.

Yet a third \PMlinkname{equivalent}{Equivalent3} definition is given as follows: A series is unconditionally convergent if
for every sequence $(\varepsilon_n)_{n=1}^\infty$, with $\varepsilon_n\in\{\pm 1\}$, the
series $\displaystyle{\sum_{n=1}^\infty \varepsilon_n x_n}$ converges.

Every absolutely convergent series is unconditionally convergent, the converse implication does not hold in general.

When $X=\sR^n$ then by a famous theorem of Riemann $(\sum x_n)$ is unconditionally convergent if and only if it is absolutely convergent.  

\begin{thebibliography}{5}

\bibitem{knopp2} K.~Knopp: {\em {Theory and application of infinite series}}.

\bibitem{knopp} K.~Knopp: {\em {Infinite sequences and series}}.

\bibitem{wojtaszczyk} P.~Wojtaszczyk: {\em {Banach spaces for analysts}}.

\bibitem{heil}
Ch.~Heil: \PMlinkexternal{A basis theory primer}{http://www.math.gatech.edu/~heil/papers/bases.pdf}.


\end{thebibliography}
%%%%%
%%%%%
\end{document}
