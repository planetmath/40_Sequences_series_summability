\documentclass[12pt]{article}
\usepackage{pmmeta}
\pmcanonicalname{Series}
\pmcreated{2013-03-22 12:41:35}
\pmmodified{2013-03-22 12:41:35}
\pmowner{mathwizard}{128}
\pmmodifier{mathwizard}{128}
\pmtitle{series}
\pmrecord{7}{32973}
\pmprivacy{1}
\pmauthor{mathwizard}{128}
\pmtype{Definition}
\pmcomment{trigger rebuild}
\pmclassification{msc}{40-00}
\pmrelated{AbsoluteConvergence}
\pmrelated{HarmonicNumber}
\pmrelated{CompleteUltrametricField}
\pmrelated{Summation}
\pmrelated{PrimeHarmonicSeries}

% this is the default PlanetMath preamble.  as your knowledge
% of TeX increases, you will probably want to edit this, but
% it should be fine as is for beginners.

% almost certainly you want these
\usepackage{amssymb}
\usepackage{amsmath}
\usepackage{amsfonts}

% used for TeXing text within eps files
%\usepackage{psfrag}
% need this for including graphics (\includegraphics)
%\usepackage{graphicx}
% for neatly defining theorems and propositions
%\usepackage{amsthm}
% making logically defined graphics
%%%\usepackage{xypic}

% there are many more packages, add them here as you need them

% define commands here
\begin{document}
Given a sequence of numbers (real or complex) $\{a_n\}$ we define a sequence of \textit{partial
sums} $\{S_N\}$, where $S_N=\sum_{n=1}^N a_n$. This sequence is called the \textit{series} with \textit{terms} $a_n$. We define the \textit{sum of the series}
$\sum_{n=1}^\infty a_n$ to be the limit of these partial sums. More precisely
\[
 \sum_{n=1}^\infty a_n = \lim_{N\to\infty} S_n
    = \lim_{N\to\infty} \sum_{n=1}^N a_n.
\]
In a context where this distinction does not matter much (this is usually the case) one identifies a series with its sum, if the latter exists.

Traditionally, as above, series are infinite sums of real numbers. However,
the formal constraints on the terms $\{a_n\}$ are much less strict. We need
only be able to add the terms and take the limit of partial sums. So in full
generality the terms could be complex numbers or even elements of certain rings,
fields, and vector spaces.
%%%%%
%%%%%
\end{document}
