\documentclass[12pt]{article}
\usepackage{pmmeta}
\pmcanonicalname{SummationByParts}
\pmcreated{2013-03-22 16:28:10}
\pmmodified{2013-03-22 16:28:10}
\pmowner{rm50}{10146}
\pmmodifier{rm50}{10146}
\pmtitle{summation by parts}
\pmrecord{8}{38632}
\pmprivacy{1}
\pmauthor{rm50}{10146}
\pmtype{Theorem}
\pmcomment{trigger rebuild}
\pmclassification{msc}{40A05}
\pmclassification{msc}{40D05}
\pmsynonym{partial summation}{SummationByParts}

% this is the default PlanetMath preamble.  as your knowledge
% of TeX increases, you will probably want to edit this, but
% it should be fine as is for beginners.

% almost certainly you want these
\usepackage{amssymb}
\usepackage{amsmath}
\usepackage{amsfonts}

% used for TeXing text within eps files
%\usepackage{psfrag}
% need this for including graphics (\includegraphics)
%\usepackage{graphicx}
% for neatly defining theorems and propositions
%\usepackage{amsthm}
% making logically defined graphics
%%%\usepackage{xypic}

% there are many more packages, add them here as you need them

% define commands here
\newtheorem{thm}{Theorem}
\newtheorem{cor}[thm]{Corollary}
\newtheorem{lem}[thm]{Lemma}
\newtheorem{prop}[thm]{Proposition}

\begin{document}
The following corollaries apply Abel's lemma to allow estimation of certain bounded sums:

\begin{cor} (Summation by parts)
\newline
Let $\{a_i\},\{b_i\}$ be sequences of complex numbers. Suppose the partial sums of the $a_i$ are bounded in magnitude by $h$, that $\sum_0^{\infty} |b_i-b_{i+1}|$ converges, and that $\lim_{i\to\infty} b_i=0$. Then $\sum_0^{\infty} a_i b_i$ converges, and
\[\left|\sum_0^{\infty}a_i b_i\right|\leq h\sum_0^{\infty}|b_i-b_{i+1}|\]
\end{cor}
\textbf{Proof. }
By Abel's lemma,
\[\sum_{i=0}^N a_ib_i = \sum_{i=0}^{N-1} A_i(b_i-b_{i+1}) + A_Nb_N\]
so that
\begin{align*}
\left\lvert \sum_{i=0}^N a_ib_i\right\rvert &= \left\lvert \sum_{i=0}^{N-1} A_i(b_i-b_{i+1}) + A_Nb_N\right\rvert \leq \sum_{i=0}^{N-1}\left\lvert A_i(b_i-b_{i+1})\right\rvert + \left\lvert A_Nb_N\right\rvert\\
&\leq h\sum_{i=0}^{N-1}\left\lvert b_i-b_{i+1}\right\rvert + h\left\lvert b_N\right\rvert
\end{align*}
The condition that the $b_i\to 0$ is easily seen to imply that the sequence $\left\lvert \sum_{i=0}^N a_ib_i\right\rvert$ is Cauchy hence convergent, so that
\[\left\lvert \sum_{i=0}^\infty a_ib_i\right\rvert \leq h\sum_{i=0}^{\infty} \left\lvert b_i-b_{i+1}\right\rvert\]
since $b_N\to 0$.


\begin{cor} (Summation by parts for real sequences)
\newline
Let $\{a_i\}$ be a sequence of complex numbers. Suppose the partial sums are bounded in magnitude by $h$. Let $\{b_i\}$ be a sequence of decreasing positive real numbers such that $\lim_{i\to\infty} b_i=0$. Then $\sum_1^{\infty} a_ib_i$ converges, and $|\sum_1^{\infty} a_ib_i|\leq hb_1$.
\end{cor}
\textbf{Proof. } This follows immediately from the above, since $|b_i-b_{i+1}|=b_i-b_{i+1}$.

%%%%%
%%%%%
\end{document}
