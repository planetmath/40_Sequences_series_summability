\documentclass[12pt]{article}
\usepackage{pmmeta}
\pmcanonicalname{SemilinearBiharmonicProblem1}
\pmcreated{2013-03-11 19:28:39}
\pmmodified{2013-03-11 19:28:39}
\pmowner{linor}{11198}
\pmmodifier{}{0}
\pmtitle{Semilinear biharmonic problem}
\pmrecord{1}{50089}
\pmprivacy{1}
\pmauthor{linor}{0}
\pmtype{Definition}

\endmetadata

%none for now
\begin{document}
\documentclass[12pt,leqno]{article}
\usepackage{amssymb}
\usepackage{color}

\newcommand{\be}{\begin{equation}}
\newcommand{\ee}{\end{equation}}
\newcommand{\dk}{d\sigma_{\xi}}
\newcommand{\dx}{d\sigma_{x}}
\newcommand{\nd}{\frac{ \partial}{ \partial n}}
\newcommand{\ndk}{\disfrac{\textstyle \partial}{\textstyle \partial n_{ \xi}}}
\newcommand{\ndx}{\disfrac{\textstyle \partial}{\textstyle \partial n_{ x}}}
\newcommand{\ik}{\int_{ \Gamma}}
\newcommand{\ts}{\textstyle}


\begin{document}

A semilinear biharmonic problem with so-called Navier boundary conditions
is 
    $$ 
       (BH) \;\; \left \{ \begin{array}{ll}
              \Delta^2 v= f(x, v)  &\, x\in \Omega \\
                v=\Delta v=0        &\, x\in \partial \Omega
         \end{array}  \right.
   $$
where $\Omega\subset {\mathbb R}^N (N\ge 1)$ is an open bounded domain,
$f(x, v)\in \mathcal{C}^{1}(\overline{\Omega} \times
\mathbb{R}; \mathbb{R})$ in $v \in \mathbb{R}$. 

Numerical results for the case $\Omega = (-2,2)\times (-2,2) $ and 
$f(x,v)=v^p$: p=0.1, 0.7, 1.3, 7. Note that $u=-\Delta v$.


\end{document}
%%%%%
\end{document}
