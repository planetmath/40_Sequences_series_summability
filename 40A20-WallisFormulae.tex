\documentclass[12pt]{article}
\usepackage{pmmeta}
\pmcanonicalname{WallisFormulae}
\pmcreated{2013-03-22 12:56:15}
\pmmodified{2013-03-22 12:56:15}
\pmowner{rspuzio}{6075}
\pmmodifier{rspuzio}{6075}
\pmtitle{Wallis formulae}
\pmrecord{8}{33294}
\pmprivacy{1}
\pmauthor{rspuzio}{6075}
\pmtype{Definition}
\pmcomment{trigger rebuild}
\pmclassification{msc}{40A20}
\pmclassification{msc}{40A10}
\pmrelated{Pi}
\pmrelated{ReductionFormulasForIntegrationOfPowers}

% this is the default PlanetMath preamble.  as your knowledge
% of TeX increases, you will probably want to edit this, but
% it should be fine as is for beginners.

% almost certainly you want these
\usepackage{amssymb}
\usepackage{amsmath}
\usepackage{amsfonts}

% used for TeXing text within eps files
%\usepackage{psfrag}
% need this for including graphics (\includegraphics)
%\usepackage{graphicx}
% for neatly defining theorems and propositions
%\usepackage{amsthm}
% making logically defined graphics
%%%\usepackage{xypic}

% there are many more packages, add them here as you need them

% define commands here
\begin{document}
Wallis' formula expresses $\pi$ as an infinite product:
$$\frac{\pi}{2} = \prod_{n = 1}^{\infty} \frac{4n^{2}}{4n^{2} - 1} = \frac{2}{1}\frac{2}{3}\frac{4}{3}\frac{4}{5} \cdots$$

It may be derived by taking the limit as $n \to \infty$ of the ratio of the following
two integrals.
$$\int_{0}^{\frac{\pi}{2}} \sin^{2n}x dx = \frac{1 \cdot 3 \cdots (2n - 1)}{2 \cdot 4 \cdots 2n} \frac{\pi}{2}$$

$$\int_{0}^{\frac{\pi}{2}} \sin^{2n + 1}x dx = \frac{2 \cdot 4 \cdots 2n}{3 \cdot 5 \cdots (2n + 1)}$$


%%%%%
%%%%%
\end{document}
