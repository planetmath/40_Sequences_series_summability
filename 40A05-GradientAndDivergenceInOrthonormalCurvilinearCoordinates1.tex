\documentclass[12pt]{article}
\usepackage{pmmeta}
\pmcanonicalname{GradientAndDivergenceInOrthonormalCurvilinearCoordinates1}
\pmcreated{2013-03-11 19:26:23}
\pmmodified{2013-03-11 19:26:23}
\pmowner{swapnizzle}{13346}
\pmmodifier{}{0}
\pmtitle{Gradient and Divergence in Orthonormal Curvilinear Coordinates}
\pmrecord{1}{50079}
\pmprivacy{1}
\pmauthor{swapnizzle}{0}
\pmtype{Definition}

%none for now
\begin{document}
\documentclass[11pt]{article}
\usepackage{amssymb}
\usepackage{amsmath}
\usepackage{amsthm}
\usepackage{amsfonts}
\usepackage{array}
\usepackage[mathcal]{eucal}
\usepackage{xy}
\textheight 9in
\textwidth 6.5in
\oddsidemargin 0in
\evensidemargin 0in
\topmargin 0in
\headheight 0in
\headsep 0in
\title{Gradient and Divergence in Orthonormal Curvilinear Coordinates}
\author{Swapnil Sunil Jain}
\date{Aug 7, 2006}
\begin{document}
\maketitle
\subsection*{Gradient in Curvilinear Coordinates}
In rectangular coordinates (where $f = f(x,y,z)$), an infinitesimal length vector $d\vec{l}$ is given by
\begin{eqnarray*}
&& d\vec{l} = dx\hat{x} + dy\hat{y} + dz\hat{z}
\end{eqnarray*} 
the gradient is given by 
\begin{eqnarray*}
&& \nabla = \frac{\partial}{\partial x}\hat{x} + \frac{\partial}{\partial y}\hat{y} + \frac{\partial}{\partial z}\hat{z}
\end{eqnarray*} 
and the differential change in the output is given by 
\begin{eqnarray*}
&& df = \nabla f \circ d\vec{l} = \frac{\partial f}{\partial x}dx + \frac{\partial f}{\partial y}dy + \frac{\partial f}{\partial z}dz 
\end{eqnarray*} 
Similarly in orthonormal curvilinear coordinates ( where $f = f(q_1,q_2,q_3)$), the infinitesimal length vector is given by\footnote[1]{See my article $\emph{Unit Vectors in Curvilinear Coordinates}$ for an insight into this expression.} 
\begin{eqnarray*}
&& d\vec{l} = h_1 dq_1 \hat{q}_1 + h_2 dq_2 \hat{q}_2 + h_3 dq_3 \hat{q}_3
\end{eqnarray*} 
where
\begin{eqnarray*}
&& h_i = \sqrt{\sum_k \Big(\frac{\partial \vec{x}_k}{\partial q_i}\Big)^2} \mbox{ and }  \hat{q}_i = \frac{1}{h_i} \Big(\frac{\partial \vec{x}_k}{\partial q_i}\Big) \mbox{ for } i\in {1,2,3}
\end{eqnarray*}  
So if 
\begin{eqnarray*}
&& \nabla = \alpha \frac{\partial}{\partial q_1} \hat{q}_1 + \beta \frac{\partial }{\partial q_2} \hat{q}_2 + \gamma \frac{\partial }{\partial q_3} \hat{q}_3
\end{eqnarray*} 
then since we know that 
\begin{eqnarray*}
&& df =  \frac{\partial f}{\partial q_1} dq_1 + \frac{\partial f}{\partial q_2} dq_2 + \frac{\partial f}{\partial q_3} dq_3
\end{eqnarray*} 
and 
\begin{eqnarray*}
&& df = \nabla F \circ d\vec{l} = \alpha h_1 \frac{\partial f}{\partial q_1} dq_1 + \beta h_2 \frac{\partial f}{\partial q_2} dq_2 + \gamma h_3 \frac{\partial f}{\partial q_3} dq_3
\end{eqnarray*} 
this implies that 
\begin{eqnarray*}
\alpha = \frac{1}{h_i}; \beta = \frac{1}{h_2}; \gamma = \frac{1}{h_3}
\end{eqnarray*}
Hence, 
\begin{eqnarray*}
\nabla &=& \frac{1}{h_1} \frac{\partial}{\partial q_1} \hat{q}_1 +  \frac{1}{h_2} \frac{\partial }{\partial q_2} \hat{q}_2 +  \frac{1}{h_3} \frac{\partial }{\partial q_3} \hat{q}_3 \\
&=& \sum_i \frac{1}{h_i} \frac{\partial}{\partial q_i} \hat{q}_i
\end{eqnarray*}
 
\subsection*{Divergence in Curvilinear Coordinates}
In the previous section we concluded that in curvilinear coordinates, the gradient operator $\nabla$ is given by 
\begin{eqnarray*}
&& \nabla = \sum_i \frac{1}{h_i} \frac{\partial}{\partial q_i} \hat{q}_i
\end{eqnarray*} 
Then for $\vec{F} = F_1\hat{q}_1 + F_2\hat{q}_2 + F_3\hat{q}_3$, the divergence of $\vec{F}$ is given by 
\begin{eqnarray*}
&& \nabla \circ \vec{F} = \Big( \sum_i \frac{1}{h_i} \frac{\partial}{\partial q_i} \hat{q}_i \Big) \circ \vec{F}
\end{eqnarray*} 
which is not equal to 
\begin{eqnarray*}
\Big( \sum_i \frac{1}{h_i} \frac{\partial}{\partial q_i} \hat{q}_i \Big) \circ \vec{F} \neq \sum_i \frac{1}{h_i} \frac{\partial F_i}{\partial q_i}
\end{eqnarray*} 
as one would think! The real expression can be derived the following way, 

\begin{eqnarray*}
&& = \sum_i \Big[ \Big(\frac{1}{h_i} \frac{\partial}{\partial q_i} \hat{q}_i \Big) \circ \vec{F}\Big] \\
&& = \sum_i \Big[ \Big(\frac{1}{h_i} \hat{q}_i \Big) \circ \Big( \frac{\partial \vec{F}}{\partial q_i} \Big) \Big] \\
&& = \sum_i \Big[ \Big(\frac{1}{h_i} \hat{q}_i \Big) \circ \Big( \frac{\partial}{\partial q_i} \Big(\sum_j F_j \hat{q}_j\Big) \Big) \Big]
\end{eqnarray*} 
\begin{eqnarray*}
&& = \sum_i \Big[ \Big(\frac{1}{h_i} \hat{q}_i \Big) \circ \Big( \sum_j \frac{\partial}{\partial q_i} \Big(F_j \hat{q}_j\Big) \Big) \Big] \\
&& = \sum_i \Big[ \Big(\frac{1}{h_i} \hat{q}_i \Big) \circ \Big( \sum_j \hat{q}_j \frac{\partial F_j}{\partial q_i} +  F_j \frac{\partial \hat{q}_j}{\partial q_i} \Big) \Big]
\end{eqnarray*} 
\begin{eqnarray*}
&& = \sum_i \Big[ \Big(\frac{1}{h_i} \hat{q}_i \Big) \circ \Big( \sum_j \hat{q}_j \frac{\partial F_j}{\partial q_i} +  \sum_j F_j \frac{\partial \hat{q}_j}{\partial q_i} \Big) \Big] \\
&& = \sum_i \Big[  \Big(\frac{1}{h_i} \hat{q}_i \Big) \circ \sum_j \hat{q}_j \frac{\partial F_j}{\partial q_i} +  \Big(\frac{1}{h_i} \hat{q}_i \Big)\circ \sum_j F_j \frac{\partial \hat{q}_j}{\partial q_i} \Big]
\end{eqnarray*} 

\begin{eqnarray*}
&& = \underbrace{ \sum_i \Big[ \Big(\frac{1}{h_i} \hat{q}_i \Big) \circ \sum_j \hat{q}_j \frac{\partial F_j}{\partial q_i} \Big]}_{\mbox{call it A}} + \underbrace{\sum_i \Big[ \Big(\frac{1}{h_i} \hat{q}_i \Big)\circ \sum_j F_j \frac{\partial \hat{q}_j}{\partial q_i} \Big]}_{\mbox{call it B}}
\end{eqnarray*} 

\begin{eqnarray*}
A &=& \sum_i \Big[ \Big(\frac{1}{h_i} \hat{q}_i \Big) \circ \sum_j \hat{q}_j \frac{\partial F_j}{\partial q_i} \Big] \\
&=& \sum_i \Big[ \frac{1}{h_i}  \sum_j \underbrace{(\hat{q}_i \circ \hat{q}_j)}_{\delta_{ij}} \frac{\partial F_j}{\partial q_i} \Big] \\
&=& \sum_i \frac{1}{h_i}\frac{\partial F_i}{\partial q_i}
\end{eqnarray*} 

\begin{eqnarray*}
B &=& \sum_i \Big[ \Big(\frac{1}{h_i} \hat{q}_i \Big) \circ \sum_j F_j \frac{\partial \hat{q}_j}{\partial q_i} \Big]
\end{eqnarray*} 

Using the following equality\footnote[2]{The proof of this identity is left as an exercise for the reader.} 
\begin{eqnarray*}
\frac{\partial \hat{q}_j}{\partial q_i} = \frac{\hat{q}_i}{h_j}\frac{\partial h_i}{\partial q_j} \qquad \forall i \neq j
\end{eqnarray*} 
we can write $B$ as
\begin{eqnarray*}
B &=& \sum_i \Big[ \Big(\frac{1}{h_i} \hat{q}_i \Big)\circ \sum_j F_j \Big( \hat{q}_i \frac{1}{h_j}\frac{\partial h_i}{\partial q_j}\Big) \Big] \qquad \forall i \neq j \\
&=& \sum_i \Big[ \frac{1}{h_i} \sum_j F_j \underbrace{(\hat{q}_i \circ \hat{q}_i)}_{1} \frac{1}{h_j}\frac{\partial h_i}{\partial q_j} \Big] \qquad \forall i \neq j \\
&=& \sum_i \Big[ \frac{1}{h_i} \sum_j F_j \frac{1}{h_j}\frac{\partial h_i}{\partial q_j} \Big] \qquad \forall i \neq j \\
&=& \sum_{i\neq j} \frac{F_j}{h_j h_i}\frac{\partial h_i}{\partial q_j} \\
&=& \sum_{i\neq 1} \frac{F_1}{h_1 h_i}\frac{\partial h_i}{\partial q_1} + \sum_{i\neq 2} \frac{F_2}{h_2 h_i}\frac{\partial h_i}{\partial q_2} + \sum_{i\neq 3} \frac{F_3}{h_3 h_i}\frac{\partial h_i}{\partial q_3}
\end{eqnarray*} 

\begin{eqnarray*}
\Rightarrow \nabla \circ \vec{F} &=& A + B \\
&=& \sum_i \frac{1}{h_i}\frac{\partial F_i}{\partial q_i} + \sum_{i\neq 1} \frac{F_1}{h_1 h_i}\frac{\partial h_i}{\partial q_1} + \sum_{i\neq 2} \frac{F_2}{h_2 h_i}\frac{\partial h_i}{\partial q_2} + \sum_{i\neq 3} \frac{F_3}{h_3 h_i}\frac{\partial h_i}{\partial q_3} \\
&=& \Big[ \frac{1}{h_1}\frac{\partial F_1}{\partial q_1} + \frac{1}{h_2}\frac{\partial F_2}{\partial q_2} + \frac{1}{h_3}\frac{\partial F_3}{\partial q_3} \Big] \\
&& + \Big[ \frac{F_1}{h_1 h_2}\frac{\partial h_2}{\partial q_1} + \frac{F_1}{h_1 h_3}\frac{\partial h_3}{\partial q_1} \Big] \\
&& + \Big[ \frac{F_2}{h_2 h_1}\frac{\partial h_1}{\partial q_2} + \frac{F_2}{h_2 h_3}\frac{\partial h_3}{\partial q_2} \Big] \\
&& + \Big[ \frac{F_3}{h_3 h_1}\frac{\partial h_1}{\partial q_3} + \frac{F_3}{h_3 h_2}\frac{\partial h_2}{\partial q_3} \Big]
\end{eqnarray*} 

Collecting similar terms together we get, 
\begin{eqnarray*}
\nabla \circ \vec{F} &=& \Big[ \frac{1}{h_1}\frac{\partial F_1}{\partial q_1} +  \frac{F_1}{h_1 h_2}\frac{\partial h_2}{\partial q_1} + \frac{F_1}{h_1 h_3}\frac{\partial h_3}{\partial q_1} \Big] \\
&& + \Big[ \frac{1}{h_2}\frac{\partial F_2}{\partial q_2} + \frac{F_2}{h_2 h_1}\frac{\partial h_1}{\partial q_2} + \frac{F_2}{h_2 h_3}\frac{\partial h_3}{\partial q_2} \Big] \\
&& + \Big[ \frac{1}{h_3}\frac{\partial F_3}{\partial q_3} + \frac{F_3}{h_3 h_1}\frac{\partial h_1}{\partial q_3} + \frac{F_3}{h_3 h_2}\frac{\partial h_2}{\partial q_3}\Big]
\end{eqnarray*} 

If we define $\Omega \equiv \Pi_i h_i$, we can further write the above expression as
\begin{eqnarray*}
\nabla \circ \vec{F} &=& \Big[ \frac{h_2 h_3}{\Omega}\frac{\partial F_1}{\partial q_1} +  \frac{F_1 h_3}{\Omega}\frac{\partial h_2}{\partial q_1} + \frac{F_1 h_2}{\Omega}\frac{\partial h_3}{\partial q_1} \Big] \\
&& + \Big[ \frac{h_1 h_3}{\Omega}\frac{\partial F_2}{\partial q_2} + \frac{F_2 h_3}{\Omega}\frac{\partial h_1}{\partial q_2} + \frac{h_1 F_2}{\Omega}\frac{\partial h_3}{\partial q_2} \Big] \\
&& + \Big[ \frac{h_1 h_2}{\Omega}\frac{\partial F_3}{\partial q_3} + \frac{h_2 F_3 }{\Omega}\frac{\partial h_1}{\partial q_3} + \frac{h_1 F_3 }{\Omega}\frac{\partial h_2}{\partial q_3}\Big] \\
&=& \frac{1}{\Omega} \Bigg(\Big[\frac{\partial F_1}{\partial q_1} h_2 h_3 + F_1 h_3\frac{\partial h_2}{\partial q_1} + F_1 h_2 \frac{\partial h_3}{\partial q_1} \Big] \\
&& + \Big[ h_1 \frac{\partial F_2}{\partial q_2} h_3 + \frac{\partial h_1}{\partial q_2} F_2 h_3 + h_1 F_2 \frac{\partial h_3}{\partial q_2} \Big] \\
&& + \Big[ h_1 h_2 \frac{\partial F_3}{\partial q_3} + \frac{\partial h_1}{\partial q_3}h_2 F_3 + h_1 \frac{\partial h_2}{\partial q_3} F_3 \Big] \Bigg) \\
&=& \frac{1}{\Omega} \Big(\frac{\partial}{\partial q_1}(F_1 h_2 h_3)  +  \frac{\partial}{\partial q_2}(h_1 F_2 h_3) + \frac{\partial}{\partial q_3}(h_1 h_2 F_3) \Big) 
\end{eqnarray*}
Hence,
\begin{eqnarray*}
\nabla \circ \vec{F} &=& \frac{1}{\Omega} \sum_i \frac{\partial}{\partial q_i}\Big(\frac{\Omega}{h_i} F_i \Big) \quad \mbox{ where } \Omega = \Pi_i h_i 
\end{eqnarray*}

\end{document}
%%%%%
\end{document}
