\documentclass[12pt]{article}
\usepackage{pmmeta}
\pmcanonicalname{SequenceDeterminingConvergenceOfSeries}
\pmcreated{2013-03-22 19:06:54}
\pmmodified{2013-03-22 19:06:54}
\pmowner{pahio}{2872}
\pmmodifier{pahio}{2872}
\pmtitle{sequence determining convergence of series}
\pmrecord{6}{42008}
\pmprivacy{1}
\pmauthor{pahio}{2872}
\pmtype{Definition}
\pmcomment{trigger rebuild}
\pmclassification{msc}{40A05}
\pmrelated{LimitComparisonTest}

% this is the default PlanetMath preamble.  as your knowledge
% of TeX increases, you will probably want to edit this, but
% it should be fine as is for beginners.

% almost certainly you want these
\usepackage{amssymb}
\usepackage{amsmath}
\usepackage{amsfonts}

% used for TeXing text within eps files
%\usepackage{psfrag}
% need this for including graphics (\includegraphics)
%\usepackage{graphicx}
% for neatly defining theorems and propositions
 \usepackage{amsthm}
% making logically defined graphics
%%%\usepackage{xypic}

% there are many more packages, add them here as you need them

% define commands here

\theoremstyle{definition}
\newtheorem*{thmplain}{Theorem}

\begin{document}
\textbf{Theorem.}\, Let $a_1\!+\!a_2\!+\ldots$ be any series of real \PMlinkescapetext{terms} $a_n$.\, If the positive numbers $r_1,\,r_2,\,\ldots$\, are such that
\begin{align}
\lim_{n\to\infty}\frac{a_n}{r_n} \;=\; L \;\neq\, 0,
\end{align}
then the series converges simultaneously with the series $r_1\!+\!r_2\!+\ldots$\\


\emph{Proof.}\, In the case that the limit (1) is positive, the supposition implies that there is an integer $n_0$ such that 
\begin{align}
0.5L \;<\; \frac{a_n}{r_n} \;<\; 1.5L \quad \textrm{for  } n \geqq n_0.
\end{align}
Therefore
$$0 \;<\; 0.5Lr_n \;<\; a_n \;<\; 1.5Lr_n \quad \textrm{for all  } n \geqq n_0,$$
and since the series $\sum_{n=1}^\infty0.5Lr_n$ and $\sum_{n=1}^\infty1.5Lr_n$ converge simultaneously with the series $r_1\!+\!r_2\!+\ldots$, the comparison test guarantees that the same concerns the given series $a_1\!+\!a_2\!+\ldots$\\

The case where (1) is negative, whence we have
$$\lim_{n\to\infty}\frac{-a_n}{r_n} \;=\; -L > 0,$$
may be handled as above.\\


\textbf{Note.}\, For the case\, $L = 0$, see the limit comparison test.


%%%%%
%%%%%
\end{document}
