\documentclass[12pt]{article}
\usepackage{pmmeta}
\pmcanonicalname{ExampleOfSummationByParts}
\pmcreated{2013-03-22 17:27:56}
\pmmodified{2013-03-22 17:27:56}
\pmowner{pahio}{2872}
\pmmodifier{pahio}{2872}
\pmtitle{example of summation by parts}
\pmrecord{8}{39849}
\pmprivacy{1}
\pmauthor{pahio}{2872}
\pmtype{Example}
\pmcomment{trigger rebuild}
\pmclassification{msc}{40A05}
\pmrelated{ExampleOfTelescopingSum}
\pmrelated{SineIntegralInInfinity}
\pmrelated{ExampleOfSolvingTheHeatEquation}

\endmetadata

% this is the default PlanetMath preamble.  as your knowledge
% of TeX increases, you will probably want to edit this, but
% it should be fine as is for beginners.

% almost certainly you want these
\usepackage{amssymb}
\usepackage{amsmath}
\usepackage{amsfonts}

% used for TeXing text within eps files
%\usepackage{psfrag}
% need this for including graphics (\includegraphics)
%\usepackage{graphicx}
% for neatly defining theorems and propositions
 \usepackage{amsthm}
% making logically defined graphics
%%%\usepackage{xypic}

% there are many more packages, add them here as you need them

% define commands here

\theoremstyle{definition}
\newtheorem*{thmplain}{Theorem}

\begin{document}
\PMlinkescapeword{identity}
\textbf{Proposition.}  The series $\displaystyle\sum_{n=1}^\infty\frac{\sin{n\varphi}}{n}$ and $\displaystyle\sum_{n=1}^\infty\frac{\cos{n\varphi}}{n}$ converge for every complex value $\varphi$ which is not an even multiple of $\pi$.

{\em Proof.}  Let $\varepsilon$ be an arbitrary positive number.  One uses the \PMlinkescapetext{identities}
\begin{align}
\sin{\varphi}+\sin{2\varphi}+\ldots+\sin{n\varphi} = 
\frac{\sin(n+\frac{1}{2})\varphi-\sin\frac{\varphi}{2}}{2\sin\frac{\varphi}{2}},
\end{align}
\begin{align}
\cos{\varphi}+\cos{2\varphi}+\ldots+\cos{n\varphi} = 
\frac{-\cos(n+\frac{1}{2})\varphi+\cos\frac{\varphi}{2}}{2\sin\frac{\varphi}{2}},
\end{align}
proved in the entry ``\PMlinkname{example of telescoping sum}{ExampleOfTelescopingSum}''.  These give the \PMlinkescapetext{estimates}
$$|\sin{\varphi}+\sin{2\varphi}+\ldots+\sin{n\varphi}| \leqq \frac{2}{2|\sin\frac{\varphi}{2}|}\, :=\, K_\varphi,$$
$$|\cos{\varphi}+\cos{2\varphi}+\ldots+\cos{n\varphi}| \leqq \frac{2}{2|\sin\frac{\varphi}{2}|}\, :=\, K_\varphi$$
for any\, $n = 1,\,2,\,3,\,\ldots$.  
We want to apply to the series $\sum_{n=1}^\infty\frac{\cos{n\varphi}}{n}$ the \PMlinkname{Cauchy general convergence criterion}{CauchyCriterionForConvergence} for series.  Let us use here the short notation
$$\cos{N\varphi}+\cos{(N\!+\!1)\varphi}+\ldots+\cos{(N\!+\!p)\varphi} := S_{N,N+p}\quad 
(p = 0,\,1,\,2,\,\ldots).$$
Then, utilizing Abel's summation by parts, we obtain
$$\left|\sum_{n=N}^{N+P}\frac{\cos{n\varphi}}{n}\right| = 
\left|\sum_{p=0}^{P}\frac{1}{N\!+\!p}\cos{(N+p)\varphi}\right| = 
\left|\sum_{p=0}^{P-1}\left(\frac{1}{N\!+\!p}-\frac{1}{N\!+\!p\!+\!1}\right)S_{N,N+p}+\frac{1}{N\!+\!P}S_{N,N+P}\right| \leqq$$
$$\leqq \sum_{p=0}^{P-1}\left(\frac{1}{N\!+\!p}-\frac{1}{N\!+\!p\!+\!1}\right)|S_{N,N+P}|
+\frac{1}{N+P}|S_{N,N+P}| <$$ 
$$< \sum_{p=0}^{P-1}\left(\frac{1}{N\!+\!p}-\frac{1}{N\!+\!p\!+\!1}\right)\cdot2K_\varphi+\frac{1}{N\!+\!P}\cdot2K_\varphi\, =\, \frac{1}{N}\cdot2K_\varphi;$$
the last form is gotten by \PMlinkname{telescoping}{TelescopingSum} the preceding sum and before that by using the identity
$$S_{N,N+p} = [\cos\varphi+\cos2\varphi+\ldots+\cos(N\!+\!p)\varphi]-[\cos\varphi+\cos2\varphi+\ldots
+\cos(N\!-\!1)\varphi].$$
Thus we see that
$$\left|\sum_{n=N}^{N+P}\frac{\cos{n\varphi}}{n}\right| < \frac{2K_\varphi}{N} < \varepsilon$$
for all\, natural numbers $P$ as soon as\, $N > \frac{2K_\varphi}{\varepsilon}$.  According to the Cauchy criterion, the latter series is convergent for the mentioned values of $\varphi$.  The former series is handled similarly.



%%%%%
%%%%%
\end{document}
