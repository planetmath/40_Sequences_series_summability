\documentclass[12pt]{article}
\usepackage{pmmeta}
\pmcanonicalname{TheDiracDeltaFunction1}
\pmcreated{2013-03-11 19:30:28}
\pmmodified{2013-03-11 19:30:28}
\pmowner{swapnizzle}{13346}
\pmmodifier{}{0}
\pmtitle{The Dirac Delta Function}
\pmrecord{1}{50098}
\pmprivacy{1}
\pmauthor{swapnizzle}{0}
\pmtype{Definition}

%none for now
\begin{document}
\documentclass[11pt]{article}
\usepackage{amssymb}
\usepackage{amsmath}
\usepackage{amsthm}
\usepackage{amsfonts}
\usepackage{array}
\usepackage{txfonts}
\usepackage[mathcal]{eucal}
\usepackage{xy}
\textheight 9in
\textwidth 7in
\oddsidemargin 0in
\evensidemargin 0in
\topmargin 0in
\headheight 0in
\headsep 0in
\title{The Dirac Delta Function}
\author{Swapnil Sunil Jain}
\date{December 27, 2006}
\begin{document}
\maketitle

\subsection*{Definition}
The Dirac delta function can be defined as 
\begin{eqnarray*}
&& \delta(t) \triangleq \lim_{\epsilon \to 0} \frac{1}{\epsilon} \Pi\Big(\frac{t}{\epsilon}\Big)
\end{eqnarray*}
where $\Pi(t)$ is the pulse function. The delta function can also be defined as 
\begin{eqnarray*}
&& \delta(t) =   
\begin{cases}
0 & \mbox{for } t \neq 0 \\
\infty & \mbox{for } t = 0,
\end{cases}
\qquad \mbox{s.t } \int_{-\infty}^{+\infty} \delta(t) dt = 1 \\ 
\end{eqnarray*}
or it can also be defined as an operator s.t.
\begin{eqnarray*}
&& \delta(t)[f(t)] \triangleq \lim_{\epsilon \to 0}
\frac{1}{\epsilon} \int_{-\infty}^{+\infty} \Pi\Big(\frac{t}{\epsilon}\Big) f(t)dt \\
\end{eqnarray*}

\subsection*{Properties}
\begin{eqnarray*}
&& \mbox{1. } \int_{-\infty}^{\infty} f(t)\delta(t-a) dt = f(a)
\end{eqnarray*}
Proof: 
\begin{align*}
\int_{-\infty}^{\infty} f(t)\delta(t-a) dt &= \lim_{\epsilon \to 0} \Big( \int_{-\infty}^{a - \epsilon} f(t)\delta(t-a) dt + \int_{1-\epsilon}^{a+\epsilon} f(t)\delta(t-a) dt + \int_{a+\epsilon}^{\infty} f(t)\delta(t-a) dt \Big) \\
&= \lim_{\epsilon \to 0} \Big( \int_{a - \epsilon}^{a + \epsilon} f(t)\delta(t-a) dt \Big) \\
&= f(a) \lim_{\epsilon \to 0} \Big( \int_{a - \epsilon}^{a + \epsilon} \delta(t-a) dt \Big) = f(a)(1) = f(a) \\
\end{align*}

\begin{eqnarray*}
&& \mbox{2. } \int_{-\infty}^{\infty} f(t)\delta(t) dt = f(0)
\end{eqnarray*}
Proof: Readily seen if we set $a=0$ in Property \#1.

\begin{eqnarray*}
&& \mbox{3. } \int_{-\infty}^{\infty} f(t)\delta(at) dt = \frac{1}{a} f(0)
\end{eqnarray*}
Proof: Set $u = at$, then
\begin{align*}
\int_{-\infty}^{\infty} f(t)\delta(at) dt &= \int_{-\infty}^{\infty} f(\frac{u}{a})\delta(u) \frac{du}{a} = \frac{1}{a} f(0) \\
\end{align*}

\begin{eqnarray*}
&& \mbox{4. } \int_{-\infty}^{\infty} f(t)\delta(at - t_0) dt = \frac{1}{a}f(\frac{t_0}{a})
\end{eqnarray*}
Proof: Set $u = at - t_0$, then
\begin{align*}
\int_{-\infty}^{\infty} f(t)\delta(at - t_0) dt &= \int_{-\infty}^{\infty} f(\frac{u - t_0}{a})\delta(u) \frac{du}{a} = \frac{1}{a} f(\frac{t_0}{a}) \\
\end{align*}

\begin{eqnarray*}
&& \mbox{5. } \int_{-\infty}^{\infty} f(t)\delta(g(t)) dt = \sum_{ \{x_i\;|\;g(x_i)\;=\; 0\} } \frac{1}{|g'(x_i)|} f(x_i)
\end{eqnarray*}

\begin{eqnarray*}
&& \mbox{6. } \int_{-\infty}^{\infty} f(t)\delta'(t) dt = -f'(0)
\end{eqnarray*}
Proof: If we let $u = f(t)$ and $dv = \delta'(t) dt$, then, using integration by parts,
\begin{align*}
\int_{-\infty}^{\infty} f(t)\delta'(t) dt &= f(t)\delta(t){\Big|}_{-\infty}^{\infty} - \int_{-\infty}^{\infty} f'(t)\delta(t) dt = -f'(0) \\
\end{align*}

\end{document}
%%%%%
\end{document}
