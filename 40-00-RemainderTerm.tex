\documentclass[12pt]{article}
\usepackage{pmmeta}
\pmcanonicalname{RemainderTerm}
\pmcreated{2013-03-22 14:51:02}
\pmmodified{2013-03-22 14:51:02}
\pmowner{PrimeFan}{13766}
\pmmodifier{PrimeFan}{13766}
\pmtitle{remainder term}
\pmrecord{8}{36522}
\pmprivacy{1}
\pmauthor{PrimeFan}{13766}
\pmtype{Definition}
\pmcomment{trigger rebuild}
\pmclassification{msc}{40-00}
\pmsynonym{remainder}{RemainderTerm}
\pmsynonym{tail of series}{RemainderTerm}
%\pmkeywords{partial sum}
\pmrelated{SumOfSeries}

% this is the default PlanetMath preamble.  as your knowledge
% of TeX increases, you will probably want to edit this, but
% it should be fine as is for beginners.

% almost certainly you want these
\usepackage{amssymb}
\usepackage{amsmath}
\usepackage{amsfonts}

% used for TeXing text within eps files
%\usepackage{psfrag}
% need this for including graphics (\includegraphics)
%\usepackage{graphicx}
% for neatly defining theorems and propositions
%\usepackage{amsthm}
% making logically defined graphics
%%%\usepackage{xypic}

% there are many more packages, add them here as you need them

% define commands here
\begin{document}
Let $S_n$ be the $n^\mathrm{th}$ partial sum of the series\, $a_1\!+\!a_2\!+\cdots$\, with real or complex \PMlinkescapetext{terms} $a_n$ ($n = 1,\,2,\,\ldots$). 
\begin{itemize}
  \item If the series is convergent with sum $S$, then we call the \PMlinkescapetext{difference}\, $R_n := S\!-\!S_n$\, the  $n^\mathrm{th}$ {\em remainder term} or simply {\em remainder} of the series ($n = 1,\,2,\,\ldots$).\, Then\, $\lim_{n\to\infty}R_n = 0$.
  \item If there exists a number $s$ such that\, $\lim_{n\to\infty}(s\!-\!S_n) = 0$,\, then the series is convergent and its sum is $s$.
\end{itemize}
%%%%%
%%%%%
\end{document}
