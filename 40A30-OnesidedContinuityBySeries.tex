\documentclass[12pt]{article}
\usepackage{pmmeta}
\pmcanonicalname{OnesidedContinuityBySeries}
\pmcreated{2013-03-22 18:34:03}
\pmmodified{2013-03-22 18:34:03}
\pmowner{pahio}{2872}
\pmmodifier{pahio}{2872}
\pmtitle{one-sided continuity by series}
\pmrecord{10}{41291}
\pmprivacy{1}
\pmauthor{pahio}{2872}
\pmtype{Theorem}
\pmcomment{trigger rebuild}
\pmclassification{msc}{40A30}
\pmclassification{msc}{26A03}
\pmsynonym{one-sided continuity of series with terms one-sidedly continuous}{OnesidedContinuityBySeries}
\pmrelated{OneSidedContinuity}
\pmrelated{SumFunctionOfSeries}

\endmetadata

% this is the default PlanetMath preamble.  as your knowledge
% of TeX increases, you will probably want to edit this, but
% it should be fine as is for beginners.

% almost certainly you want these
\usepackage{amssymb}
\usepackage{amsmath}
\usepackage{amsfonts}

% used for TeXing text within eps files
%\usepackage{psfrag}
% need this for including graphics (\includegraphics)
%\usepackage{graphicx}
% for neatly defining theorems and propositions
 \usepackage{amsthm}
% making logically defined graphics
%%%\usepackage{xypic}

% there are many more packages, add them here as you need them

% define commands here

\theoremstyle{definition}
\newtheorem*{thmplain}{Theorem}

\begin{document}
\textbf{Theorem.}\, If the function series
\begin{align}
\sum_{n=1}^\infty f_n(x)
\end{align}
is uniformly convergent on the interval \,$[a,\,b]$,\, on which the \PMlinkescapetext{terms} $f_n(x)$ are continuous from the right or from the left, then the sum function $S(x)$ of the series has the same property.\\


{\em Proof.}\, Suppose that the terms $f_n(x)$ are continuous from the right.\, Let $\varepsilon$ be any positive number and
               $$S(x) \;:=\; S_n(x)+R_{n+1}(x),$$
where $S_n(x)$ is the $n^\mathrm{th}$ partial sum of (1) ($n \,=\, 1,\,2,\,\ldots$).\, The uniform convergence implies the existence of a number $n_\varepsilon$ such that on the whole interval we have
     $$|R_{n+1}(x)| < \frac{\varepsilon}{3} \quad \mathrm{when}\;\; n > n_\varepsilon. $$
Let now\, $n > n_\varepsilon$\, and\; $x_0,\, x_0\!+\!h \in [a,\,b]$\, with\, $h > 0$.\, Since every $f_n(x)$ is continuous from the right in $x_0$, the same is true for the finite sum $S_n(x)$, and therefore there exists a number $\delta_\varepsilon$ such that
             $$|S_n(x_0\!+\!h)-S_n(x_0)| < \frac{\varepsilon}{3} \quad \mathrm{when}\;\; 0 < h < \delta_\varepsilon.$$
Thus we obtain that
\begin{align*}
|S(x_0\!+\!h)-S(x_0)| & \;=\; |[S_n(x_0\!+\!h)-S_n(x_0)]+R_{n+1}(x_0\!+\!h)-R_{n+1}(x_0|\\
                      & \;\leqq\; |S_n(x_0\!+\!h)-S_n(x_0)|+|R_{n+1}(x_0\!+\!h)|+|R_{n+1}(x_0)|\\
                      & \;<\; \frac{\varepsilon}{3}\!+\!\frac{\varepsilon}{3}\!+\!\frac{\varepsilon}{3} \;=\; \varepsilon
\end{align*}
as soon as
$$0 < h < \delta_\varepsilon.$$
This means that $S$ is continuous from the right in an arbitrary point $x_0$ of\, $[a,\,b]$.

Analogously, one can prove the assertion concerning the continuity from the left.
%%%%%
%%%%%
\end{document}
