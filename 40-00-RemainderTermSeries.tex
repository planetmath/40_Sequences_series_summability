\documentclass[12pt]{article}
\usepackage{pmmeta}
\pmcanonicalname{RemainderTermSeries}
\pmcreated{2014-05-16 21:09:46}
\pmmodified{2014-05-16 21:09:46}
\pmowner{pahio}{2872}
\pmmodifier{pahio}{2872}
\pmtitle{remainder term series}
\pmrecord{13}{88073}
\pmprivacy{1}
\pmauthor{pahio}{2872}
\pmtype{Definition}
\pmclassification{msc}{40-00}

% this is the default PlanetMath preamble.  as your knowledge
% of TeX increases, you will probably want to edit this, but
% it should be fine as is for beginners.

% almost certainly you want these
\usepackage{amssymb}
\usepackage{amsmath}
\usepackage{amsfonts}

% used for TeXing text within eps files
%\usepackage{psfrag}
% need this for including graphics (\includegraphics)
%\usepackage{graphicx}
% for neatly defining theorems and propositions
% \usepackage[utf8]{inputenc}
 \usepackage{amsthm}
 \usepackage[T2A]{fontenc}
 \usepackage[russian, english]{babel}

% making logically defined graphics
%%%\usepackage{xypic}

% there are many more packages, add them here as you need them

% define commands here
\usepackage[T2A]{fontenc}
\usepackage[russian,english]{babel}


\theoremstyle{definition}
\newtheorem*{thmplain}{Theorem}
\begin{document}
For any series
\begin{align}
a_1+a_2+a_3+\ldots
\end{align}
of real or complex terms $a_j$ one may interpret its $m$'th 
remainder term 
\begin{align}
R_m \;:=\; a_{m+1}+a_{m+2}+\ldots
\end{align}
as a series.\, This {\it remainder term series} has its own 
partial sums
\begin{align}
S_m^{(n)}\;:=\; a_{m+1}+a_{m+2}+\ldots+a_{m+n} \qquad 
(n \;=\; 1,2,\ldots).
\end{align}
If\, $m+n =k$, then the $k^{\mathrm{th}}$ partial sum of the original series (1) 
is
\begin{align}
S_k \;=\; S_m+S_m^{(n)}.
\end{align}
For a fixed $m$, the limit $\lim_{k\to\infty}S_k$ apparently 
exists iff the limit 
$\lim_{n\to\infty}S_m^{(n)}$ exists.\, Thus we can write the\\
\textbf{Theorem.}\, The series (1) is convergent if and only if 
each remainder term series (2) is convergent.\\

Cf. the entry ``finite changes in convergent series''.


\begin{thebibliography}{9}
\bibitem{K} \CYRL. \CYRD. 
\CYRK\cyru\cyrd\cyrr\cyrya\cyrv\cyrc\cyre\cyrv:
\emph{\CYRM\cyra\cyrt\cyre\cyrm\cyra\cyrt\cyri\cyrch\cyre\cyrs\cyrk\cyri\cyrishrt\,\cyra\cyrn\cyra\cyrl\cyri\cyrz}.\,\CYRI\cyrz\cyrd\cyra\cyrt\cyre\cyrl\cyrsftsn\cyrs\cyrt\cyrv\cyro 
\, ``\CYRV\cyrery\cyrs\cyrsh\cyra\cyrya \, \cyrsh\cyrk\cyro\cyrl\cyra''. 
\CYRM\cyro\cyrs\cyrk\cyrv\cyra \,(1970).
\end{thebibliography} \\


\end{document}
