\documentclass[12pt]{article}
\usepackage{pmmeta}
\pmcanonicalname{ExampleUsingStolzCesaroTheorem}
\pmcreated{2013-03-22 15:31:02}
\pmmodified{2013-03-22 15:31:02}
\pmowner{georgiosl}{7242}
\pmmodifier{georgiosl}{7242}
\pmtitle{example using Stolz-Cesaro theorem}
\pmrecord{4}{37386}
\pmprivacy{1}
\pmauthor{georgiosl}{7242}
\pmtype{Example}
\pmcomment{trigger rebuild}
\pmclassification{msc}{40A05}
\pmrelated{StolzCesaroTheorem}
\pmrelated{LHpitalsRule}

\endmetadata

% this is the default PlanetMath preamble.  as your knowledge
% of TeX increases, you will probably want to edit this, but
% it should be fine as is for beginners.

% almost certainly you want these
\usepackage{amssymb}
\usepackage{amsmath}
\usepackage{amsfonts}

% used for TeXing text within eps files
%\usepackage{psfrag}
% need this for including graphics (\includegraphics)
%\usepackage{graphicx}
% for neatly defining theorems and propositions
%\usepackage{amsthm}
% making logically defined graphics
%%%\usepackage{xypic}

% there are many more packages, add them here as you need them

% define commands here
\begin{document}
\textbf{Example:}
We try to determine the value of
$$\lim_{n\to \infty}\frac{1^k+2^k+...+n^k}{n^{k+1}},\,k\in \mathbb{N}.$$
We consider the sequences $\alpha_{n\geq 1}=1^k+2^k+...+n^k$ and $\beta_{n\geq 1}=n^k$
and using the Stolz-Cesaro theorem we have that
\begin{eqnarray}
\lim_{n\to \infty}\frac{1^k+2^k+...+n^k}{n^{k+1}}=\\
\lim_{n\to \infty}\frac{(1^k+2^k+...+(n+1)^k)-(1^k+2^k+...+n^k)}{(n+1)^{k+1}-n^{k+1}}=\\
\lim_{n\to \infty}\frac{(n+1)^k}{{(n+1)^{k+1}-n^{k+1}}}.
\end{eqnarray}
\\Now we try to get the expression in the indeterminate 
form $\frac{0}{0}$ as n approaches $\infty$, dividing numerator and denominator of 
(3) by $(n+1)^k$.  
\begin{eqnarray}
\lim_{n\to \infty}\frac{1}{(n+1)-n^{k+1}(n+1)^{-k}}=\\
\lim_{n\to \infty}\frac{1}{n(1+n^{-1}-(1+n^{-1})^{-k})}=\\
\lim_{n\to \infty}\frac{n^{-1}}{1+n^{-1}-(1+n^{-1})^{-k}}.
\end{eqnarray}
By applying L'H\^opital's rule once we get
\begin{eqnarray}
\lim_{n\to \infty}\frac{n^{-1}}{1+n^{-1}-(1+n^{-1})^{-k}}=\\
\lim_{n\to \infty}\frac{-n^{-2}}{-n^{-2}-k(1
+n^{-1})^{-k-1}n^{-2}}=\\
\lim_{n\to \infty}\frac{1}{1+k(1+n^{-1})^{-k-1}}=\\
\frac{1}{1+k}.
\end{eqnarray}
%%%%%
%%%%%
\end{document}
