\documentclass[12pt]{article}
\usepackage{pmmeta}
\pmcanonicalname{DerivingTheTrigonometricAdditionFormulaeUsingAreaAndCosineRule1}
\pmcreated{2013-03-11 19:55:01}
\pmmodified{2013-03-11 19:55:01}
\pmowner{dkrbabajee}{19083}
\pmmodifier{}{0}
\pmtitle{Deriving the trigonometric addition formulae using area and cosine rule}
\pmrecord{1}{50173}
\pmprivacy{1}
\pmauthor{dkrbabajee}{0}
\pmtype{Definition}

%none for now
\begin{document}
\documentclass[10pt]{article}
\usepackage{url}
\usepackage{amssymb,amsmath,amstext}
\usepackage{amsfonts}
\usepackage{graphicx}
\renewcommand{\baselinestretch}{1.2}
%This is the command that spaces the manuscript for easy reading

\begin{document}
%\thispagestyle{empty}
\begin{center}
\Large
% TITLE GOES HERE
Deriving the trigonometric addition formulae using area and cosine rule
\end{center}




\begin{abstract}
The trigonometric addition formulae are very important and useful in Mathematics.
They can be derived from geometric proof, relations from analytical geometry, Euler formula or relations from
vectorial analysis. In this work, we prove the sine addition formula by considering area. Using cosine
rule, we prove the cosine addition formula. Our proof is simpler because
it requires the knowledge of area of triangles and cosine rule which are easily understood by students.
From the addition formulae, we obtain the difference formulae by solving simultaneous equations.\\
\bigskip
{\bf Keywords}\\
Trigonometric addition and difference formulae, area of triangles, cosine rule\\
\bigskip
\end{abstract}


\section{Introduction}
The sine and cosine addition and difference  formulae are given by
\begin{equation}\label{eq:trisinsum}
\sin{(\alpha\pm\beta)} = \sin{\alpha}\cos{\beta}\pm\sin{\beta}\cos{\alpha}
\end{equation}
and
\begin{equation}\label{eq:tricossum}
\cos{(\alpha\pm\beta)} = \cos{\alpha}\cos{\beta}\mp\sin{\alpha}\sin{\beta},
\end{equation}
respectively. The sine and cosine addition and diference formulae can be obtained from geometric proofs ,
Euler's formula, analytical geometry and vectorial analysis . Feldman  uses the cosine rule
and Pythagoras theorem to get the cosine difference formula. In this work, we consider the area of triangles to obtain the
sine addition formula since the area of any triangle depends on the sine of the angle. Then, using the cosine rule twice, we obtain
the cosine addition formula. Finally, we obtain the trigonometric difference formulae from addition formulae by solving
two simultaneous equations.
\section{Preliminaries}
We use the important formulas and identities:\\
From Fig. (\ref{fig:tri}), we have
\begin{figure}[t]
\centering
\includegraphics[width=5in,height=3in]
{tri.eps}
\caption{Triangle XYZ}\label{fig:tri}
\end{figure}
$$\text{area of $\triangle$ XYZ}=\frac{1}{2}x y \sin{\theta}$$
and the cosine rule gives
$$z^{2}=x^{2}+y^{2}-2x y\cos{\theta}.$$
The simple trigonometric identities are given by
\begin{eqnarray}\label{eq:triden1}
% \nonumber to remove numbering (before each equation)
  &&\sin{(180^{\circ}-\theta)}=\sin{\theta},\\ \label{eq:triden2}
  &&\cos{(180^{\circ}-\theta)}=-\cos{\theta}, \\ \label{eq:triden3}
  &&\sin^{2}{\theta}+\cos^{2}{\theta}=1.
\end{eqnarray}
\section{Derivation of Addition Trigonometric Formulae}
\begin{figure}[t]
\centering
\includegraphics[width=5in,height=3in]
{tri1.eps}
\caption{Quadrilateral ABCD}\label{fig:tri1}
\end{figure}
Fig. (\ref{fig:tri1}) shows a quadrilateral ABCD which consists of two right-angled triangles
$ABC$ and $ACD$.
In $\triangle\ ABC$, $C\widehat{A}B=\alpha$, $AC=s$. By simple trigonometry, we have
\begin{equation}\label{eq:triabc}
\begin{array}{cc}
  AB=s\ \cos{\alpha}, & BC=s\ \sin{\alpha}.
\end{array}
\end{equation}
In $\triangle\ ACD$, $C\widehat{A}D=\beta$, $AD=h$. We have the following trigonometric equations:
\begin{equation}\label{eq:triacd}
\begin{array}{cc}
  \cos{\beta}=\displaystyle{\frac{s}{h}}, & CD=h\ \sin{\beta}.
\end{array}
\end{equation}
We also note that $B\widehat{C}D=180^{\circ}-\alpha.$\\
We can express the area of quadrilateral ABCD as:
\begin{eqnarray}\label{eq:quadarea}\nonumber
% \nonumber to remove numbering (before each equation)
&&\text{Area of}\ \triangle\ ABC+ \text{Area of}\ \triangle\ ACD =\text{Area of}\ \triangle\ ABD+\text{Area of}\ \triangle\ BCD  \\ \nonumber
  &&\frac{1}{2}\ (AB)\ (BC)+ \frac{1}{2}\ (AC)\ (CD) =\frac{1}{2}\ (AB)\ h\ \sin{(\alpha+\beta)}\\&&\hspace{7.4cm}+
  \frac{1}{2}\ (CD)\ (BC)\ \sin{(180^{\circ}-\alpha)}.
\end{eqnarray}
Substituting eqs. (\ref{eq:triden1}), (\ref{eq:triabc}) and (\ref{eq:triacd}) into eq. (\ref{eq:quadarea}), we have, after simplifications,
\begin{equation}\label{eq:sim1}
\frac{1}{2}s^{2}\sin{\alpha}\cos{\alpha}+ \frac{1}{2}\ hs\sin{\beta}  =\frac{1}{2}hs\cos{\alpha} \sin{(\alpha+\beta)}+
\frac{1}{2}hs \sin{\beta}\sin^{2}{\alpha}.
\end{equation}
Dividing eq. (\ref{eq:sim1}) by $\displaystyle{\frac{1}{2}hs},$ we obtain
$$\frac{s}{h}\sin{\alpha}\cos{\alpha}+\sin{\beta}=\cos{\alpha}\sin{(\alpha+\beta)}+\sin{\beta}\sin^{2}{\alpha}.$$
Simplifying and using eq. (\ref{eq:triden3}), we have
\begin{eqnarray}\label{eq:sim2}\nonumber
% \nonumber to remove numbering (before each equation)
  \cos{\alpha}\sin{(\alpha+\beta)} &=& \frac{s}{h}\sin{\alpha}\cos{\alpha}+\sin{\beta}(1-\sin^{2}{\alpha}),  \\
   &=& \frac{s}{h}\sin{\alpha}\cos{\alpha}+\sin{\beta}\cos^{2}{\alpha}.
\end{eqnarray}
Dividing eq. (\ref{eq:sim2}) by $\cos{\alpha}$ and using eq. (\ref{eq:triacd}),
we finally obtain the sine addition formula given by eq. (\ref{eq:trisinsum}) with the $+$ sign.\\
We next consider $\triangle\  BCD$. By using cosine rule and eqs. (\ref{eq:triden2}), (\ref{eq:triabc}) and (\ref{eq:triacd}),
we have
\begin{eqnarray}\label{eq:bc}\nonumber
% \nonumber to remove numbering (before each equation)
  BD^{2} &=& CD^{2}+BC^{2}-2(CD)(BC)\cos{(180^{\circ}-\alpha)} , \\
   &=& h^{2}\sin^{2}{\beta}+s^{2}\sin^{2}{\alpha}+2hs\sin{\beta}\sin{\alpha}\cos{\alpha}.
\end{eqnarray}
Using cosine rule in $\triangle\ ABD$ and  eq. (\ref{eq:bc}),
we obtain
\begin{eqnarray}\label{eq:cos}\nonumber
% \nonumber to remove numbering (before each equation)
  \cos{(\alpha+\beta)} &=& \frac{h^{2}+AB^{2}-BD^{2}}{2h(AB)}  \\
   &=& \frac{h^{2}(1-\sin^{2}{\beta})+s^{2}(\cos^{2}{\alpha}-\sin^{2}{\alpha})-2hs\sin{\beta}\sin{\alpha}\cos{\alpha}}{2hs\cos{\alpha}}.
\end{eqnarray}
Using eq. (\ref{eq:triden3}) and $s^{2}=h^{2}\cos^{2}{\beta}$, eq. (\ref{eq:cos}) reduces to
\begin{eqnarray}\label{eq:cossim}\nonumber
% \nonumber to remove numbering (before each equation)
  \cos{(\alpha+\beta)} &=& \frac{h^{2}\cos^{2}{\beta}+s^{2}(\cos^{2}{\alpha}-\sin^{2}{\alpha})-2hs\sin{\beta}\sin{\alpha}\cos{\alpha}}{2hs\cos{\alpha}}\\ \nonumber
  &=& \frac{s^{2}(1+\cos^{2}{\alpha}-\sin^{2}{\alpha})-2hs\sin{\beta}\sin{\alpha}\cos{\alpha}}{2hs\cos{\alpha}}\\ \nonumber
  &=& \frac{2s^{2}\cos^{2}{\alpha}-2hs\sin{\beta}\sin{\alpha}\cos{\alpha}}{2hs\cos{\alpha}}\\
  &=& \frac{s}{h}\cos{\alpha}-\sin{\alpha}\sin{\beta}.
\end{eqnarray}
Substituting eq. (\ref{eq:triacd}) into eq. (\ref{eq:cossim}),
we finally obtain cosine addition formula given by eq. (\ref{eq:tricossum}) with the $+$ sign.
We point out the proof of the cosine addition formula is a generalization to that of Feldman 
who used $s=h=1.$
\section{Derivation of Trigonometric Difference Formulae}
The difference formulas can be obtained by replacing $\beta$ by $-\beta$ in eqs. (\ref{eq:trisinsum})
and (\ref{eq:tricossum}).\\
But we adopt a different approach. We set $\gamma=\alpha+\beta$ so that $\beta=\gamma-\alpha.$\\
Then  eqs. (\ref{eq:trisinsum}) and (\ref{eq:tricossum}) reduce to
\begin{equation}\label{eq:trisinsum1}
\sin{(\gamma)} = \sin{\alpha}\cos{(\gamma-\alpha)}+\cos{\alpha}\sin{(\gamma-\alpha)}
\end{equation}
and
\begin{equation}\label{eq:tricossum1}
\cos{\gamma} = \cos{\alpha}\cos{(\gamma-\alpha)}-\sin{\alpha}\sin{(\gamma-\alpha)},
\end{equation}
respectively. We obtain the trigonometric difference formulae by solving eqs. (\ref{eq:trisinsum1})
and (\ref{eq:tricossum1}) simultaneously.\\ $(\ref{eq:trisinsum1})\times \cos{\alpha}-(\ref{eq:tricossum1})\times \sin{\alpha}$
yields
$$
\sin{\gamma}\cos{\alpha}-\cos{\gamma}\sin{\alpha}   = (\sin^{2}{\alpha}+\cos^{2}{\alpha})\  \sin{(\gamma-\alpha)}
$$
so that
\begin{equation}\label{eq:tridiffsin}
\sin{(\gamma-\alpha)}=\sin{\gamma}\cos{\alpha}-\cos{\gamma}\sin{\alpha}
\end{equation}
using eq. (\ref{eq:triden3}).\\
Similarly, $(\ref{eq:trisinsum1})\times \sin{\alpha}+(\ref{eq:tricossum1})\times \cos{\alpha}$
results in
\begin{equation}\label{eq:tridiffsin}
\cos{(\gamma-\alpha)}=\cos{\gamma}\cos{\alpha}+\sin{\gamma}\sin{\alpha}.
\end{equation}

\bibliographystyle{plain}
\bibliography{References}

\end{document}

%%%%%
\end{document}
