\documentclass[12pt]{article}
\usepackage{pmmeta}
\pmcanonicalname{GronwallsTheorem}
\pmcreated{2013-03-22 19:33:44}
\pmmodified{2013-03-22 19:33:44}
\pmowner{pahio}{2872}
\pmmodifier{pahio}{2872}
\pmtitle{Gronwall's theorem}
\pmrecord{8}{42548}
\pmprivacy{1}
\pmauthor{pahio}{2872}
\pmtype{Theorem}
\pmcomment{trigger rebuild}
\pmclassification{msc}{40A99}
\pmclassification{msc}{11A25}
\pmclassification{msc}{26A12}
\pmrelated{EulerMascheroniConstant}
\pmrelated{RobinsTheorem}

% this is the default PlanetMath preamble.  as your knowledge
% of TeX increases, you will probably want to edit this, but
% it should be fine as is for beginners.

% almost certainly you want these
\usepackage{amssymb}
\usepackage{amsmath}
\usepackage{amsfonts}

% used for TeXing text within eps files
%\usepackage{psfrag}
% need this for including graphics (\includegraphics)
%\usepackage{graphicx}
% for neatly defining theorems and propositions
 \usepackage{amsthm}
% making logically defined graphics
%%%\usepackage{xypic}

% there are many more packages, add them here as you need them

% define commands here

\theoremstyle{definition}
\newtheorem*{thmplain}{Theorem}

\begin{document}
\PMlinkescapeword{means}

The function
$$G(n) \; :=\; \frac{\sigma(n)}{n\ln(\ln{n})} \qquad (n \;=\; 2,\,3,\,4,\,\ldots),$$
in which $\sigma(n)$ means the sum of the positive divisors of $n$, satisfies the equation
$$\limsup_{n\to\infty}\,G(n) \;=\; e^{\gamma}$$
where $\gamma$ is the Euler--Mascheroni constant.


\begin{thebibliography}{8}
\bibitem{G}{\sc T. H. Gronwall}:\, Some asymptotic expressions in the theory of numbers.\, $-$ 
\emph{Trans. Amer. Math. Soc.} \textbf{14} (1913) 113--122.
\end{thebibliography}

%%%%%
%%%%%
\end{document}
