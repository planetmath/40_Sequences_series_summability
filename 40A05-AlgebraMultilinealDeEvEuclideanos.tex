\documentclass[12pt]{article}
\usepackage{pmmeta}
\pmcanonicalname{AlgebraMultilinealDeEvEuclideanos}
\pmcreated{2013-03-11 19:23:36}
\pmmodified{2013-03-11 19:23:36}
\pmowner{juanman}{12619}
\pmmodifier{}{0}
\pmtitle{Algebra multilineal de e.v. euclideanos}
\pmrecord{1}{50064}
\pmprivacy{1}
\pmauthor{juanman}{0}
\pmtype{Definition}

%none for now
\begin{document}
\documentclass[12pt]{article}

\DeclareSymbolFont{AMSb}{U}{msb}{m}{n}
\DeclareSymbolFontAlphabet{\Bbb}{AMSb}
\pagestyle{empty}
\usepackage{graphicx}
\usepackage{xypic}

\newcommand{\modu}[1] {\left|\begin{array}{c} #1 \end{array}\right| }
\newcommand{\paren}[1]{\left(\begin{array}{c} #1 \end{array}\right) }
\newcommand{\llav}[1] {\left\{\begin{array}{c} #1 \end{array}\right\} }
\newcommand{\corch}[1]{\left[\begin{array}{c} #1 \end{array}\right] }

\begin{document} 

\centerline{\Large\bf \'Algebra Multilineal de E.V. Euclideanos.}
\bigskip
\centerline{Juan Manuel M\'arquez B.}


\vskip1cm
\sf
Estos son algunos hechos b\'asicos de los
--- {\bf espacios vectoriales euclideanos} --- 
\index{espacios euclideanos}
espacios vectoriales con producto interior y las ideas de transformaciones multilineales 
conocidas como {\bf tensores}.
\bigskip

Sea $V$ un espacio euclideano con $\langle\quad ,\quad\rangle$ su producto interior en el, es decir
un mapeo $V\times V\to \Bbb{R}$ que cumple:

\bigskip

i) $\langle rX+sY,Z\rangle=r\langle X,Z\rangle+s\langle Y,Z\rangle$

ii) $\langle X,rY+sZ\rangle=r\langle X,Y\rangle+s\langle X,Z\rangle$

iii) $\langle X,Y\rangle=\langle Y,X\rangle$

iv) $\langle X,X\rangle>0$ cuando $X\neq0$

v) $\langle X,X\rangle=0$ si y solo si $X=0$,

\bigskip

\noindent es decir, $\langle\quad,\quad\rangle$ es un mapeo bilineal, sim\'etrico y positivo-definido.


\begin{enumerate}

\item El conjunto de transformaciones lineales tambi\'en es un e.v.

\item Sean $V$ y $W$ dos espacios vectoriales reales de dimensi\'on finita. 
      Entonces su producto tensorial $V\otimes W$ es el espacio vectorial real de dimensi\'on igual al
      producto de $\dim V$ por $\dim W$, y cuya base son los s\'\i mbolos:
      $$b_i\otimes c_j$$
      donde los $b_i$ son base para $V$ y los $c_j$ base para $W$.

      Por ejemplo, si $V=\langle\{b_1,b_2,...,b_8\}\rangle$ y $W=\langle\{c_1,c_2,...,c_{11}\}\rangle$	
      entonces 
      $$V\otimes W=\langle\{b_1\otimes c_1,\ b_1\otimes c_2,\ b_1\otimes c_3,...,\ b_8\otimes c_{11}\}\rangle$$
      de donde vemos que $\dim(V\otimes W)=88$.

\item Un elemento t\'\i pico $B\in V\otimes W$ se escribe como una combinaci\'on lineal de los $b_i\otimes c_j$:
      $$B=B^{\mu\nu}b_{\mu}\otimes c_{\nu}$$
      donde se esta usando la convenci\'on de la suma de Einstein y que significa:
      $$B^{\mu\nu}b_{\mu}\otimes c_{\nu}=B^{11}b_1\otimes c_1+B^{12}b_1\otimes c_2+B^{21}b_2\otimes c_1+
                           \cdots+B^{8,11}b_8\otimes c_{11}$$
      

\item El espacio dual $V^*$ 
      \index{espacio dual}
      es el conjunto de transformaciones lineales
      $V\to\Bbb{R}$ tambi\'en llamados funcionales lineales o 
      {\bf covectores}\index{covector}.\\
      $V^*$ es un espacio vectorial con la suma $(f+g)(X)=f(X)+g(X)$ y la acci\'on de los escalares v\'\i a
      $(rf)(X)=rf(X)=f(rX)$.

\item Cada $f\in V^*$ puede representarse a trav\'es de
      $$f(X)=\langle X,a\rangle,$$
      donde $a\in V$ es fijo (Lema de Riesz).\\
      O dicho de otra forma: {\it las \'unicas transformaciones lineales
      $V\to\Bbb{R}$ son} 
      $$X\longmapsto \langle X, a\rangle$$

\item Sea $\{ b_1,...,b_n \}$ una base para $V$ entonces la matriz de
      Gram\index{matriz de Gram} o tensor m\'etrico \index{tensor m\'etrico} de esta base es
      $$G=\left[\begin{array}{c}    
         g_{ij}=\langle b_i, b_j\rangle
         \end{array}\right].$$
      Tal matriz $G=[g_{ij}]$ es no singular debido a la independencia lineal 
      de los $b_i$.

\item Se denota con $g^{ij}$ las entradas de la matriz $G^{-1}=[g_{ij}]^{-1}$.
      Observe que $\sum_k g^{ik}g_{kj}=\delta^i_j$ y que $\sum_k g_{ik}g^{kj}=\delta^j_i$. 
      Que mediante la convenci\'on de la suma se escribe solamente
      $$g^{ik}g_{kj}=\delta^i_j$$
      pues $G^{-1}G=1$ como producto de matrices,
      y
      $$g_{ik}g^{kj}=\delta^j_i.$$
      ya que $GG^{-1}=1$ tambi\'en como producto de matrices.

\item La {\bf base reciproca} \index{base reciproca} $b^i$ a $b_i$ se construye con
      $$b^i=g^{ij}b_j,$$
      con suma sobre $j$. Enfaticemos que $g^{ij}b_j=g^{i1}b_1+g^{i2}b_2+\cdots g^{in}b_n$

\item Representamos con $b^i$ los  covectores
      $$\beta^i(\quad):=\langle\quad,b^i \rangle,$$
      $\beta^i$ que llamaremos {\bf covectores b\'asicos}.

\item Si $a\in V$ entonces $a=a^ib_i$ con suma sobre $i$ (convenci\'on de la suma de Einstein)
      \index{Einstein}. Los $a^i$ se llaman componentes contra-variantes \index{componentes contravariantes} de $a$. 
      En la base reciproca $a=a_ib^i$ y llamamos a los $a_i$ componentes co-variantes
      \index{componentes covariantes} de $a$.
      Estas son las leyes de subir y bajar indices \index{leyes de subir y bajar indices}.
      Tenemos el siguiente "truco"
      $$a=a^ib_i=a^i\delta_i^jb_j=a^ig_{ik}g^{kj}b_j=a_kb^k$$
      lo que muestra como se escribe $a$ como combinaci\'on lineal en las dos bases: $b_i$ y $b^j$ de $V$.

\item La relaci\'on entre los componentes es entonces $$a_i=g_{ij}a^j$$ o bien
      $$a^i=g^{ij}a_j.$$

\item Ahora si $f\in V^*$ entonces
      $$f(\quad)=\langle\quad,a\rangle=\langle\quad,a_ib^i\rangle=a_i\langle\quad,b^i \rangle=a_i\beta^i(\quad)$$
      en otras palabras $f=a_i\beta^i$. As\'\i\ los $\beta^i$ generan a $V^*$. Y por lo tanto 
      $\dim V=\dim V^*$

\item $V^*$ tambi\'en es euclideano. Sean $f,g\in V^*$ y $a,b\in V$ sus
      representantes respectivamente, entonces 
      $$\langle f, g\rangle^*:= \langle a, b\rangle$$ 
      define un producto interior en $V^*$.

\item Sea $V^{**}=(V^*)^*$ el dual del dual. Se identifica con $V$, pues si $T:V^*\to\Bbb{R}$ entonces
      existir\'a $f\in V^*$ talque       
      $$F(\mu)=\langle \mu,f\rangle^*$$
      para todo covector $\mu$ y $f$ fija. Pero si $f(X)=\langle X,a\rangle$ entonces la aplicaci\'on
      $$F\longmapsto a,$$
      ser\'a un isomorfismo-isometr\'\i a $V^{**}\cong V$.
      En otras palabras podemos representar con los elementos de $V$ a los
      $V^{**}$ mediante la f\'ormula
      $$F(\mu)=\langle \mu,f\rangle^*=\langle m,a\rangle=\langle a,m\rangle=\mu(a)$$
      donde $\mu(\ )=\langle\quad ,m\rangle$. 	

      Mediante el mismo proceso de dualizaci\'on que hicimos para construir $V^*$ desde $V$, ahora habr\'a 
      funcionales lineales $B_i:V^*\to\Bbb{R}$ dados mediante la f\'ormula
      $$B_i(\quad)=\langle\quad,\beta_i\rangle^*$$
      donde los $\beta_i=g_{ij}\beta^j$ son los b\'asicos reciprocos de los $\beta^j$.
      As\'\i\ 
      $$F(\quad)=\langle\quad,f\rangle^*=\langle\quad,a^k\beta_k\rangle^*=a^k\langle\quad,\beta_k\rangle^*=a^kB_k(\quad )$$ 


      
\item Una {\bf tranformaci\'on k-lineal covariante} 
      \index{tranformaci\'on k-lineal covariante}
      o {\bf k-tensor covariante} es un
      $$A:V\times\cdots\times V\to\Bbb{R}$$ que satisface (para cualquiera sean los escalares $r,s$):
      $$A(rX_1+s Y_1,X_2,...,X_k)=
      rA(X_1,X_2,...,X_k)+sA(Y_1,X_2,...,X_k),$$
      $$A(X_1,rX_2+sY_2,...,X_k)=
      rA(X_1,X_2,...,X_k)+sA(X_1,Y_2,...,X_k),$$
      $$...$$
      $$A(X_1,X_2,...,rX_k+sY_k)=
      rA(X_1,X_2,...,X_k)+sA(X_1,X_2,...,Y_k),$$
      Es decir $A$ es lineal en cada uno de sus $k$ argumentos.
      En estos t\'erminos $\langle \ ,\ \rangle$ es un mapeo $2$-lineal.
      Tambi\'en $\det$ es $n$-lineal.

\item Similarmente las tranformaciones mul\-ti\-li\-nea\-les 
      con\-tra\-va\-rian\-tes 
      o los {\bf k-ten\-sores con\-tra\-va\-rian\-tes} 
      \index{k-ten\-sores con\-tra\-va\-rian\-tes}
      son los mapeos 
      $k$-lineales
      $$B:V^*\times\cdots\times V^*\to\Bbb{R}.$$

\item Sean $T^{(k,0)}V$ el conjunto de los $k$-co tensores y $T^{(0,l)}V$
      los $l$-contra tensores. Ellos forman espacios vectoriales con las 
      operaciones obvias. Tambi\'en es posible $T^{(k,l)}V$ que es el 
      conjunto de los tensores mixtos $k$-covariantes y $l$-contravariantes.
      Se tiene que 
 
\item Para contruir las bases (vectoriales) del espacio $T^{(2,0)}V$ 
      definimos el {\bf producto tensorial} 
      \index{producto tensorial}
      de basicos de $V^*$
      $$\beta^i\otimes\beta^j(X,Y):=\beta^i(X)\beta^j(Y),\index{$\beta^i\otimes\beta^j$}$$
      \noindent para $2$ factores, siendo el lado derecho de la relaci\'on de 
      arriba un simple producto en $\Bbb{R}$.
      Similarmente para $p$ covectores b\'asicos:
      $$\beta^{i_1}\otimes \cdots \otimes \beta^{i_p}(X_1,...,X_p)
      :=\beta^{i_1}(X_1)\cdots\beta^{i_p}(X_p).$$
      No es dif\'\i cil ver que tales construciones son verdaderamente transformaciones $p$-multilineales.

\item Para $T^{(0,l)}V$ tenemos
      $$b^{i_1}\otimes\cdots\otimes b^{i_l}(f_1,...,f_l):=f_1(b^{i_1})\cdots f_l(b^{i_l}),$$
      donde $f_i$ son $l$ covectores arbitrarios.

\item Para los mixtos vemos por ejemplo que en $T^{(1,1)}V$ tenemos
      $$\beta^i\otimes b^j(X,f):=\beta^i(X)f(b^j),$$ 
      ...etc.

\item Con aquellas bases, si $A\in T^{(3,0)}V$ entonces
      $$A=A_{\mu\nu\lambda}
      \beta^{\mu}\otimes\beta^{\nu}\otimes\beta^{\lambda},$$
      ejemplo que ilustra el caso general para co-tensores.
      Los $A_{\mu\nu\lambda}$ se llaman {\bf componentes covariantes}
      \index{componentes covariantes}
      del tensor $A$.
\item Sean $T\in T^{(k,0)}V$ y $W\in T^{(l,0)}V$ entonces $T\otimes W$
      ser\'a un $k+l$-tensor covariante cuyos componentes son
      $$(T\otimes W)_{i_1...i_{k+l}}=T_{i_1...i_k}W_{i_{k+1}...i_{k+l}}.$$
\item Con relaciones del tipo
      $$g^{\mu\nu}T_{\alpha\nu\lambda}={ { T_{\alpha} }^{\mu} }_{\lambda},$$
      (suma sobre $\nu$) se cambia de componentes covariantes a mixtos y a
      contravariantes.
\item Observe que 
                $$T^{(1,0)}V=V^*.$$
                $$T^{(0,1)}V=V^{**}=V.$$
                $$T^{(1,1)}V=\hom(V,V).$$
\item Tambi\'en tenemos
                $$\langle\ ,\ \rangle=g_{ij}\beta^i\otimes\beta^j.$$
      Adem\'as de que podemos transformar $T^{(2,0)}V\rightarrow T^{(0,2)}V$
      isom\'orficamente conforme a
      $$A_{ij}\beta^i\otimes\beta^j\mapsto A^{ij} b_i\otimes b_j.$$
      visto como tensor $T^{(1,1)}V$ se ver\'\i a como
      $${A_i}^j\beta^i\otimes b_j.$$

\item Un tensor covariante se dice {\bf alternante} 
      \index{tensor alternante}
      si cualquier cambio de 
      dos indices en los componentes corresponde un cambio de signo 
      con respecto al nuevo componente, como por ejemplo
      $$A_{\mu\nu\lambda\kappa}=-A_{\nu\mu\lambda\kappa}.$$
\item Sea $\Lambda^kV$ el conjunto de todos los tensores $k$-covariantes 
      alternantes, tambi\'en llamados $k$-formas\index{formas}. 
      $\Lambda^k(V)$ es un espacio vectorial.

\item La base para $\Lambda^2V=\Lambda^2(V)$ se logra con el {\bf producto exterior}
      \index{producto exterior}
      \index{$\beta^i\wedge\beta^j$}
      \begin{eqnarray*}
      \beta^i\wedge\beta^j(X,Y)&=&
      (\beta^i\otimes\beta^j-\beta^j\otimes\beta^i)(X,Y),\\
      &=&\beta^i\otimes\beta^j(X,Y)-\beta^j\otimes\beta^i(X,Y),\\
      &=&\beta^i(X)\beta^j(Y)-\beta^j(X)\beta^i(Y).
      \end{eqnarray*}
      Esta construcci\'on es bilineal y alternante.
      $\beta^i\wedge\beta^j$ se llama {\bf bivector b\'asico}.
      \index{bivector b\'asico}
      
      Observe que 
      $$\beta^i\wedge\beta^i=0,$$
      $$\beta^i\wedge\beta^j=-\beta^j\wedge\beta^i.$$
      Por lo tanto
      $$\dim \Lambda^2V=n!/2!(n-2)!={n\choose 2}.$$

\item 



Similarmente $\Lambda^3V$ es el espacio de las 3-formas que esta generado 
      por la construcci\'on;
      $$\beta^i\wedge \beta^j\wedge \beta^k=\sum_{p\in S_3}
       (-1)^p \beta^{p(i)}\otimes \beta^{p(j)}\otimes \beta^{p(k)},$$
      donde $S_3$ son las $6$ permutaciones de $\{i,j,k\}$ y $(-1)^p$ su 
      {\em paridad}. 

Observe que
$$\beta^i\wedge \beta^j\wedge \beta^k(b_s,b_t,b_u)=
\delta^i_s\delta^j_t\delta^k_u
-\delta^i_s\delta^k_t\delta^j_u
+\delta^k_s\delta^i_t\delta^j_u
-\delta^k_s\delta^j_t\delta^i_u
+\delta^j_s\delta^k_t\delta^i_u
-\delta^j_s\delta^i_t\delta^k_u
$$ 

Entonces $\beta^i\wedge \beta^j\wedge \beta^k(b_s,b_u,b_t)=0$ si los conjuntos 
      $\{ i,j,k\}$ y $\{ s,t,u\}$ son diferentes.

Adem\'as si $X=X^sb_s,Y=Y^tb_t,Z=Z^ub_u$ entonces

$$\beta^i\wedge \beta^j\wedge \beta^k(X,Y,Z)=
X^iY^jZ^k
-X^iY^kZ^j
+X^kY^iZ^j
-X^kY^jZ^i
+X^jY^kZ^i
-X^jY^iZ^k
$$ 

\bigskip
\bigskip
\bigskip


      Entonces
      $$\dim \Lambda^3V=n!/3!(n-3)!={n\choose 3}.$$
\item No es dif\'\i cil visualizar que
      $$\dim \Lambda^kV=n!/k!(n-k)!={n\choose n-k}.$$
      Y que $\Lambda^{n+r}V=0$ para cada $r\in\Bbb{Z}$.



\item Sea $A:V\to W$ una transformaci\'on lineal y si $f:W\to\Bbb{R}$ es un covector en $W$ entonces
      podemos hacer el "pullback" de $f$, denotado y calculado mediante
      $$A^*(f)=f\cdot A$$
      Observe que esto \'ultimo es la composici\'on $V\stackrel{A}\to W\stackrel{f}\to\Bbb{R}$.
      En otras palabras tenemos que $A$ induce una transformaci\'on lineal $A^*:W^*\to V^*$.		
\end{enumerate}

\pagebreak

\bigskip
\centerline{\bf ******************}


1. ?` Cu\'al es la dimensi\'on de espacio vectorial real $\langle\{b_i\otimes b_j\}\rangle$ (generado por los $b_i\otimes b_j$) ?\\
{\em 
RESP: solo hay $n^2$ objetos basicos:
$$b_1\otimes b_1$$
$$b_1\otimes b_2$$
$$b_1\otimes b_3$$
$$...$$ 
$$b_1\otimes b_n$$
$$b_2\otimes b_1$$
$$b_2\otimes b_2$$
$$...$$

$$...$$
$$b_{n-1}\otimes b_n$$
$$b_n\otimes b_1$$
$$b_n\otimes b_2$$
$$...$$
$$b_n\otimes b_n$$
para escribir combinaciones lineales 
$$B=B^{ij}b_i\otimes b_j=B^{11}b_1\otimes b_1+B^{12}b_1\otimes b_2+...+B^{nn}b_n\otimes b_n$$
estos $B$ son los 2-contra-tensores.

\rule{.1in}{.1in} 
}


2. ?` Y de $\langle\{b_{i_1}\otimes\cdots\otimes b_{i_k}\}\rangle$?\\
{\em 
RESP: solo hay $n^k$ objetos basicos:
$$b_1\otimes b_1\otimes... \otimes b_1$$
$$b_1\otimes b_1\otimes... \otimes b_2$$
$$b_1\otimes b_1\otimes... \otimes b_3$$
$$...$$ 
$$b_1\otimes b_1\otimes... \otimes b_n$$

$$b_1\otimes b_2\otimes... \otimes...b_1$$
$$b_1\otimes b_2\otimes... \otimes...b_2$$
$$...$$ 
$$b_1\otimes b_2\otimes... \otimes b_n$$
$$...$$

$$...$$
$$b_{n-1}\otimes b_n\otimes... \otimes b_1$$
$$...$$
$$b_{n-1}\otimes b_n\otimes... \otimes b_n$$
$$b_n\otimes b_n\otimes... \otimes b_1$$
$$b_n\otimes b_n\otimes... \otimes b_2$$
$$...$$

$$...$$
$$b_n\otimes b_n\otimes... \otimes b_n$$

\rule{.1in}{.1in} 
}

3. La relaci\'on con la geometr\'\i a esta dada mediante la substituci\'on
$$b_k=\partial_k=\frac{\partial}{\partial x^k}$$ donde
$$\partial_k=d\Phi|_pe_k$$
y donde $\Phi$ es una parametrizaci\'on local 
$\Phi\colon D\to M\subset{\Bbb{R}}^N$.

?` Cu\'al es la dimensi\'on de $\Omega^r(T_pM)$, siendo $T_pM$ de dimensi\'on $n$?

4. Si $f\colon U\to V$ y $g\colon V\to W$  son tranformaciones lineales entre espacios vectoriales reales demuestre que $(g\circ f)^* = f^*\circ g^*$.

5. Explique como transforman las 2-formas, si cambiamos de coordenadas.

6. Calcule los $\Gamma_{jk}^i$ para la parametrizac\'on $\Phi(v,w)=(5v+3w-1,-v+7w,-2v+3)$.

7. Calcular todos los $\Gamma_{jk}^i$ para el paraboloide $(v,w)\mapsto(v,w,v^2+w^2)$. ({\bf Resuelto en cursur.pdf})

8. ?` Cu\'al es el volumen generado por los campos vectoriales $S(u,v,w)=(2u+v,w+1,vw)$ , $T(u,v,w)=(u,v+w,w^2)$ y $D_ST$ en el punto $p=(1,1,2)$?

9. Sean $a=(a_1,a_2,...,a_n)$ y $b=(b_1,b_2,...,b_n)$. Definimos 
$$a\times_1 b=a_1b_2-a_2b_1$$
Demuestre que $\times_1$ define un mapeo bilineal ${\Bbb{R}}^n\times{\Bbb{R}}^n\to{\Bbb{R}}$.\\
{\em RESP: Se debe ver que si ponemos una combinaci\'on lineal en la primera entrada
$$(ka+k'a')\times_1 b=k(a\times_1 b)+k'(a'\times_1 b)$$
y tambi\'en en la segunda
$$a\times_1 (kb+k'b')=k(a\times_1 b)+k'(a\times_1 b')$$

\rule{.1in}{.1in} 
}

10. Similar al anterior pero con 
$$a\times_2 b=a_1b_3-a_3b_1$$

11. ?` Cu\'antos mapeos bilineales como los previos se pueden construir cuando $n=4$?

12. Demuestre que $\nabla_{\partial_k}(dx^a\wedge dx^b)$ satisface la regla de Leibniz.

13. Diga como son los componentes ${{A^{\mu}}_{\nu}}_{; k}$ en t\'erminos de derivadas parciales y symbolos de Chistoffel, para un tensor mixto
$A={A^{\mu}}_{\nu}\partial_{\mu}\otimes dx^{\nu}$.

14. Sean $A=A^{\mu\nu}\partial_{\mu}\otimes\partial_{\nu}$ y $B=B^{\sigma\tau}\partial_{\sigma}\otimes\partial_{\tau}$ dos contratensores de rango 2. Definimos
$$\langle A,B\rangle_2=g_{\nu\sigma}g_{\mu\tau}A^{\mu\nu}B^{\sigma\tau}={A^{\mu}}_{\sigma}{B^{\sigma}}_{\mu}$$
Verifique que $\langle\quad ,\quad\rangle_2$ define un producto interior entre los contratensores de rango 2.\\
{\em RESP: Se deben de checar los axiomas de espacio vectorial con producto interior:

i) Bilinealidad: $\langle kA+k'A',B\rangle_2=k\langle A,B\rangle_2+k'\langle A',B\rangle_2$ y\\ 
$\langle A,kB+k'B'\rangle_2=k\langle A,B\rangle_2+k'
\langle A,B\rangle_2$;

ii) Simetr\'ia: $\langle A,B\rangle_2=\langle B,A\rangle_2$;

iii) Positivo definido: $\langle A,A\rangle_2\ge 0$ si $A\neq0$;

iv) No degeneado: $\langle A,A\rangle_2=0$ si y solo si $A=0$.

\rule{.1in}{.1in} 
}




15. ?` Cu\'al es la ley de transformaci\'on de $\nabla_{\partial_k}\partial_l$ cuando cambiamos coordenadas? 

{\em RESP: Por un lado tenemos 

$$\Gamma_{ij}^s\partial_s=
\nabla_{\partial_i}\partial_j
\quad\mbox{tanto como}\quad\tilde{\Gamma}_{ij}^s\tilde{\partial}_s=
\nabla_{\tilde{\partial}_i}\tilde{\partial}
$$

y como 

$$\partial_k=\frac{\partial y^s}{\partial x^k}\tilde{\partial}_s
\quad
\mbox{implica}
\quad
\tilde{\partial}_k=\frac{\partial x^s}{\partial y^k}\partial_s$$

entonces 
$$\tilde{\Gamma}_{ij}^s\tilde{\partial}_s
=
\nabla_{\tilde{\partial}_i}\tilde{\partial}_j
=
\nabla_{
(\frac{\partial x^s}{\partial y^i}\partial_s)
}
(\frac{\partial x^t}{\partial y^j}\partial_t)
=
\frac{\partial x^s}{\partial y^i}
\nabla_{\partial_s}(\frac{\partial x^t}{\partial y^j}\partial_t)
$$

$$\tilde{\Gamma}_{ij}^s\tilde{\partial}_s
=\frac{\partial x^s}{\partial y^i}
\corch{
\frac{\partial}{\partial x^s}
(\frac{\partial x^t}{\partial y^j})\partial_t+
\frac{\partial x^t}{\partial y^j}\nabla_{\partial_s}\partial_t
}
$$
$$
=\frac{\partial x^s}{\partial y^i}
\corch{
\frac{\partial}{\partial x^s}
(\frac{\partial x^t}{\partial y^j})\partial_t+
\frac{\partial x^t}{\partial y^j}
\Gamma_{st}^r\partial_r
}
$$
$$
=\frac{\partial x^s}{\partial y^i}
\corch{
\frac{\partial}{\partial x^s}
(\frac{\partial x^t}{\partial y^j})+
\frac{\partial x^r}{\partial y^j}
\Gamma_{sr}^t
}\partial_t
$$

$$
=
\corch{
\frac{\partial}{\partial x^s}
(\frac{\partial x^t}{\partial y^j})
\frac{\partial x^s}{\partial y^i}
+
\frac{\partial x^r}{\partial y^j}
\frac{\partial x^s}{\partial y^i}
\Gamma_{sr}^t
}
\partial_t
$$
pero
$\partial_t=\frac{\partial y^u}{\partial x^t}\tilde{\partial}_u$
entonces
$$\tilde{\Gamma}_{ij}^u\tilde{\partial}_u
=
\corch{
\frac{\partial}{\partial x^s}
(\frac{\partial x^t}{\partial y^j})
\frac{\partial x^s}{\partial y^i}
\frac{\partial y^u}{\partial x^t}
+
\frac{\partial x^r}{\partial y^j}
\frac{\partial x^s}{\partial y^i}
\Gamma_{sr}^t
\frac{\partial y^u}{\partial x^t}
}
\tilde{\partial}_u
$$
por lo tanto
$$
\tilde{\Gamma}_{ij}^u
=
\frac{\partial}{\partial x^s}
(\frac{\partial x^t}{\partial y^j})
\frac{\partial x^s}{\partial y^i}
\frac{\partial y^u}{\partial x^t}
+
\frac{\partial x^r}{\partial y^j}
\frac{\partial x^s}{\partial y^i}
\Gamma_{sr}^t
\frac{\partial y^u}{\partial x^t}
$$

$$
\tilde{\Gamma}_{ij}^u
=
\frac{\partial^2 x^t}{\partial y^j\partial y^i}
\frac{\partial y^u}{\partial x^t}
+
\frac{\partial x^r}{\partial y^j}
\frac{\partial x^s}{\partial y^i}
\Gamma_{sr}^t
\frac{\partial y^u}{\partial x^t}
$$

\rule{.1in}{.1in} 
}

16. ?` Y de $\nabla_{\partial_k}A\otimes B$ 
donde $A,B$ son 1-covariantes? 

17 ?` Y de la derivada covariante en direcci\'on $X$ de un mixto 2-covariante y 1-contravariante?

\bigskip

Los siguientes ejemplos estan dedicados a mi amigo Perucho de Venezuela...

\bigskip

18.- Demuestre que $\Gamma^i_{ir}={1\over 2}g^{is}g_{si,r}$.

{\em RESP: Desde la definición 
$\Gamma^i_{jr}={1\over 2}g^{is}\corch{g_{js,r}+g_{sr,j}-g_{jr,s}}$ haciendo $j=i$ tendremos
$$\Gamma^i_{ir}={1\over 2}g^{is}\corch{g_{is,r}+g_{sr,i}-g_{ir,s}}$$
$$\Gamma^i_{ir}={1\over 2}\corch{g^{is}g_{is,r}+g^{is}g_{sr,i}-g^{is}g_{ir,s}}$$
$$={1\over 2}\corch{g^{is}g_{is,r}+g^{is}g_{sr,i}-g^{si}g_{sr,i}}$$

$$\Gamma^i_{ir}={1\over 2}g^{is}g_{is,r}$$

\rule{.1in}{.1in} 
}

19. El siguiente esquema permite encontrar la matriz de una transformaci\'on lineal (definida usando la base standar) en un cambio de base: 
\[
\begin{xy}
(35,20)*+{\Bbb{R}^3,e}="b"; (70,20)*+{\Bbb{R}^3,e}="c";%
(0,20)*+{\Bbb{R}^3,b}="a"; (105,20)*+{\Bbb{R}^3,b}="d";
%
{\ar "a";"b"}?*!/_5mm/{
\tiny
\left[\begin{array}{ccc}
1&0&0\\
0&1&0\\
0&-8&1 
\end{array}\right]
};
{\ar "b";"c"}?*!/_5mm/{
\tiny
\left[\begin{array}{ccc}
1&0&0\\
4&1&0\\
-1&1&1 
\end{array}\right]
};
{\ar "c";"d"}?*!/_5mm/{
\tiny
\left[\begin{array}{ccc}
1&0&0\\
0&1&0\\
0&8&1 
\end{array}\right]
};
\end{xy}
\]

%\begin{center}
%\includegraphics{prodtensor}
%\end{center}


\end{document}
%%%%%
\end{document}
