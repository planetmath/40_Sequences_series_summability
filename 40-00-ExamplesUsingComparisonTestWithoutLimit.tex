\documentclass[12pt]{article}
\usepackage{pmmeta}
\pmcanonicalname{ExamplesUsingComparisonTestWithoutLimit}
\pmcreated{2013-03-22 15:08:55}
\pmmodified{2013-03-22 15:08:55}
\pmowner{pahio}{2872}
\pmmodifier{pahio}{2872}
\pmtitle{examples using comparison test without limit}
\pmrecord{9}{36895}
\pmprivacy{1}
\pmauthor{pahio}{2872}
\pmtype{Example}
\pmcomment{trigger rebuild}
\pmclassification{msc}{40-00}
\pmrelated{PTest}
\pmdefines{over-harmonic series}

\endmetadata

% this is the default PlanetMath preamble.  as your knowledge
% of TeX increases, you will probably want to edit this, but
% it should be fine as is for beginners.

% almost certainly you want these
\usepackage{amssymb}
\usepackage{amsmath}
\usepackage{amsfonts}

% used for TeXing text within eps files
%\usepackage{psfrag}
% need this for including graphics (\includegraphics)
%\usepackage{graphicx}
% for neatly defining theorems and propositions
%\usepackage{amsthm}
% making logically defined graphics
%%%\usepackage{xypic}

% there are many more packages, add them here as you need them

% define commands here
\begin{document}
Do the following series converge?
\begin{align}
\sum_{n=1}^\infty \frac{1}{n^2+n+1}
\end{align}
\begin{align}
\sum_{n=1}^\infty \frac{n^3+n+1}{n^4+n+1}
\end{align}
The general \PMlinkescapetext{term} of (1) may be estimated upwards:
$$0 < \frac{1}{n^2+n+1} < \frac{1}{n^2+0+0} = \frac{1}{n^2}$$
Because\, $\sum_{n=1}^\infty \frac{1}{n^2}$ (an {\em over-harmonic series}) converges, then also (1) converges.

The general \PMlinkescapetext{term} of (2) may be estimated downwards:
$$\frac{n^3+n+1}{n^4+n+1} > \frac{n^3+0+0}{n^4+n^4+n^4} = 
  \frac{1}{3}\cdot\frac{1}{n} > 0$$
Because $\sum_{n=1}^\infty \frac{1}{3}\frac{1}{n}$ (the harmonic series divided by 3) diverges, then also (2) diverges.
%%%%%
%%%%%
\end{document}
