\documentclass[12pt]{article}
\usepackage{pmmeta}
\pmcanonicalname{SumOfSeries}
\pmcreated{2014-02-15 19:17:15}
\pmmodified{2014-02-15 19:17:15}
\pmowner{pahio}{2872}
\pmmodifier{pahio}{2872}
\pmtitle{sum of series}
\pmrecord{14}{36493}
\pmprivacy{1}
\pmauthor{pahio}{2872}
\pmtype{Definition}
\pmcomment{trigger rebuild}
\pmclassification{msc}{40-00}
\pmrelated{SumFunctionOfSeries}
\pmrelated{ManipulatingConvergentSeries}
\pmrelated{RemainderTerm}
\pmrelated{RealPartSeriesAndImaginaryPartSeries}
\pmrelated{LimitOfSequenceAsSumOfSeries}
\pmrelated{PlusSign}
\pmdefines{partial sum}

% this is the default PlanetMath preamble.  as your knowledge
% of TeX increases, you will probably want to edit this, but
% it should be fine as is for beginners.

% almost certainly you want these
\usepackage{amssymb}
\usepackage{amsmath}
\usepackage{amsfonts}

% used for TeXing text within eps files
%\usepackage{psfrag}
% need this for including graphics (\includegraphics)
%\usepackage{graphicx}
% for neatly defining theorems and propositions
%\usepackage{amsthm}
% making logically defined graphics
%%%\usepackage{xypic}

% there are many more packages, add them here as you need them

% define commands here
\begin{document}
If a series $\sum_{n = 1}^\infty a_n$ of real or complex 
numbers is convergent and the limit of its partial sums is $S$,
then $S$ is said to be the {\em sum of the series}.\, This 
circumstance may be denoted by
     $$\sum_{n = 1}^\infty a_n \;=\; S$$
or equivalently
     $$a_1+a_2+a_3+\ldots \;=\; S.$$
The sum of series has the distributive property
$$c\,(a_1+a_2+a_3+\ldots) \;=\; ca_1+ca_2+ca_3+\ldots$$
with respect to multiplication.\, Nevertheless, one must not 
think that the sum series means an addition of infinitely many 
numbers --- it's only a question of the limit
$$\lim_{n\to\infty}
\underbrace{(a_1+a_2+\ldots+a_n)}_{\textrm{partial sum}}.$$
See also the entry ``manipulating convergent series''!

The sum of the series is equal to the sum of a partial sum and 
the corresponding remainder term.
%%%%%
%%%%%
\end{document}
