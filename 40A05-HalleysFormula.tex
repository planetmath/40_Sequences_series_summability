\documentclass[12pt]{article}
\usepackage{pmmeta}
\pmcanonicalname{HalleysFormula}
\pmcreated{2013-03-22 19:34:39}
\pmmodified{2013-03-22 19:34:39}
\pmowner{pahio}{2872}
\pmmodifier{pahio}{2872}
\pmtitle{Halley's formula}
\pmrecord{8}{42564}
\pmprivacy{1}
\pmauthor{pahio}{2872}
\pmtype{Result}
\pmcomment{trigger rebuild}
\pmclassification{msc}{40A05}
\pmrelated{ListOfCommonLimits}

\endmetadata

% this is the default PlanetMath preamble.  as your knowledge
% of TeX increases, you will probably want to edit this, but
% it should be fine as is for beginners.

% almost certainly you want these
\usepackage{amssymb}
\usepackage{amsmath}
\usepackage{amsfonts}

% used for TeXing text within eps files
%\usepackage{psfrag}
% need this for including graphics (\includegraphics)
%\usepackage{graphicx}
% for neatly defining theorems and propositions
 \usepackage{amsthm}
% making logically defined graphics
%%%\usepackage{xypic}

% there are many more packages, add them here as you need them

% define commands here

\theoremstyle{definition}
\newtheorem*{thmplain}{Theorem}

\begin{document}
The following formula is due to the English scientist and mathematician Edmond Halley (1656 \`a 1742):
\begin{align}
\ln{x} \;=\; \lim_{n\to\infty}(\sqrt[n]{x}-1)n
\end{align}

{\it Proof.}\, We change the $n$th root to power of $e$ and use the power series expansion of exponential function:
\begin{align*}
(\sqrt[n]{x}-1)n &\;=\; (e^{\frac{\ln{x}}{n}}-1)n\\
&\;=\; \left(\sum_{m=0}^\infty\frac{(\ln{x}/n)^m)}{m!}-1\right)\!n\\ 
&\;=\; \sum_{m=1}^\infty\frac{(\ln{x}/n)^mn}{m!}\\
&\;=\; \ln{x}+\frac{1}{n}\sum_{m=2}^\infty\frac{(\ln{x})^m}{m!n^{m-2}}
\end{align*}
The last converging series has a finite sum, and as\, $n \to \infty$,\, the asserted formula follows.\\

\textbf{Note.}\, The formula (1) was known also by Leonhard Euler, who used it for defining the natural logarithm.\, Using  (1), one can easily prove the well-known laws of logarithm, e.g.
\begin{align*}
\ln{xy} &\;=\; \lim_{n\to\infty}(\sqrt[n]{x}\sqrt[n]{y}-1)n\\
&\;=\; \lim_{n\to\infty}(\sqrt[n]{x}\sqrt[n]{y}-\sqrt[n]{y}+\sqrt[n]{y}-1)n\\
&\;=\; \lim_{n\to\infty}y^{\frac{1}{n}}(\sqrt[n]{x}-1)n+\lim_{n\to\infty}(\sqrt[n]{y}-1)n\\
&\;=\; y^0\ln{x}+\ln{y}\\
&\;=\; \ln{x}+\ln{y}.
\end{align*}


\begin{thebibliography}{8}
\bibitem{L}{\sc Paul Loya}: {\em Amazing and Aesthetic Aspect of Analysis: On the incredible infinite}. A course in undergraduate analysis, fall 2006.  Available \PMlinkexternal{here}{http://www.math.binghamton.edu/dennis/478.f07/EleAna.pdf}.
\end{thebibliography}

%%%%%
%%%%%
\end{document}
