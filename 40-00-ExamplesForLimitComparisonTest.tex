\documentclass[12pt]{article}
\usepackage{pmmeta}
\pmcanonicalname{ExamplesForLimitComparisonTest}
\pmcreated{2013-03-22 15:08:48}
\pmmodified{2013-03-22 15:08:48}
\pmowner{alozano}{2414}
\pmmodifier{alozano}{2414}
\pmtitle{examples for limit comparison test}
\pmrecord{4}{36893}
\pmprivacy{1}
\pmauthor{alozano}{2414}
\pmtype{Example}
\pmcomment{trigger rebuild}
\pmclassification{msc}{40-00}

% this is the default PlanetMath preamble.  as your knowledge
% of TeX increases, you will probably want to edit this, but
% it should be fine as is for beginners.

% almost certainly you want these
\usepackage{amssymb}
\usepackage{amsmath}
\usepackage{amsthm}
\usepackage{amsfonts}

% used for TeXing text within eps files
%\usepackage{psfrag}
% need this for including graphics (\includegraphics)
%\usepackage{graphicx}
% for neatly defining theorems and propositions
%\usepackage{amsthm}
% making logically defined graphics
%%%\usepackage{xypic}

% there are many more packages, add them here as you need them

% define commands here

\newtheorem{thm}{Theorem}
\newtheorem{defn}{Definition}
\newtheorem{prop}{Proposition}
\newtheorem{lemma}{Lemma}
\newtheorem{cor}{Corollary}

\theoremstyle{definition}
\newtheorem{exa}{Example}

% Some sets
\newcommand{\Nats}{\mathbb{N}}
\newcommand{\Ints}{\mathbb{Z}}
\newcommand{\Reals}{\mathbb{R}}
\newcommand{\Complex}{\mathbb{C}}
\newcommand{\Rats}{\mathbb{Q}}
\newcommand{\Gal}{\operatorname{Gal}}
\newcommand{\Cl}{\operatorname{Cl}}
\newcommand{\lc}{\lim_{x\to c}}
\newcommand{\lzero}{\lim_{x\to 0}}
\newcommand{\lhzero}{\lim_{h\to 0}}
\newcommand{\linf}{\lim_{x\to \infty}}
\newcommand{\limn}{\lim_{n\to\infty}}
\newcommand{\sumi}{\sum_{i=1}^\infty }
\newcommand{\sumn}{\sum_{n=1}^\infty }
\newcommand{\sumno}{\sum_{n=0}^\infty }
\newcommand{\sumio}{\sum_{i=1}^\infty }
\begin{document}
\begin{exa}
Does the following series converge?
$$\sumn \frac{1}{n^2+n+1}$$
The series is similar to $\sumn 1/n^2$ which converges (use $p$-test, for example). Next we compute the limit:
$$\limn \frac{\frac{1}{n^2+n+1}}{\frac{1}{n^2}}=\limn \frac{n^2}{n^2+n+1} = 1$$
Therefore, since $1\neq 0$, by the Limit Comparison Test (with $a_n=1/(n^2+n+1)$ and $b_n=1/n^2$), the series converges.\\
\end{exa}

\begin{exa}
Does the following series converge?
$$\sumn \frac{n^3+n+1}{n^4+n+1}$$
If we ``forget'' about the lower order terms of $n$:
$$\frac{n^3+n+1}{n^4+n+1} \sim \frac{n^3}{n^4}=\frac{1}{n}$$
and $\sumn 1/n$ is the harmonic series which diverges (by the $p$-test). Thus, we take $b_n=1/n$ and compute:
$$\limn \frac{\frac{n^3+n+1}{n^4+n+1}}{\frac{1}{n}}=\limn \frac{n(n^3+n+1)}{n^4+n+1}=
\limn \frac{n^4+n^2+n}{n^4+n+1}=\limn \frac{1+1/n^2+1/n^3}{1+1/n^3+1/n^4}=1$$
Therefore the series diverges like the harmonic does.\\
\end{exa}
%%%%%
%%%%%
\end{document}
