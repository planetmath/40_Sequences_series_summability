\documentclass[12pt]{article}
\usepackage{pmmeta}
\pmcanonicalname{ProofOfComparisonTest}
\pmcreated{2013-03-22 13:22:06}
\pmmodified{2013-03-22 13:22:06}
\pmowner{mathwizard}{128}
\pmmodifier{mathwizard}{128}
\pmtitle{proof of comparison test}
\pmrecord{4}{33895}
\pmprivacy{1}
\pmauthor{mathwizard}{128}
\pmtype{Proof}
\pmcomment{trigger rebuild}
\pmclassification{msc}{40A05}

% this is the default PlanetMath preamble.  as your knowledge
% of TeX increases, you will probably want to edit this, but
% it should be fine as is for beginners.

% almost certainly you want these
\usepackage{amssymb}
\usepackage{amsmath}
\usepackage{amsfonts}

% used for TeXing text within eps files
%\usepackage{psfrag}
% need this for including graphics (\includegraphics)
%\usepackage{graphicx}
% for neatly defining theorems and propositions
%\usepackage{amsthm}
% making logically defined graphics
%%%\usepackage{xypic}

% there are many more packages, add them here as you need them

% define commands here
\begin{document}
Assume $|a_k|\leq b_k$ for all $k>n$. Then we define
$$s_k:=\sum_{i=k}^\infty |a_i|$$
and
$$t_k:=\sum_{i=k}^\infty b_i.$$
Obviously $s_k\leq t_k$ for all $k>n$. Since by assumption $(t_k)$ is \PMlinkid{convergent}{601} $(t_k)$ is bounded and so is $(s_k)$. Also $(s_k)$ is monotonic and therefore \PMlinkescapeword{convergent}. Therefore $\sum_{i=0}^\infty a_i$ is absolutely convergent.

Now assume $b_k\leq a_k$ for all $k>n$. If $\sum_{i=k}^\infty b_i$ is divergent then so is $\sum_{i=k}^\infty a_i$ because otherwise we could apply the test we just proved and show that $\sum_{i=0}^\infty b_i$ is convergent, which is is not by assumption.
%%%%%
%%%%%
\end{document}
