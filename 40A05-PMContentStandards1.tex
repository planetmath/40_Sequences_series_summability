\documentclass[12pt]{article}
\usepackage{pmmeta}
\pmcanonicalname{PMContentStandards1}
\pmcreated{2013-03-11 19:52:32}
\pmmodified{2013-03-11 19:52:32}
\pmowner{CWoo}{3771}
\pmmodifier{}{0}
\pmtitle{PM Content Standards}
\pmrecord{1}{50155}
\pmprivacy{1}
\pmauthor{CWoo}{0}
\pmtype{Definition}

%none for now
\begin{document}
\documentclass{article}
\usepackage{hyperref}
\usepackage{ulem}
\begin{document}

\title{PlanetMath Content Standards - Draft Version}

\date{\today}

\maketitle

\tableofcontents

There are current four major content sections on PlanetMath:
\begin{enumerate}
\item Encyclopedia
\item Papers
\item Books
\item Expositions
\end{enumerate}

We will discuss the standards in which users should apply when he/she makes a contribution.

\section{Encyclopedia Entries}

\subsection{Content}

\begin{enumerate}
\item This spells out what is considered appropriate PM encyclopedia material, and, when specific cases can be explicitly stated, what should be excluded.

\item The content of a PlanetMath entry material can be roughly broken down into two major categories: \emph{encyclopedic} or \emph{non-encyclopedic}, the second of which is also known as ``commons''.  Below is a list of what material is considered encyclopedic and what is not.

\item Entries that are considered encyclopedic include: 
\begin{enumerate}
\item mathematical constructs (things you find in algebra, geometry, etc..., definitions, theorems, etc...) that are well established; 
\item biographies on well-known mathematicians with significant mathematical contributions; biographies of well-known mathematicians without much content will be considered ``non-publishable''; biographies of lesser known mathematicians will be considered ``commons''.
\item mathematics education and pedagogy;
\item recreational mathematics (puzzles, instructional, etc...); 
\item useful and meaningful mathematical tables (tables of values of well-known statistical distributions, etc...), others??
However, the list above is not exhaustive.  It serves to illustrate some of the most common examples of a typical article found in a math encyclopedia.  In some cases, the determination of an entry being encyclopedic may need to involve the Content Committee.
\end{enumerate}

\item It is tempting to say that whatever is not encyclopedic must be non-encyclopedic.  However, this is not quite true, since the content of any entry on PlanetMath should still contain ``some'' amount of mathematics.  Below we look at some examples of non-encyclopedic entries, as well as some that are considered non-PlanetMath.

\item Things that are ``commons'' include, but not exclusively, the following:
\begin{enumerate}

\item mathematical constructs invented by the author or associates that are esoteric and not well-established.  [We may establish a separate place where people can put their personal ideas on PM]

\item frivolous numerical tables and lists.  A list of numerical evaluations of a function, or successively more accurate approximations of a mathematical constant (pi or e, etc...), which can be easily derived on a spreadsheet, a pocket calculator, or a simple computer program, is considered frivolous, unless sufficient explanations or demonstrations are given in the body of the entry as to its significance.  

A numerical list that is well-known, whose usefulness (pedagogical or whatever else) is clear is not considered a frivolous table.  Some examples of non-frivolous tables or lists are the statistical tables (useful in academia and industry) and a simple multiplication table (useful in teaching and pedagogy).  In the second example, however, the language must be written in such a way as to be age-appropriate.  Another example of a non-frivolous list would be, say, a list of values of some highly non-trivial functions or operations (including cardinalities of some special sets, orders of some special groups or algebras, dimensions of some special geometric constructs, etc...)

\item Personal views of mathematical structures, or mathematics at large.  In order for an entry to be consider encyclopedia-worthy, please avoid, if possible, the usage of ``I'' in the exposition.

\item Other scientific disciplines where math is used.  This is tricky,   because in some instances, the material is ``encyclopedia''-worthy, in some cases ``commons''-worthy, while in others, not even PlanetMath-worthy.  The following example  serves as guideline to the Content Committee: illustrations of applications of diff geometry to physics can be part of the PM encyclopedia, but physics itself should not be even be on PlanetMath (so an entry on Newton's Laws would not be appropriate on PM).  Another example, Boolean algebra, Turing machines, and abstract concepts in database design are part of encyclopedia on PM, but entries describing syntax/history or what not of specific programming languages like C or LISP, etc... or mathematical software packages may be considered ``commons''-worthy, while other programming languages, in particular scripting languages, such as HTML, or description on how to use a scientific calculator or a personal computer are not PM-worthy (they belong to PlanetComputing!)

\item Certain games where mathematics may be used.  Mere description of the games and history of the games is only ``commons''-worthy.  If the mathematics behind the games is described and defined, the material can be considered encyclopedic.

\item Description of specific numbers, or sets of numbers, without substantially mentioning the purpose of these numbers and their relevance to mathematics.
\end{enumerate}
Again, the above list is not exhaustive, and certain entries may need to be determined by the Content Committee on a case-by-case basis.

\item Material that should not be on PlanetMath includes anything that is not mathematical, even if the material ``appears" to contains mathematics.  Some of these items include 
\begin{enumerate}

\item certain facts that involve numbers: dates and times, numbering of musical works, numerical codes used by various industries, etc...

\item history, geography, literature; pop culture phenomena; biographies of non-mathematicians; autobiographies.

\item Ads or commercials selling particular products or services (even if they are mathematical); subject matter giving emphasis on the brand name of a commercial product or a licensed programming language for mathematics.

\item Anything that is incidental to the development of PM entries. ``sandboxes'', as well as lists of to-do items on PM, while useful, do not belong as a PlanetMath entry.  There may be other places on PlanetMath where these items can reside (where? maybe an improvement goal for us?).

\item Content containing insults and objectionable language.

\item Scientific disciplines (other than mathematics) where math is used.  This has already been discussed above.
\end{enumerate}

Once more, the list is not exhaustive, but only serve as a guidelines for users.
\end{enumerate}

\subsection{Standard}

\begin{enumerate}
\item This spells out that, given a PM entry, what standard is to be expected from the entry.  First, we discuss what are considered deletable material with respect to the ``standard''.

\item The standard of PM is that any copy-righted or plagiarized material, however mathematics content-worthy it may be, is not permitted.  If such an entry is found, it will be deleted automatically by the Content Committee, with or without notice of the author.

Furthermore, author's right to contribute may be suspended (temporarily or indefinitely for repeat offenders).

\item Duplicate entries are allowed, as long as they illustrate the various aspects of the same thing.  In the case exact or almost-exact duplicates, only one such entry should be kept and the rest deleted.  As to which one will be deleted will be determined by the Content Committee.

\item Given that a PM entry is here to stay, there are two levels of standard that can be expected on such an entry: \emph{publishable} (substantially complete) or \emph{non-publishable} (incomplete).  Below we describe what they mean.

\item Any publishable entry should be written in plain English.  It should be clear, informative, and substantially complete in terms of its exposition.  Entries including unfamiliar concepts should contain examples (or counterexamples), illustrations, and in particular, references.

\item Non-publishable entries are any entries that are deemed \textit{not} publishable, either by the author of the entry, or by the Content Committee.  Some specific types of non-publishable entries are listed below.
\begin{enumerate}

\item One-line entries are not publishable.

\item Material that is indicated, implicitly or explicitly, by the author to the effect that the material is not finished or complete is not publishable.  For example, a proof half-finished is not publishable, or an entry that the author intends to finish on another date is not publishable.

\item Entries purposely hidden by the author are not publishable.  This does not include entries with missing cache output.  Entries with missing cache output can be resolved either with re-rendering or a correction notice to the author of the entry.

\item Entries that include too many unexplained concepts, or too many terms that are undefined elsewhere on PlanetMath is not publishable.  The said entry is deemed publishable only when sufficient background is given.  This may mean that additional definitions defined, theorems listed (or proved), etc... and links to the said entry properly set up.

\item If the material is legally copied more or less straight from another source, the entry is considered non-publishable until the material is revised sufficiently to represent the author's own take on the subject matter.

\item Biographical material not containing enough descriptions of the mathematical contributions of the person is not publishable.  For example, a mere mention of a person's involvement in particular areas of mathematics does not constitute ``enough'' descriptions, and is considered not publishable.  What particular contributions, what major results proved by the person should be mentioned.

\item Entries with severe stylistic problems.  See the section on PM Entry Style for more detail.
\end{enumerate}
This list is by no means exhaustive, more may be included later.  In some cases, the ContentCommittee will decide on the publishability of an entry.
\end{enumerate}

\section*{Revisions}

\begin{enumerate}
\item Split from the ``original'' draft of the PM Community Guideline (11-11-2008)  --[[CWoo]]
\end{enumerate}

\end{document}
%%%%%
\end{document}
