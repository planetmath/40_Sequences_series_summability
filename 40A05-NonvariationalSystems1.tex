\documentclass[12pt]{article}
\usepackage{pmmeta}
\pmcanonicalname{NonvariationalSystems1}
\pmcreated{2013-03-11 19:28:55}
\pmmodified{2013-03-11 19:28:55}
\pmowner{linor}{11198}
\pmmodifier{}{0}
\pmtitle{Nonvariational systems}
\pmrecord{1}{50092}
\pmprivacy{1}
\pmauthor{linor}{0}
\pmtype{Definition}

%none for now
\begin{document}
\documentclass[12pt,leqno]{article}
\usepackage{amssymb}
\usepackage{color}

\newcommand{\be}{\begin{equation}}
\newcommand{\ee}{\end{equation}}
\newcommand{\dk}{d\sigma_{\xi}}
\newcommand{\dx}{d\sigma_{x}}
\newcommand{\nd}{\frac{ \partial}{ \partial n}}
\newcommand{\ndk}{\disfrac{\textstyle \partial}{\textstyle \partial n_{ \xi}}}
\newcommand{\ndx}{\disfrac{\textstyle \partial}{\textstyle \partial n_{ x}}}
\newcommand{\ik}{\int_{ \Gamma}}
\newcommand{\ts}{\textstyle}


\begin{document}

A typical nonvariational elliptic system has the form
$$
(NV) \;\; \left \{ \begin{array}{ll}
-\Delta u= f(x; u,v), &\, x\in \Omega \nonumber \\
-\Delta v= g(x; u,v), &\, x\in \Omega \\
u=v=0 \; \mbox{or} \;
\frac{\partial u}{\partial n}= \frac{\partial v}{\partial n}=0
&\, x\in \partial \Omega
\end{array} \right.
$$
where $\Omega\subset {\mathbb R}^N (N\ge 1)$ is an open bounded domain,
$f(x; u,v), g(x; u,v) \in \mathcal{C}^{1}(\overline{\Omega} \times
\mathbb{R}^2; \mathbb{R})$ in the variables $(u,v) \in \mathbb{R}^2$.
Here, we further assume that there exists no function $G(x;u,v)$ with
$\nabla G =(f, \pm g)$ or $\nabla G =(g, f)$. Under this assumption,
it is easy to see that problem (NV) is nonvariational.


\end{document}
%%%%%
\end{document}
