\documentclass[12pt]{article}
\usepackage{pmmeta}
\pmcanonicalname{EveryBoundedSequenceHasLimitAlongAnUltrafilter}
\pmcreated{2013-03-22 15:32:26}
\pmmodified{2013-03-22 15:32:26}
\pmowner{kompik}{10588}
\pmmodifier{kompik}{10588}
\pmtitle{every bounded sequence has limit along an ultrafilter}
\pmrecord{4}{37435}
\pmprivacy{1}
\pmauthor{kompik}{10588}
\pmtype{Theorem}
\pmcomment{trigger rebuild}
\pmclassification{msc}{40A05}
\pmclassification{msc}{03E99}
\pmrelated{Ultrafilter}

% this is the default PlanetMath preamble. as your knowledge
% of TeX increases, you will probably want to edit this, but
% it should be fine as is for beginners.

% almost certainly you want these
\usepackage{amssymb}
\usepackage{amsmath}
\usepackage{amsfonts}
\usepackage{amsthm}

% used for TeXing text within eps files
%\usepackage{psfrag}
% need this for including graphics (\includegraphics)
%\usepackage{graphicx}
% for neatly defining theorems and propositions
%
% making logically defined graphics
%%%\usepackage{xypic}

% there are many more packages, add them here as you need them

% define commands here

\newcommand{\sR}[0]{\mathbb{R}}
\newcommand{\sC}[0]{\mathbb{C}}
\newcommand{\sN}[0]{\mathbb{N}}
\newcommand{\sZ}[0]{\mathbb{Z}}
\newcommand{\N}[0]{\mathbb{N}}


\usepackage{bbm}
\newcommand{\Z}{\mathbbmss{Z}}
\newcommand{\C}{\mathbbmss{C}}
\newcommand{\R}{\mathbbmss{R}}
\newcommand{\Q}{\mathbbmss{Q}}



\newcommand*{\norm}[1]{\lVert #1 \rVert}
\newcommand*{\abs}[1]{| #1 |}

\newcommand{\Map}[3]{#1:#2\to#3}
\newcommand{\Emb}[3]{#1:#2\hookrightarrow#3}
\newcommand{\Mor}[3]{#2\overset{#1}\to#3}

\newcommand{\Cat}[1]{\mathcal{#1}}
\newcommand{\Kat}[1]{\mathbf{#1}}
\newcommand{\Func}[3]{\Map{#1}{\Cat{#2}}{\Cat{#3}}}
\newcommand{\Funk}[3]{\Map{#1}{\Kat{#2}}{\Kat{#3}}}

\newcommand{\intrv}[2]{\langle #1,#2 \rangle}

\newcommand{\vp}{\varphi}
\newcommand{\ve}{\varepsilon}

\newcommand{\Invimg}[2]{\inv{#1}(#2)}
\newcommand{\Img}[2]{#1[#2]}
\newcommand{\ol}[1]{\overline{#1}}
\newcommand{\ul}[1]{\underline{#1}}
\newcommand{\inv}[1]{#1^{-1}}
\newcommand{\limti}[1]{\lim\limits_{#1\to\infty}}

\newcommand{\Ra}{\Rightarrow}

%fonts
\newcommand{\mc}{\mathcal}

%shortcuts
\newcommand{\Ob}{\mathrm{Ob}}
\newcommand{\Hom}{\mathrm{hom}}
\newcommand{\homs}[2]{\mathrm{hom(}{#1},{#2}\mathrm )}
\newcommand{\Eq}{\mathrm{Eq}}
\newcommand{\Coeq}{\mathrm{Coeq}}

%theorems
\newtheorem{THM}{Theorem}
\newtheorem{DEF}{Definition}
\newtheorem{PROP}{Proposition}
\newtheorem{LM}{Lemma}
\newtheorem{COR}{Corollary}
\newtheorem{EXA}{Example}

%categories
\newcommand{\Top}{\Kat{Top}}
\newcommand{\Haus}{\Kat{Haus}}
\newcommand{\Set}{\Kat{Set}}

%diagrams
\newcommand{\UnimorCD}[6]{
\xymatrix{ {#1} \ar[r]^{#2} \ar[rd]_{#4}& {#3} \ar@{-->}[d]^{#5} \\
& {#6} } }

\newcommand{\RovnostrCD}[6]{
\xymatrix@C=10pt@R=17pt{
& {#1} \ar[ld]_{#2} \ar[rd]^{#3} \\
{#4} \ar[rr]_{#5} && {#6} } }

\newcommand{\RovnostrCDii}[6]{
\xymatrix@C=10pt@R=17pt{
{#1} \ar[rr]^{#2} \ar[rd]_{#4}&& {#3} \ar[ld]^{#5} \\
& {#6} } }

\newcommand{\RovnostrCDiiop}[6]{
\xymatrix@C=10pt@R=17pt{
{#1}  && {#3} \ar[ll]_{#2}  \\
& {#6} \ar[lu]^{#4} \ar[ru]_{#5} } }

\newcommand{\StvorecCD}[8]{
\xymatrix{
{#1} \ar[r]^{#2} \ar[d]_{#4} & {#3} \ar[d]^{#5} \\
{#6} \ar[r]_{#7} & {#8}
}
}

\newcommand{\TriangCD}[6]{
\xymatrix{ {#1} \ar[r]^{#2} \ar[rd]_{#4}&
{#3} \ar[d]^{#5} \\
& {#6} } }

\newcommand{\F}{\mc F}
\newcommand{\Flim}{\operatorname{\F\text{-}\lim}}
\begin{document}
\begin{THM}
Let $\F$ be an ultrafilter on $\N$ and $(x_n)$ be a real bounded
sequence. Then $\Flim x_n$ exists.
\end{THM}

\begin{proof}
Let $(x_n)$ be a bounded sequence. Choose $a_0$ and $b_0$ such
that $a_0\leq x_n \leq b_0$. Put $c_0:=\frac{a_0+b_0}2$. Then
precisely one of the sets $\{n\in\N; x_n \in \intrv{a_0}{c_0}\}$,
$\{n\in\N; x_n \in \intrv{c_0}{b_0}\}$ belongs to the filter $\F$.
(Their union is $\N$ and the filter $\F$ is an ultrafilter.) We
choose $\intrv{a_1}{b_1}$ as that subinterval from
$\intrv{a_0}{c_0}$ and $\intrv{c_0}{b_0}$ for which $C:=\{n\in\N;
x_n \in \intrv{a_1}{b_1}\}$ belongs to $\F$.

Now we again bisect the interval $\intrv{a_1}{b_1}$ by putting
$c_1=\frac{a_1+b_1}2$. Denote $A:=\{n\in\N; x_n \in
\intrv{a_1}{c_1}\}$, $B:=\{n\in\N; x_n \in \intrv{c_1}{b_1}\}$. It
holds $B\cup A\cup (\N\setminus C)=\N$. By the alternative
characterization of ultrafilters we get that one of these sets is
in $\F$. The set $\N\setminus C$ doesn't belong to $\F$, therefore
it must be one of the sets $A$ and $B$. We choose the
corresponding interval for $\intrv{a_2}{b_2}$.

By induction we obtain the monotonous sequences $(a_n)$, $(b_n)$
with the same limit $\limti n a_n = \limti n b_n :=L$ such that
for any $n\in\N$ it holds $\{n\in\N; x_n \in
\intrv{a_1}{b_1}\}\in\F$.

We claim that $\Flim x_n=L$. Indeed, for any $\ve>0$ there is
$n\in\N$ such that $\intrv{a_n}{b_n}\subseteq (L-\ve,L+\ve)$, thus
$\{n\in\N; x_n \in \intrv{a_n}{b_n}\} \subseteq A(\ve)$. The set
$\{n\in\N; x_n \in \intrv{a_1}{b_1}\}$ belongs to $\F$, hence
$A(\ve)\in\F$ as well.
\end{proof}

Note that, if we modify the definition of $\F$-limit in a such way
that we admit the values $\pm\infty$, then every sequence has
$\F$-limit along an ultrafilter $\F$. (The limit is $+\infty$ if
for each neighborhood $V$ of infinity, the set $\{n\in\N; x_n\in
V\}$ belongs to $\F$. Similarly for $-\infty$.)

\begin{thebibliography}{1}

\bibitem{agg}
M.~A. Alekseev, L.~{Yu.} Glebsky, and E.~I. Gordon, \emph{On
approximations of
  groups, group actions {and Hopf} algebras}, Journal of Mathematical Sciences
  \textbf{107} (2001), no.~5, 4305--4332.

\bibitem{balste}
B.~Balcar and P.~{\v{S}}t\v{e}p\'anek, \emph{Teorie mno\v{z}in},
Academia,
  Praha, 1986 (Czech).

\bibitem{hrjech}
K.~Hrbacek and T.~Jech, \emph{{Introduction to set theory}},
{Marcel Dekker},
  New York, 1999.

\end{thebibliography}
%%%%%
%%%%%
\end{document}
