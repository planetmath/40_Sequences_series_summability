\documentclass[12pt]{article}
\usepackage{pmmeta}
\pmcanonicalname{HarmonicSeries}
\pmcreated{2013-03-22 13:02:46}
\pmmodified{2013-03-22 13:02:46}
\pmowner{CWoo}{3771}
\pmmodifier{CWoo}{3771}
\pmtitle{harmonic series}
\pmrecord{8}{33449}
\pmprivacy{1}
\pmauthor{CWoo}{3771}
\pmtype{Definition}
\pmcomment{trigger rebuild}
\pmclassification{msc}{40A05}
\pmrelated{HarmonicNumber}
\pmrelated{PrimeHarmonicSeries}
\pmrelated{SumOfPowers}
\pmdefines{p-series}
\pmdefines{harmonic series of order}

\usepackage{amssymb}
\usepackage{amsmath}
\usepackage{amsfonts}

%\usepackage{psfrag}
%\usepackage{graphicx}
%%%\usepackage{xypic}
\begin{document}
\textbf{The} \emph{harmonic series} is 

 $$ h = \sum_{n=1}^\infty \frac{1}{n} $$

The harmonic series is known to diverge.  This can be proven via the integral test; compare $h$ with 

 $$ \int_{1}^\infty \frac{1}{x} \; dx. $$ 

The harmonic series is a special case of the \emph{$p$-series}, $h_p$, which has the form

$$ h_p = \sum_{n=1}^\infty \frac{1}{n^p} $$

where $p$ is some positive real number.  The series is known to converge (leading to the p-series test for series convergence) iff $p > 1$.  In using the comparison test, one can often compare a given series with positive terms to some $h_p$.

\textbf{Remark 1.}  One could call $h_p$ with\, $p > 1$\, an {\em overharmonic series} and $h_p$ with\, $p < 1$\, an {\em underharmonic series}; the corresponding names are known at least in Finland.

\textbf{Remark 2.}  A $p$-series is sometimes called \emph{a} harmonic series, so that \emph{the} harmonic series is a harmonic series with $p=1$.\\

For complex-valued $p$, $h_p = \zeta(p)$, the Riemann zeta function.

A famous $p$-series is $h_2$ (or $\zeta(2)$), which converges to $\frac{\pi^2}{6}$.  In general no $p$-series of odd $p$ has been solved analytically.

A $p$-series which is not summed to $\infty$, but instead is of the form

$$ h_p(k) = \sum_{n=1}^k \frac{1}{n^p} $$

is called a $p$-series (or a harmonic series) of order $k$ of $p$.
%%%%%
%%%%%
\end{document}
