\documentclass[12pt]{article}
\usepackage{pmmeta}
\pmcanonicalname{ProofOfAbsoluteConvergenceTheorem}
\pmcreated{2013-03-22 13:41:52}
\pmmodified{2013-03-22 13:41:52}
\pmowner{paolini}{1187}
\pmmodifier{paolini}{1187}
\pmtitle{proof of absolute convergence theorem}
\pmrecord{6}{34373}
\pmprivacy{1}
\pmauthor{paolini}{1187}
\pmtype{Proof}
\pmcomment{trigger rebuild}
\pmclassification{msc}{40A05}

\endmetadata

% this is the default PlanetMath preamble.  as your knowledge
% of TeX increases, you will probably want to edit this, but
% it should be fine as is for beginners.

% almost certainly you want these
\usepackage{amssymb}
\usepackage{amsmath}
\usepackage{amsfonts}

% used for TeXing text within eps files
%\usepackage{psfrag}
% need this for including graphics (\includegraphics)
%\usepackage{graphicx}
% for neatly defining theorems and propositions
%\usepackage{amsthm}
% making logically defined graphics
%%%\usepackage{xypic}

% there are many more packages, add them here as you need them

% define commands here
\begin{document}
Suppose that $\sum a_n$ is absolutely convergent, i.e., that $\sum \vert a_n \vert$ is convergent.
First of all, notice that 
\[
  0\le a_n + \vert a_n \vert \le 2 \vert a_n\vert,
\]
and since the series $\sum (a_n+\vert a_n\vert)$ has non-negative terms it can be compared with $\sum 2\vert a_n\vert=2\sum \vert a_n\vert$ and hence converges.

On the other hand
\[
 \sum_{n=1}^N a_n = \sum_{n=1}^N (a_n+\vert a_n\vert) - \sum_{n=1}^N \vert a_n\vert.
\]
Since both the partial sums on the right hand side are convergent, the partial 
sum on the left hand side is also convergent. So, the series $\sum a_n$ is convergent.
%%%%%
%%%%%
\end{document}
