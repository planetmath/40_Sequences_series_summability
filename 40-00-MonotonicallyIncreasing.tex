\documentclass[12pt]{article}
\usepackage{pmmeta}
\pmcanonicalname{MonotonicallyIncreasing}
\pmcreated{2013-03-22 12:22:35}
\pmmodified{2013-03-22 12:22:35}
\pmowner{akrowne}{2}
\pmmodifier{akrowne}{2}
\pmtitle{monotonically increasing}
\pmrecord{10}{32134}
\pmprivacy{1}
\pmauthor{akrowne}{2}
\pmtype{Definition}
\pmcomment{trigger rebuild}
\pmclassification{msc}{40-00}
\pmsynonym{monotone increasing}{MonotonicallyIncreasing}
\pmsynonym{strictly increasing}{MonotonicallyIncreasing}
\pmrelated{MonotonicallyDecreasing}

\endmetadata

\usepackage{amssymb}
\usepackage{amsmath}
\usepackage{amsfonts}

%\usepackage{psfrag}
%\usepackage{graphicx}
%%%\usepackage{xypic}
\begin{document}
A sequence $(s_n)$, $s_n \in \mathbb{R} $ is called \emph{monotonically increasing} if 

$$ s_m > s_n \; \forall \; m > n $$

Similarly, a real function $f(x)$ is called monotonically increasing if 

$$ f(x) > f(y) \; \forall \; x > y $$

Compare this to monotonically nondecreasing.

\textbf{Conflict note.}  This condition is also sometimes called \emph{strictly increasing} \cite{NIST}.  In such a context, ``monotonically increasing'' has the same meaning as monotonically nondecreasing.

\begin{thebibliography}{3}
\bibitem{NIST} ``\PMlinkexternal{strictly increasing}{http://www.nist.gov/dads/HTML/strictlyIncreasing.html},'' from the NIST Dictionary of Algorithms and Data Structures, Paul E. Black, ed.
\end{thebibliography}
%%%%%
%%%%%
\end{document}
