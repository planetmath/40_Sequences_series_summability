\documentclass[12pt]{article}
\usepackage{pmmeta}
\pmcanonicalname{CauchyProduct}
\pmcreated{2013-03-22 13:37:14}
\pmmodified{2013-03-22 13:37:14}
\pmowner{msihl}{2134}
\pmmodifier{msihl}{2134}
\pmtitle{Cauchy product}
\pmrecord{7}{34254}
\pmprivacy{1}
\pmauthor{msihl}{2134}
\pmtype{Definition}
\pmcomment{trigger rebuild}
\pmclassification{msc}{40-00}

\endmetadata

% this is the default PlanetMath preamble.  as your knowledge
% of TeX increases, you will probably want to edit this, but
% it should be fine as is for beginners.

% almost certainly you want these
\usepackage{amssymb}
\usepackage{amsmath}
\usepackage{amsfonts}
%\usepackage{bbm}

% used for TeXing text within eps files
%\usepackage{psfrag}
% need this for including graphics (\includegraphics)
%\usepackage{graphicx}
% for neatly defining theorems and propositions
%\usepackage{amsthm}
% making logically defined graphics
%%%\usepackage{xypic}

% there are many more packages, add them here as you need them

% define commands here
\begin{document}
Let $a_k$ and $b_k$ be two sequences of real or complex numbers for 
$k \in {\mathbb N}_0$ ( ${\mathbb N}_0$ is the set of natural numbers containing zero).
The Cauchy product is defined by:
\begin{equation}
(a \circ b)(k) = \sum_{l=0}^k a_l b_{k-l}.
\end{equation}
This is basically the convolution for two sequences.
Therefore the product of two series $\sum_{k=0}^{\infty} a_k$, $\sum_{k=0}^{\infty} b_k$ is given by:
\begin{equation}
\sum_{k=0}^{\infty} c_k = \left(\sum_{k=0}^{\infty} a_k \right)\cdot \left(\sum_{k=0}^{\infty} b_k \right) =
\sum_{k=0}^{\infty} \sum_{l=0}^k a_l b_{k-l}.
\end{equation}
A sufficient condition for the resulting series $\sum_{k=0}^{\infty} c_k$ to be absolutely convergent is that $\sum_{k=0}^{\infty} a_k$ and $\sum_{k=0}^{\infty} b_k$ both converge absolutely .
%%%%%
%%%%%
\end{document}
