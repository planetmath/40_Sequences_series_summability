\documentclass[12pt]{article}
\usepackage{pmmeta}
\pmcanonicalname{ArithmeticgeometricSeries}
\pmcreated{2013-03-22 16:02:15}
\pmmodified{2013-03-22 16:02:15}
\pmowner{perucho}{2192}
\pmmodifier{perucho}{2192}
\pmtitle{arithmetic-geometric series}
\pmrecord{6}{38087}
\pmprivacy{1}
\pmauthor{perucho}{2192}
\pmtype{Derivation}
\pmcomment{trigger rebuild}
\pmclassification{msc}{40C99}

\endmetadata

% this is the default PlanetMath preamble.  as your knowledge
% of TeX increases, you will probably want to edit this, but
% it should be fine as is for beginners.

% almost certainly you want these
\usepackage{amssymb}
\usepackage{amsmath}
\usepackage{amsfonts}

% used for TeXing text within eps files
%\usepackage{psfrag}
% need this for including graphics (\includegraphics)
%\usepackage{graphicx}
% for neatly defining theorems and propositions
%\usepackage{amsthm}
% making logically defined graphics
%%%\usepackage{xypic}

% there are many more packages, add them here as you need them

% define commands here

\begin{document}
It is well known that a finite geometric series is  given by
\begin{align}
G_n(q)=\sum_{k=1}^nq^k=\frac{q}{1-q}(1-q^n), \qquad q\neq 1,
\end{align}
where in general $q=re^{i\theta}$ is complex.
When we are dealing with such sums it is common to consider the expression
\begin{align}
H_n(q):=\sum_{k=1}^n kq^k, \qquad q\neq 1,
\end{align} 
which we shall call  an {\em arithmetic-geometric series}. Let us derive a formula for $H_n(q)$.
\begin{align*}
H_n(q)=\sum_{k=1}^n kq^k, \qquad qH_n(q)=\sum_{k=1}^n kq^{k+1}.
\end{align*}
Subtracting,
\begin{align*}
(1-q)H_n(q)=\sum_{k=1}^n kq^k-\sum_{k=1}^n kq^{k+1}=
\sum_{k=1}^n kq^k-\sum_{k=2}^{n+1}(k-1)q^k=
\sum_{k=1}^n kq^k-\sum_{k=2}^n(k-1)q^k-nq^{n+1}.
\end{align*}
We will proceed to eliminate the right-hand side sums.
\begin{align*}
(1-q)H_n(q)=q+\sum_{k=2}^n q^k -nq^{n+1}=
\sum_{k=1}^n q^k-nq^{n+1}.
\end{align*}
By using (1) and solving for $H_n(q)$, we obtain
\begin{align}
H_n(q)=\sum_{k=1}^n kq^k=\frac{q}{(1-q)^2}(1-q^n)-\frac{nq^{n+1}}{1-q}\:\cdot
\end{align}
The formula (3) holds in any commutative ring with 1, as long as $(1-q)$
is invertible. If $q$ is a complex number and
$|q|<1$, (3) is the partial sum of the convergent series
\begin{align*}
H(q)=\lim_{n\to\infty}H_n(q)=\lim_{n\to\infty}\sum_{k=1}^n kq^k=
\lim_{n\to\infty}\bigg[\frac{q}{(1-q)^2}(1-q^n)-\frac{nq^{n+1}}{1-q}\bigg],
\end{align*}
that is,
\begin{align}
H(q)=\sum_{k=1}^\infty kq^k=\frac{q}{(1-q)^2},\, \qquad |q|<1.
\end{align}

This last result giving the sum of a converging arithmetic-geometric series may be, naturally, obtained also from the sum formula of the converging geometric series, i.e.
$$1\!+\!q\!+q^2\!+\!q^3\!+...\, = \frac{1}{1-q},$$
when one differentiates both sides with respect to $q$ and then multiplies them by $q$:
$$1\!+\!2q\!+\!3q^2\!+...\, = \frac{1}{(1\!-\!q)^2},$$
$$q\!+\!2q^2\!+\!3q^3\!+...\, = \frac{q}{(1\!-\!q)^2}$$
(A power series can be differentiated termwise on the open interval of convergence.)

%%%%%
%%%%%
\end{document}
