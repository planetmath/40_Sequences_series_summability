\documentclass[12pt]{article}
\usepackage{pmmeta}
\pmcanonicalname{CauchyCriterionForConvergence}
\pmcreated{2013-03-22 13:22:03}
\pmmodified{2013-03-22 13:22:03}
\pmowner{mathwizard}{128}
\pmmodifier{mathwizard}{128}
\pmtitle{Cauchy criterion for convergence}
\pmrecord{14}{33894}
\pmprivacy{1}
\pmauthor{mathwizard}{128}
\pmtype{Theorem}
\pmcomment{trigger rebuild}
\pmclassification{msc}{40A05}

% this is the default PlanetMath preamble.  as your knowledge
% of TeX increases, you will probably want to edit this, but
% it should be fine as is for beginners.

% almost certainly you want these
\usepackage{amssymb}
\usepackage{amsmath}
\usepackage{amsfonts}

% used for TeXing text within eps files
%\usepackage{psfrag}
% need this for including graphics (\includegraphics)
%\usepackage{graphicx}
% for neatly defining theorems and propositions
%\usepackage{amsthm}
% making logically defined graphics
%%%\usepackage{xypic}

% there are many more packages, add them here as you need them

% define commands here
\begin{document}
A series $\sum_{i=0}^\infty a_i$ in a Banach space $(V,\|\cdot\|)$ is \PMlinkid{convergent}{2311} iff for every $\varepsilon>0$ there is a number $N\in\mathbb{N}$ such that
$$\|a_{n+1}+a_{n+2}+\cdots+a_{n+p}\|<\varepsilon$$
holds for all $n>N$ and $p\geq1$.


\subsection*{Proof:}
First define
$$s_n:=\sum_{i=0}^n a_i.$$
Now, since $V$ is complete, $(s_n)$ converges if and only if it is a Cauchy sequence, so if for every $\varepsilon>0$ there is a number $N$, such that for all $n,m>N$ holds:
$$\|s_m-s_n\|<\varepsilon.$$
We can assume $m>n$ and thus set $m=n+p$. The series is \PMlinkescapetext{convergent} iff
$$\|s_{n+p}-s_n\|=\|a_{n+1}+a_{n+2}+\cdots+a_{n+p}\|<\varepsilon.$$
%%%%%
%%%%%
\end{document}
