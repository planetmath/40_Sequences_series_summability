\documentclass[12pt]{article}
\usepackage{pmmeta}
\pmcanonicalname{BohrsTheorem}
\pmcreated{2015-04-13 12:52:55}
\pmmodified{2015-04-13 12:52:55}
\pmowner{pahio}{2872}
\pmmodifier{pahio}{2872}
\pmtitle{Bohr's theorem}
\pmrecord{12}{41951}
\pmprivacy{1}
\pmauthor{pahio}{2872}
\pmtype{Theorem}
\pmcomment{trigger rebuild}
\pmclassification{msc}{40A30}
\pmclassification{msc}{30B10}

% this is the default PlanetMath preamble.  as your knowledge
% of TeX increases, you will probably want to edit this, but
% it should be fine as is for beginners.

% almost certainly you want these
\usepackage{amssymb}
\usepackage{amsmath}
\usepackage{amsfonts}

% used for TeXing text within eps files
%\usepackage{psfrag}
% need this for including graphics (\includegraphics)
%\usepackage{graphicx}
% for neatly defining theorems and propositions
 \usepackage{amsthm}
% making logically defined graphics
%%%\usepackage{xypic}

% there are many more packages, add them here as you need them

% define commands here

\theoremstyle{definition}
\newtheorem*{thmplain}{Theorem}

\begin{document}
\textbf{\PMlinkescapetext{Theorem} (Bohr 1914).}\, If the power series 
$\displaystyle\sum_{n=0}^\infty a_nz^n$ satisfies 
\begin{align}
\left|\sum_{n=0}^\infty a_nz^n\right| \;<\; 1
\end{align}
in the unit disk \,$|z| < 1$,\, then (1) and \PMlinkescapetext{even} the inequality
\begin{align}
\sum_{n=0}^\infty|a_nz^n| \;<\; 1
\end{align}
is true in the disk \,$|z| < \frac{1}{3}$.\, Here, the radius $\frac{1}{3}$ is the best possible.\\

\emph{Proof.}\, One needs \emph{Carath\'eodory's inequality} which says that if the real part of a holomorphic function 
$$g(z) \;:=\; \sum_{n=0}^\infty b_nz^n$$
is positive in the unit disk, then 
$$|b_n| \;\leqq\, 2\,\mbox{Re}\,b_0 \quad \mbox{for}\; n = 1,\,2,\,\ldots$$
Choosing now\, $g(z) := 1\!-\!e^{i\varphi}f(z)$\, where $\varphi$ is any real number and $f(z)$ the sum function of the series in the theorem, we get
$$|a_n| \;\leqq\; 2\,\mbox{Re}\,(1\!-\!e^{i\varphi}a_0) \;=\; 2(1\!-\!a_0\cos\varphi),$$
and especially
$$|a_n| \;\leqq\; 2(1\!-\!|a_0|), \quad \mbox{for}\; n = 1,\,2,\,\ldots$$
If\, $f(z) \not\equiv a_0$,\, in the disk\, $|z| < \frac{1}{3}$\, we thus have
$$\sum_{n=0}^\infty|a_nz^n| \;<\; |a_0|+2(1\!-\!|a_0|)\sum_{n=1}^\infty\left(\frac{1}{3}\right)^n \;=\;1.$$\\
Take then in particular the function defined by
$$f(z) \;:=\; \frac{z\!-\!c}{1\!-\!cz}$$
with\, $0 < c < 1$.\, Its series expansion
$$f(z) \;=\; \sum_{n=0}^\infty a_nz^n \;=\; -c+(1\!-\!c^2)z+(1\!-\!c^2)cz^2+(1\!-\!c^2)c^2z^3+\ldots$$
shows that
$$\sum_{n=0}^\infty|a_nz^n| \;=\; f(|z|)+2c,$$
which last form can be seen to become greater than 1 for\, $\displaystyle|z| > \frac{1}{1\!+\!2c}$.\, Because $c$ may come from below arbitrarily \PMlinkescapetext{near} to 1, one sees that the value $\frac{1}{3}$ in the theorem cannot be increased.



\begin{thebibliography}{9}
\bibitem{Bohr}{\sc Harald Bohr}: ``A theorem concerning power series''. -- \emph{Proc. London Math. Soc.} \textbf{13} (1914).
\bibitem{Boas}{\sc Harold P. Boas}: ``Majorant series''. -- \emph{J. Korean Math. Soc.} \textbf{37} (2000).
\end{thebibliography}


%%%%%
%%%%%
\end{document}
