\documentclass[12pt]{article}
\usepackage{pmmeta}
\pmcanonicalname{ProofOfAlternatingSeriesTest}
\pmcreated{2014-07-22 16:20:39}
\pmmodified{2014-07-22 16:20:39}
\pmowner{Wkbj79}{1863}
\pmmodifier{pahio}{2872}
\pmtitle{proof of alternating series test}
\pmrecord{13}{32936}
\pmprivacy{1}
\pmauthor{Wkbj79}{2872}
\pmtype{Proof}
\pmcomment{trigger rebuild}
\pmclassification{msc}{40A05}

% this is the default PlanetMath preamble.  as your knowledge
% of TeX increases, you will probably want to edit this, but
% it should be fine as is for beginners.

% almost certainly you want these
\usepackage{amssymb}
\usepackage{amsmath}
\usepackage{amsfonts}

% used for TeXing text within eps files
%\usepackage{psfrag}
% need this for including graphics (\includegraphics)
%\usepackage{graphicx}
% for neatly defining theorems and propositions
%\usepackage{amsthm}
% making logically defined graphics
%%%\usepackage{xypic}

% there are many more packages, add them here as you need them

% define commands here
\begin{document}
The series has partial sum
\[
S_{2n+2}=a_1-a_2+a_3-+...-a_{2n}+a_{2n+1}-a_{2n+2},
\] 
where the $a_j$'s are all nonnegative and nonincreasing.  
From above, we have the following:
\begin{align*}
S_{2n+1} & =S_{2n}+a_{2n+1}; \\
\\
S_{2n+2} & =S_{2n}+(a_{2n+1}-a_{2n+2}); \\
\\
S_{2n+3} & =S_{2n+1}-(a_{2n+2}-a_{2n+3}) \\
         & =S_{2n+2}+a_{2n+3}
\end{align*}

Since $a_{2n+1} \geq a_{2n+2}\geq a_{2n+3}$, we have $S_{2n+1}\geq S_{2n+3} \geq S_{2n+2} \geq S_{2n}$.  Moreover,
\[
S_{2n+2}=a_1-(a_2-a_3)-(a_4-a_5)-\cdots-(a_{2n}-a_{2n+1})-a_{2n+2}.
\]
Because the $a_j$'s are nonincreasing, we have $S_n \geq 0$ for any $n$.  Also, $S_{2n+2} \leq S_{2n+1} \leq a_1$.  Thus, $a_1 \geq S_{2n+1} \geq S_{2n+3} \geq S_{2n+2} \geq S_{2n} \geq 0$.  Hence, the even partial sums $S_{2n}$ and the odd partial sums $S_{2n+1}$ are bounded.  Also, the even partial sums $S_{2n}$'s are monotonically nondecreasing, while the odd partial sums $S_{2n+1}$'s are monotonically nonincreasing.  Thus, the even and odd series both converge.

We note that $S_{2n+1}-S_{2n}=a_{2n+1}$.  Therefore, the sums converge to the \emph{same} limit if and only if $a_n\to 0$ as $n\to\infty$. The theorem is then established.
%%%%%
%%%%%
\end{document}
