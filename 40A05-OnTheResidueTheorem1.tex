\documentclass[12pt]{article}
\usepackage{pmmeta}
\pmcanonicalname{OnTheResidueTheorem1}
\pmcreated{2013-03-11 19:29:41}
\pmmodified{2013-03-11 19:29:41}
\pmowner{swapnizzle}{13346}
\pmmodifier{}{0}
\pmtitle{On the Residue Theorem}
\pmrecord{1}{50096}
\pmprivacy{1}
\pmauthor{swapnizzle}{0}
\pmtype{Definition}

\endmetadata

%none for now
\begin{document}
\documentclass[11pt]{article}
\usepackage{amssymb}
\usepackage{amsmath}
\usepackage{amsthm}
\usepackage{amsfonts}
\usepackage{array}
\usepackage[mathcal]{eucal}
\usepackage{xy}
\textheight 9in
\textwidth 7in
\oddsidemargin 0in
\evensidemargin 0in
\topmargin 0in
\headheight 0in
\headsep 0in
\title{On the Residue Theorem}
\author{Swapnil Sunil Jain}
\date{December 26, 2006}
\begin{document}
\maketitle

\subsection*{The Residue Theorem}

If $\gamma$ is a simply closed contour and f is analytic within the region bounded by $\gamma$ except for some finite number of poles $z_0,z_1,...,z_n$ then
\begin{eqnarray*}
\int_{\gamma} f(z) dz &=& 2\pi i \sum_{k=0}^{n} Res_{z=z_k} f(z)
\end{eqnarray*}
where $Res_{z=z_k} f(z)$ is the reside of $f(z)$ at $z_k$.

\subsection*{Calculating Residues}

The Residue of $f(z)$ at a particular pole $p$ depends on the characteristic of the pole.

For a single pole $p$, $Res_{z=p} f(z) = \lim_{z \to p} \Big[ (z-p)f(z) \Big]$

For a double pole $p$, $Res_{z=p} f(z) = \lim_{z \to p} \Big[ \frac{d}{dz} (z-p)^2 f(z) \Big]$

For a n-tuple pole $p$, $Res_{z=p} f(z) = \lim_{z \to p} \Big[ \frac{1}{(n-1)!}\frac{d^{(n-1)}}{dz^{(n-1)}} (z-p)^n f(z) \Big]$
 
\subsection*{Evaluation of Real-Valued Definite Integrals}

We can use the Residue theorem to evaluate real-valued definite integral of the form
\begin{eqnarray}
&& \int_{0}^{2\pi} f(\sin(n\theta), \cos(n\theta)) d\theta
\end{eqnarray}
If we let $z=e^{i\theta}$, then $\frac{dz}{d\theta} = ie^{i\theta} = iz$ which implies that $d\theta = \frac{dz}{iz}$. Then using the identity $\cos(n\theta) = \frac{1}{2}(z^{n} + z^{-n})$ and $\sin(n\theta) = \frac{1}{2i}(z^{n} - z^{-n})$, we can re-write (1) as 
\begin{eqnarray}
&& \int_{\gamma} g(z) \frac{dz}{iz}
\end{eqnarray}
where $g(z) = f( \frac{1}{2i}(z^{n} - z^{-n}), \frac{1}{2}(z^{n} + z^{-n}))$ and $\gamma$ is a contour that traces the unit circle. Then, by the Residue theorem, (2) is equal to 
\begin{eqnarray*}
&& 2\pi i \sum_{k=0}^{n} Res_{z=z_k} (\frac{g(z)}{iz}) = \frac{2\pi i}{i} \sum_{k=0}^{n} Res_{z=z_k} (\frac{g(z)}{z}) = 2\pi \sum_{k=0}^{n} Res_{z=z_k} (\frac{g(z)}{z})
\end{eqnarray*}
where $z_k$ are the poles of $\frac{g(z)}{z}$. 

\end{document}
%%%%%
\end{document}
