\documentclass[12pt]{article}
\usepackage{pmmeta}
\pmcanonicalname{RiemannsTheoremOnRearrangements}
\pmcreated{2013-03-22 17:31:52}
\pmmodified{2013-03-22 17:31:52}
\pmowner{Gorkem}{3644}
\pmmodifier{Gorkem}{3644}
\pmtitle{Riemann's theorem on rearrangements}
\pmrecord{23}{39928}
\pmprivacy{1}
\pmauthor{Gorkem}{3644}
\pmtype{Theorem}
\pmcomment{trigger rebuild}
\pmclassification{msc}{40A05}
\pmsynonym{Riemann series theorem}{RiemannsTheoremOnRearrangements}
\pmrelated{UnconditionallyConvergent}
\pmrelated{FiniteChangesInConvergentSeries}
\pmrelated{FiniteChangesInConvergentSeries2}

\usepackage{amssymb}
\usepackage{amsmath}
\usepackage{amsfonts}
\usepackage{mathrsfs}
\usepackage{amssymb,amsbsy}
\usepackage{graphicx,color}
\usepackage{epsfig}


% used for TeXing text within eps files
%\usepackage{psfrag}
% need this for including graphics (\includegraphics)
%\usepackage{graphicx}
% for neatly defining theorems and propositions
%\usepackage{amsthm}
% making logically defined graphics
%%%\usepackage{xypic}

% there are many more packages, add them here as you need them

% define commands here





\newtheorem{thm}{Theorem}[section]
\newtheorem{defn}[thm]{Definition}
\newtheorem{lemma}[thm]{Lemma}
\newtheorem{prop}[thm]{Proposition}
\newtheorem{rk}[thm]{Remark}
\newtheorem{crl}[thm]{Corollary}
\newtheorem{stp}{Step}

\newcommand{\disp}{\displaystyle}
\newcommand{\dintl}{\disp\int\limits}
\newcommand{\dsuml}{\disp\sum\limits}
\newcommand{\hsp}{\hspace{30pt}}
\newcommand{\ba}{\begin{array}}
\newcommand{\ea}{\end{array}}
\newcommand{\trns}{\,\widehat{}\,\,}

\newcommand{\bthm}{\begin{thm}}
\newcommand{\ethm}{\end{thm}}
\newcommand{\bstp}{\begin{stp}}
\newcommand{\estp}{\end{stp}}
\newcommand{\blemma}{\begin{lemma}}
\newcommand{\elemma}{\end{lemma}}
\newcommand{\bprop}{\begin{prop}}
\newcommand{\eprop}{\end{prop}}
\newcommand{\bpf}{\begin{pf}}
\newcommand{\epf}{\end{pf}}
\newcommand{\bdefn}{\begin{defn}}
\newcommand{\edefn}{\end{defn}}
\newcommand{\brk}{\begin{rk}}
\newcommand{\erk}{\end{rk}}
\newcommand{\bcrl}{\begin{crl}}
\newcommand{\ecrl}{\end{crl}}


\newcommand{\norm}[1]{\left\|#1\right\|}
\newcommand{\brackets}[1]{\left[#1\right]}
\newcommand{\beqn}{\begin{equation}}
\newcommand{\eeqn}{\end{equation}}
\newcommand{\supnorm}[1]{\norm{#1}_\infty}
\newcommand{\normt}[1]{\norm{#1}_2}
\newcommand{\ip}[2]{\left\langle#1 , #2\right\rangle}
\newcommand{\supp}{\operatorname{supp}}
\newcommand{\calg}[1]{\mathcal{#1}}


\newcommand{\sinc}{\operatorname{sinc}}
\newcommand{\qed}{$\ \ \ \ \ \Box$}
\newcommand{\qedin}{\ \ \ \ \ \Box}
\newcommand{\Tr}{\operatorname{\Tr}}

\newenvironment{pf}{\begin{trivlist}\item[\hskip%
\labelsep{\bf Proof.}]}
{\rm\end{trivlist}}

\begin{document}
\PMlinkescapeword{order}

If the map $n \mapsto n'$ is a bijection on $\mathbb{N}$, we say that the sequence $(a_{n'})$ is a rearrangement of $(a_{n})$. 

The following theorem, which is due to Riemann, shows that the convergence of a conditionally convergent series depends so much on the order of its terms; in particular, a conditionally convergent series can be made to converge to any real number by changing the order of its terms. 

\textbf{Theorem (Riemann series theorem).}
Let $(a_n)$ be a sequence in $\mathbb{R}$  such that $\sum_{n=1}^\infty a_n$ converges but $\sum_{n=1}^\infty |a_n| = \infty$, i.e, $\sum a_n$ is conditionally convergent.   Let $-\infty\leq \alpha < \beta \leq \infty $ be arbitrary. Then there exists a rearrangement $(a_{n'})$  such that 
$$
\liminf_N \sum_{n'=1}^N a_{n'} = \alpha \ \ \ \mbox{and} \ \ \  \limsup_N \sum_{n'=1}^N a_{n'} = \beta.
$$


{\em Proof.}
Let $a_n^+ = \max\{0,a_n\}$ and $a_n^- = \min\{0,-a_n\}$.  Then we have $a_n = a_n^+ - a_n^-$ and $|a_n| = a_n^+ + a_n^-$.  Since $\sum a_n < \infty$, both $\sum a_n^+$ and $\sum a_n^-$ diverge or converge simultaneously. But since $\sum|a_n|=\infty$, we see that at least one of $\sum a_n^+$ and $\sum a_n^-$ must diverge.  It follows that  $\sum a_n^+ = \infty$ and $\sum a_n^- = \infty$.

Also by the $n$th term test, $\lim_n a_n^+ = \lim_n a_n^- = 0$. 

Now we pass to subsequence of $(a_n^+)$ by removing all terms with $a_n^+ = 0$ and $a_n \neq 0$.  For $a_n^-$, we remove all terms with $a_n^- = 0$.   Let us denote the subsequences still as $(a_n^+)$ and $(a_n^-)$.  Since only zeros have been removed, $\sum a_n^+$ and $\sum a_n^-$ are still divergent.

Now we will define integers $m_j$ and $k_j$ for $j\in\mathbb{N}$, and consider the series
\begin{align*}
a^+_1 & + a^+_2 + \cdots + a^+_{m_1}\\
&- a^-_1 - a^-_2 - \cdots - a^-_{k_2}\\
& + a^+_{m_1+1}  + a^+_2 + \cdots + a^+_{m_2}\\
&- a^-_{k_1+1} - a^-_2 - \cdots - a^-_{k_2}+ \\
& \ \ \ \vdots 
\end{align*}
This series is clearly a rearrangement of $\sum a_n$, by our choice of the subsequences $a^+_n$ and $a^-_n$.  

We pick up two sequences $\alpha_j$ and $\beta_j$ such that $\alpha_j\rightarrow \alpha $ , $\beta_j \rightarrow \beta$, $\alpha_n \leq \beta_n$ and  $\beta_1 >0$.  We choose $m_1$ such that $\sum_{n=1}^{m_1} \geq \beta_1$ but $\sum_{n=1}^{m_1-1} < \beta_1$. We choose $k_1$ such that $\sum_{n=1}^{m_1}  - \sum_{n=1}^{k_1}\leq \alpha_1$ but $\sum_{n=1}^{m_1}  - \sum_{n=1}^{k_1-1}>\alpha_1$.  We continue this way, inductively.


Since $\lim_n a_n^+ = \lim_n a_n^- = 0$, the subsequences of the sequence of partial sums that end with $a^+_{m_j}$ and $a^-_{k_j}$ converge to $\beta$ and $\alpha$, and it can be seen that no subsequence can be found with a limit larger than $\beta$ or lower than $\alpha$. \qed

\begin{thebibliography}{9}
\bibitem{rudin}Rudin, W., \textsl{Principles of Mathematical Analysis},  McGraw Hill, 1976.

\end{thebibliography}
%%%%%
%%%%%
\end{document}
