\documentclass[12pt]{article}
\usepackage{pmmeta}
\pmcanonicalname{AddingAndRemovingParenthesesInSeries}
\pmcreated{2013-03-22 18:54:09}
\pmmodified{2013-03-22 18:54:09}
\pmowner{pahio}{2872}
\pmmodifier{pahio}{2872}
\pmtitle{adding and removing parentheses in series}
\pmrecord{13}{41750}
\pmprivacy{1}
\pmauthor{pahio}{2872}
\pmtype{Topic}
\pmcomment{trigger rebuild}
\pmclassification{msc}{40A05}
\pmrelated{EmptySum}

\endmetadata

% this is the default PlanetMath preamble.  as your knowledge
% of TeX increases, you will probably want to edit this, but
% it should be fine as is for beginners.

% almost certainly you want these
\usepackage{amssymb}
\usepackage{amsmath}
\usepackage{amsfonts}

% used for TeXing text within eps files
%\usepackage{psfrag}
% need this for including graphics (\includegraphics)
%\usepackage{graphicx}
% for neatly defining theorems and propositions
 \usepackage{amsthm}
% making logically defined graphics
%%%\usepackage{xypic}

% there are many more packages, add them here as you need them

% define commands here

\theoremstyle{definition}
\newtheorem*{thmplain}{Theorem}

\begin{document}
We consider series with real or complex terms.

\begin{itemize}

\item If one groups the terms of a convergent series by adding parentheses but not changing the order of the terms, the series remains convergent and its sum the same. (See theorem 3 of the \PMlinkid{parent entry}{6517}.)

\item A divergent series can become convergent if one adds an infinite amount of parentheses; e.g.\\
$1-1+1-1+1-1+-\ldots$ diverges but $(1-1)+(1-1)+(1-1)+\ldots$ converges.

\item A convergent series can become divergent if one removes an infinite amount of parentheses; cf. the preceding example.

\item If a series \PMlinkescapetext{contains} parentheses, they can be removed if the obtained series converges; in this case also the original series converges and both series have the same sum.

\item If the series
\begin{align}
(a_1+\ldots+a_r)+(a_{r+1}+\ldots+a_{2r})+(a_{2r+1}+\ldots+a_{3r})+\ldots
\end{align}
converges and
\begin{align}
\lim_{n\to\infty}a_n \;=\; 0,
\end{align}
then also the series 
\begin{align}
a_1+a_2+a_3\ldots
\end{align}
converges and has the same sum as (1).\\

\emph{Proof.}\, Let $S$ be the sum of the (1).\, Then for each positive integer $n$, there exists an integer $k$ such that\, $kr < n \leqq (k\!+\!1)r$.\, The partial sum of (3) may be written
$$a_1+\ldots+a_n \;=\; \underbrace{(a_1+\ldots+a_{kr})}_{s}+\underbrace{(a_{kr+1}+\ldots+a_n)}_{s'}.$$
When\, $n \to \infty$, we have
$$s \to S$$
by the convergence of (1) to $S$, and
$$s' \to 0$$
by the condition (2).\, Therefore the whole partial sum will tend to $S$, Q.E.D.\\


\textbf{Note.}\, The parenthesis expressions in (1) need not be ``equally long'' --- it suffices that their lengths are under an finite bound.

\end{itemize}

%%%%%
%%%%%
\end{document}
