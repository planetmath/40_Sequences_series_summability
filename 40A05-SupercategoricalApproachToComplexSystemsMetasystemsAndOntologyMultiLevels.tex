\documentclass[12pt]{article}
\usepackage{pmmeta}
\pmcanonicalname{SupercategoricalApproachToComplexSystemsMetasystemsAndOntologyMultiLevels}
\pmcreated{2013-03-11 19:53:05}
\pmmodified{2013-03-11 19:53:05}
\pmowner{bci1}{20947}
\pmmodifier{}{0}
\pmtitle{Supercategorical Approach to Complex Systems, Meta-systems  and Ontology Multi-Levels}
\pmrecord{1}{50159}
\pmprivacy{1}
\pmauthor{bci1}{0}
\pmtype{Definition}

%none for now
\begin{document}
% comaredv13bTAO1.tex 11/21/07
%% Title: Towards Non-Abelian SpaceTime Ontology in Highly Complex Systems: I. Ion Baianu, Ronald Brown and James F. Glazebrook
%% format:  LaTeX2e
%% complexr10.tex reduced  Acomared19_TAO1proceeds3 on 08/19/08

\documentclass[9pt]{amsart}
\usepackage{amsmath, amssymb, amsfonts, amsthm, amscd, latexsym,color,enumerate}
\usepackage{xypic}
\xyoption{curve}
\usepackage[mathscr]{eucal}

\setlength{\textwidth}{7.1in}
%\setlength{\textwidth}{16cm}
\setlength{\textheight}{9.2in}
%\setlength{\textheight}{24cm}

\hoffset=-1.0in     %%ps format
%\hoffset=-1.0in     %%hp format
\voffset=-.30in

%the next gives two direction arrows at the top of a 2 x 2 matrix

\newcommand{\directs}[2]{\def\objectstyle{\scriptstyle}  \objectmargin={0pt}
\xy
(0,4)*+{}="a",(0,-2)*+{\rule{0em}{1.5ex}#2}="b",(7,4)*+{\;#1}="c"
\ar@{->} "a";"b" \ar @{->}"a";"c" \endxy }

\theoremstyle{plain}
\newtheorem{lemma}{Lemma}[section]
\newtheorem{proposition}{Proposition}[section]
\newtheorem{theorem}{Theorem}[section]
\newtheorem{corollary}{Corollary}[section]
\newtheorem{conjecture}{Conjecture}[section]

\theoremstyle{definition}
\newtheorem{definition}{Definition}[section]
\newtheorem{example}{Example}[section]
%\theoremstyle{remark}
\newtheorem{remark}{Remark}[section]
\newtheorem*{notation}{Notation}
\newtheorem*{claim}{Claim}


\theoremstyle{plain}
\renewcommand{\thefootnote}{\ensuremath{\fnsymbol{footnote}}}
\numberwithin{equation}{section}
\newcommand{\Ad}{{\rm Ad}}
\newcommand{\Aut}{{\rm Aut}}
\newcommand{\Cl}{{\rm Cl}}
\newcommand{\Co}{{\rm Co}}
\newcommand{\DES}{{\rm DES}}
\newcommand{\Diff}{{\rm Diff}}
\newcommand{\Dom}{{\rm Dom}}
\newcommand{\Hol}{{\rm Hol}}
\newcommand{\Mon}{{\rm Mon}}
\newcommand{\Hom}{{\rm Hom}}
\newcommand{\Ker}{{\rm Ker}}
\newcommand{\Ind}{{\rm Ind}}
\newcommand{\IM}{{\rm Im}}
\newcommand{\Is}{{\rm Is}}
\newcommand{\ID}{{\rm id}}
\newcommand{\GL}{{\rm GL}}
\newcommand{\Iso}{{\rm Iso}}
\newcommand{\Sem}{{\rm Sem}}
\newcommand{\St}{{\rm St}}
\newcommand{\Sym}{{\rm Sym}}
\newcommand{\SU}{{\rm SU}}
\newcommand{\Tor}{{\rm Tor}}
\newcommand{\U}{{\rm U}}

\newcommand{\A}{\mathcal A}
\newcommand{\D}{\mathcal D}
\newcommand{\E}{\mathcal E}
\newcommand{\F}{\mathcal F}
\newcommand{\G}{\mathcal G}
\newcommand{\R}{\mathcal R}
\newcommand{\cS}{\mathcal S}
\newcommand{\cU}{\mathcal U}
\newcommand{\W}{\mathcal W}

\newcommand{\Ce}{\mathsf{C}}
\newcommand{\Q}{\mathsf{Q}}
\newcommand{\grp}{\mathsf{G}}
\newcommand{\dgrp}{\mathsf{D}}

\newcommand{\bA}{\mathbb{A}}
\newcommand{\bB}{\mathbb{B}}
\newcommand{\bC}{\mathbb{C}}
\newcommand{\bD}{\mathbb{D}}
\newcommand{\bE}{\mathbb{E}}
\newcommand{\bF}{\mathbb{F}}
\newcommand{\bG}{\mathbb{G}}
\newcommand{\bK}{\mathbb{K}}
\newcommand{\bM}{\mathbb{M}}
\newcommand{\bN}{\mathbb{N}}
\newcommand{\bO}{\mathbb{O}}
\newcommand{\bP}{\mathbb{P}}
\newcommand{\bR}{\mathbb{R}}
\newcommand{\bV}{\mathbb{V}}
\newcommand{\bZ}{\mathbb{Z}}

\newcommand{\bfE}{\mathbf{E}}
\newcommand{\bfX}{\mathbf{X}}
\newcommand{\bfY}{\mathbf{Y}}
\newcommand{\bfZ}{\mathbf{Z}}

\renewcommand{\O}{\Omega}
\renewcommand{\o}{\omega}
\newcommand{\vp}{\varphi}
\newcommand{\vep}{\varepsilon}

\newcommand{\diag}{{\rm diag}}
\newcommand{\desp}{{\mathbb D^{\rm{es}}}}
\newcommand{\Geod}{{\rm Geod}}
\newcommand{\geod}{{\rm geod}}
\newcommand{\hgr}{{\mathbb H}}
\newcommand{\mgr}{{\mathbb M}}
\newcommand{\ob}{\operatorname{Ob}}
\newcommand{\obg}{{\rm Ob(\mathbb G)}}
\newcommand{\obgp}{{\rm Ob(\mathbb G')}}
\newcommand{\obh}{{\rm Ob(\mathbb H)}}
\newcommand{\Osmooth}{{\Omega^{\infty}(X,*)}}
\newcommand{\ghomotop}{{\rho_2^{\square}}}
\newcommand{\gcalp}{{\mathbb G(\mathcal P)}}

\newcommand{\rf}{{R_{\mathcal F}}}
\newcommand{\glob}{{\rm glob}}
\newcommand{\loc}{{\rm loc}}
\newcommand{\TOP}{{\rm TOP}}

\newcommand{\wti}{\widetilde}
\newcommand{\what}{\widehat}

\renewcommand{\a}{\alpha}
\newcommand{\be}{\beta}
\newcommand{\ga}{\gamma}
\newcommand{\Ga}{\Gamma}
\newcommand{\de}{\delta}
\newcommand{\del}{\partial}
\newcommand{\ka}{\kappa}
\newcommand{\si}{\sigma}
\newcommand{\ta}{\tau}


\newcommand{\lra}{{\longrightarrow}}
\newcommand{\ra}{{\rightarrow}}
\newcommand{\rat}{{\rightarrowtail}}
\newcommand{\oset}[1]{\overset {#1}{\ra}}
\newcommand{\osetl}[1]{\overset {#1}{\lra}}
\newcommand{\hr}{{\hookrightarrow}}


\newcommand{\hdgb}{\boldsymbol{\rho}^\square}
\newcommand{\hdg}{\rho^\square_2}

\newcommand{\med}{\medbreak}
\newcommand{\medn}{\medbreak \noindent}
\newcommand{\bign}{\bigbreak \noindent}

\renewcommand{\leq}{{\leqslant}}
\renewcommand{\geq}{{\geqslant}}

\def\red{\textcolor{red}}
\def\magenta{\textcolor{magenta}}
\def\blue{\textcolor{blue}}
\def\<{\langle}
\def\>{\rangle}
\begin{document}

\title[Towards A Non-Abelian SpaceTime Ontology in Highly Complex Systems:]
{SUPERCATEGORICAL ONTOLOGY OF COMPLEX SYSTEMS, META--SYSTEMS AND LEVELS: \\
The Emergence of Life, Human Consciousness and Society.}

 \textbf{acomared19-TAO1proceeds3.tex}

\date{August 19th, 2008}

\author[I. C. Baianu, R. Brown and J. F. Glazebrook]
{I. C. Baianu, R. Brown and J. F. Glazebrook}

\address{University of Illinois at Urbana--Champaign\\
FSHN and NPRE Departments\\AFC--NMR and NIR Microspectroscopy Facility
\\ Urbana IL 61801 USA}

\email[I. C. Baianu]{ibaianu@uiuc.edu}

\address{School of Informatics\\
University of Wales\\ Dean St. Bangor\\ Gwynedd LL57 1UT UK}

\email[R. Brown]{r.brown@bangor.ac.uk}

\address{Department of Mathematics and Computer Science\\
 Eastern Illinois University\\ 600 Lincoln Ave.
 \\Charleston IL 61920--3099 USA}

\email[J. F. Glazebrook]{jfglazebrook@eiu.edu}


\subsection{Mathematical and Metaphysics Notes.}

\subsubsection{AN-1. \emph{On the Logical Foundations of Arithmetics}}

 According to a website contributed entry (at $http://www.philosophypages.com/hy/6h.htm$):
  ``The culmination of the new approach to logic lay in its capacity to illuminate the nature of the mathematical reasoning. While the idealists sought to reveal the internal coherence of absolute reality and the pragmatists offered to account for human inquiry as a loose pattern of investigation, the new logicians hoped to show that the most significant relations among things could be understood as \emph{purely formal and external}. \emph{Mathematicians like Richard Dedekind realized that on this basis it might be possible to establish mathematics firmly on logical grounds}. Giuseppe Peano had demonstrated in 1889 that \emph{all of arithmetic could be reduced to an axiomatic system with a carefully restricted set of preliminary postulates}. Frege promptly sought to express these postulates in the \emph{symbolic notation} of his own invention. By 1913, Russell and Whitehead had completed the monumental ``Principia Mathematica'' (1913), taking \emph{three massive volumes to move from a few logical axioms through a definition of number} to a proof that `` 1 + 1 = 2 .'' Although the work of G\"odel (less than two decades later) made clear the \emph{inherent limitations of this approach}, \emph{its significance} for our understanding of logic \emph{and mathematics remains}''.
  
 
\textbf{AN-2.7}:
\subsubsection{Local--to--Global (LG) Construction Principles consistent with Quantum `Axiomatics'.}

 A novel approach to QST construction in Algebraic/Axiomatic QFT involves the use
of generalized fundamental theorems of algebraic topology from
specialized, `globally well-behaved' topological spaces, to
arbitrary ones (Baianu et al, 2007c). In this category, are the generalized, \emph{Higher
Homotopy van Kampen theorems (HHvKT)} of Algebraic Topology with
novel and unique non-Abelian applications. Such theorems greatly aid
the calculation of higher homotopy of topological spaces.  R. Brown and coworkers (1999, 2004a,b,c)  generalized the van Kampen theorem, at first to fundamental  groupoids on a set of base points (Brown,1967), and then, to higher dimensional algebras involving, for example,
homotopy double groupoids and 2-categories (Brown, 2004a). The more sensitive \emph{algebraic invariant} of topological spaces seems to be, however, captured only by \emph{cohomology} theory through an algebraic \emph{ring} structure that is not accessible either in
homology theory, or in the existing homotopy theory.  Thus, two arbitrary topological spaces that have isomorphic homology groups may not have isomorphic cohomological ring structures, and may also
not be homeomorphic, even if they are of the same homotopy type. 
Furthermore, several \emph {non-Abelian} results in algebraic topology could only be derived from the Generalized van Kampen Theorem (cf. Brown, 2004a), so that
one may find links of such results to the expected \emph
{`non-commutative} geometrical' structure of quantized space--time
(Connes, 1994). In this context, the important algebraic--topological concept of a \emph{Fundamental Homotopy Groupoid (FHG) is applied to a Quantum Topological Space (QTS)} as a ``partial classifier" of the \emph{invariant} topological properties
of quantum spaces of \emph{any} dimension; quantum topological
spaces are then linked together in a \emph{crossed complex over a
quantum groupoid} (Baianu, Brown and Glazebrook, 2006), thus
suggesting the construction of global topological structures from
local ones with well-defined quantum homotopy groupoids. The latter
theme is then further pursued through defining locally topological
groupoids that can be globally characterized by applying the
Globalization Theorem, which involves the \emph{unique} construction
of the Holonomy Groupoid. We are considering in a separate publication(Baianu et al 2007c) how such concepts  might be applied in the context of Algebraic or Axiomatic Quantum 
Field Theory (AQFT) to provide a local-to-global construction of Quantum space-times which would still be valid in the presence of intense gravitational fields without generating singularities as in GR. The result of such a construction is a \emph{Quantum Holonomy Groupoid}, (QHG) which is unique up to an isomorphism.

\subsubsection{The Object-Based Approach \emph{vs} Process-Based, \emph{Dynamic Ontology}.}

 In classifications, such as those developed over time in Biology
for organisms, or in Chemistry for chemical elements, the
\emph{objects} are the basic items being classified even if the
`ultimate' goal may be, for example, either evolutionary or
mechanistic studies. Rutherford's comment is pertinent in this
context: \emph{``There are two major types of science: physics or stamp collecting." }
 An ontology based strictly on object classification may have
little to offer from the point of view of its cognitive content.
It is interesting that many psychologists, especially behavioural
ones, emphasize the object-based approach rather than the
process-based approach to the ultra-complex process of
consciousness occurring `in the mind' --with the latter thought as
an `object'. Nevertheless, as early as the work of William James
in 1850, consciousness was considered as a \emph{`continuous
stream that never repeats itself}'--a Heraclitian concept that
does also apply to super-complex systems and life, in general. We
shall see more examples of the object-based approach to psychology
in \textbf{Section 8}.

\subsubsection{Procedures and Advantages of Poli's Ontological Theory of Levels}

According to Poli (2001), the
ontological procedures provide:

\begin{itemize}
\item coordination between categories (for instance, the
interactions and parallels between biological and ecological
reproduction as in Poli, 2001);
\item modes of dependence between levels (for instance, how the
co-evolution/interaction of social and mental realms depend and
impinge upon the material);
\item the categorical closure (or completeness) of levels.
\end{itemize}
Already we can underscore a significant component of this essay
that relates the ontology to geometry and topology; specifically, if a
level is defined \emph{via} `iterates of local procedures' (viz. `items in
iteration', Poli, 2001), then we have some handle on describing
its intrinsic governing dynamics (with feedback ) and, to quote
Poli (2001), to `restrict the \emph{multi-dynamic} frames to their linear
fragments'. On each level of this ontological hierarchy there is a significant
amount of connectivity through inter-dependence, interactions or
general relations often giving rise to complex patterns that are
not readily analyzed by partitioning or through stochastic methods
as they are neither simple, nor are they random connections. But we
claim that such complex patterns and processes have their
logico-categorical representations quite apart from classical,
Boolean mechanisms. This ontological situation gives rise to a
wide variety of networks, graphs, and/or mathematical categories,
all with different connectivity rules, different types of
activities, and also \emph{a hierarchy of super-networks of networks of sub-networks}. 
Then, the important question arises what types of
basic symmetry or patterns such super-networks of items can have,
and also how do the effects of their sub-networks `percolate' through the
various levels. From the categorical viewpoint, these are of two
basic types: they are either \emph{commutative} or \emph{non-commutative},
where, at least at the quantum level, the latter takes precedence over the former.

 It is often thought or taken for granted that the \emph{object-oriented} approach can be readily converted into a process-based one. It would seem, however, that the answer to this question depends critically on the ontological level selected. For example, at the quantum level, \emph{object and process become inter-mingled}. Either comparing or moving between
levels, requires ultimately a \emph{process-based} approach, especially
in Categorical Ontology where relations and inter-process
connections are essential to developing any valid theory. At the
fundamental level of `elementary particle physics' however the
answer to this question of process-vs. object becomes quite
difficult as a result of the `blurring' between the particle and
the wave concepts. Thus, it is well-known that any `elementary
quantum object' is considered by all accepted versions of quantum
theory not just as a `particle' or just a `wave' but both: the
quantum `object' is \emph{both} wave and particle, \emph{at the
same-time}, a proposition accepted since the time when it was
proposed by de Broglie. At the quantum microscopic level, the
object and process are inter-mingled, they are no longer separate
items. Therefore, in the quantum view the `object-particle' and
the dynamic process-`wave' are united into a single dynamic entity
or item, called \emph{the wave-particle quantum}, which strangely enough
is \emph{neither discrete nor continuous}, but both at the same time, thus
`refusing' intrinsically to be an item consistent with Boolean
logic. Ontologically, the quantum level is a fundamentally important
starting point which needs to be taken into account by any theory
of levels that aims at completeness. Such completeness may not be
attainable, however, simply because an `extension' of G\"odel's theorem may
hold here also. The fundamental quantum level is generally accepted to be
dynamically, or intrinsically \emph{non-commutative}, in the sense
of the\emph{ non-commutative quantum logic} and also in the sense of
\emph{non-commuting quantum operators} for the essential quantum
observables such as position and momentum. Therefore, any comprehensive theory of levels, in the sense of incorporating the quantum level, is thus --\emph{mutatis
mutandis}-- \emph {non-Abelian}. Furthermore, as the non-Abelian case is the more general one, from a strictly formal viewpoint, a non-Abelian Categorical Ontology is
the preferred choice. A paradigm-shift towards a \emph{non-Abelian Categorical Ontology} has already started (Brown et al, 2007: \emph{`Non-Abelian Algebraic Topology'}; Baianu, Brown and Glazebrook, 2006: NA-QAT; Baianu et al 2007a,b,c).\\

\med
\subsubsection{Fundamental Concepts of Algebraic Topology with Potential Application to Ontology Levels Theory and Space-Time Structures.}
We shall consider briefly the potential impact of novel Algebraic Topology concepts, methods and results on the problems of defining and classifying rigorously Quantum space-times. With the advent of Quantum Groupoids--generalizing Quantum Groups, Quantum Algebra and Quantum Algebraic Topology, several fundamental concepts and new theorems of Algebraic Topology may also acquire an enhanced importance through their potential applications to current problems in theoretical
and mathematical physics, such as those described in an available preprint (Baianu, Brown and Glazebrook, 2006), and also in several recent publications (Baianu et al 2007a,b; Brown et al 2007). 

 Now, if quantum mechanics is to reject the notion of a continuum,
then it must also reject the notion of the real line and the notion
of a path. How then is one to construct a homotopy theory?
One possibility is to take the route signalled by \v{C}ech, and which
later developed in the hands of Borsuk into `Shape Theory' (see,
Cordier and Porter, 1989). Thus a quite general space is studied by
means of its approximation by open covers. Yet another possible
approach is briefly pointed out in \textbf{AN-2.6}. A few fundamental concepts of Algebraic Topology and Category Theory are summarized in \textbf{AN-2.6} that have an extremely wide range of applicability to the higher complexity levels of reality as well as to the fundamental, quantum level(s). We have omitted in this section the technical details in order to focus only on the ontologically-relevant aspects; full mathematical details are however also available in a recent paper by Brown et al (2007) that focuses on a mathematical/conceptual framework for a completely formal approach to categorical ontology and the theory of levels.

\subsubsection{Towards Biological Postulates and Principles.}

Often, Rashevsky considered in his Relational Biology papers, and
indeed made comparisons, between established physical theories and principles. He was searching for new, more general relations in Biology and Sociology that were also compatible with the former. Furthermore, Rashevsky also proposed two biological principles that add to
Darwin's natural selection of species and the `survival of the fittest principle', \emph{the emergent
relational structure thus defining adaptive organisms}:

 \textbf{1. The Principle of Optimal Design},
 and
 \textbf{2. The Principle of Relational Invariance} (phrased by Rashevsky as \emph{``Biological Epimorphism"}).

  In essence, the `Principle of Optimal Design' defines the `fittest' organism which survives in the natural selection process of competition between species, in terms of an extremal
criterion, similar to that of Maupertuis; the optimally `designed'
organism is that which acquires maximum functionality essential to
survival of the successful species at the lowest `cost' possible.
The `costs' are defined in the context of the environmental niche
in terms of material, energy, genetic and organismic processes
required to produce/entail the pre-requisite biological function(s) and
their supporting anatomical structure(s) needed for competitive survival
in the selected niche. Further details were presented by Robert
Rosen in his short but significant book on optimality (1970). The `Principle of
Biological Epimorphism' on the other hand states that the highly specialized
biological functions of higher organisms can be mapped (through an epimorphism) onto those
of the simpler organisms, and ultimately onto those of a (hypothetical) primordial organism (which was assumed to be unique up to an isomorphism or \emph{selection-equivalence}). The latter proposition, as formulated by Rashevsky, is more akin to a postulate
than a principle. However, it was then generalized and re-stated in the form of the
existence of a \emph{limit} in the category of living organisms and their
functional genetic networks ($\textbf{GN}^i$), as a directed family of
objects, $\textbf{GN}^i(-t)$ projected backwards in time (Baianu and Marinescu, 1968), or
subsequently as a super-limit (Baianu, 1970 to 1987; Baianu, Brown, Georgescu and Glazebrook, 2006); then, it was re-phrased as the \emph{Postulate of Relational Invariance}, represented by a \emph{colimit} with the arrow of time pointing forward (Baianu, Brown, Georgescu and
Glazebrook, 2006). Somewhat similarly, a dual principle and colimit construction was invoked
for the ontogenetic development of organisms (Baianu, 1970), and also for
populations evolving forward in time; this was subsequently applied to biological evolution although on a much longer time scale --that of evolution-- also with the arrow of time pointing towards the future in a representation operating through Memory Evolutive Systems (MES) by A. Ehresmann and Vanbremeersch (2006).\\
\med

\subsubsection{Selective Boundaries and Homeostasis. Varying Boundaries \emph{vs} Horizons.}

 Boundaries are especially relevant to \emph{closed} systems. 
According to Poli (2008): \emph{``they serve to distinguish what is internal to the system from what is external to it"}, thus defining the fixed, overall structural topology
of a \emph{closed} system. By virtue of possessing boundaries, a whole (entity) is something on the basis of which there is an interior and an exterior (\emph{viz.} Baianu and Poli, 2007).
One notes however that a boundary, or boundaries, may either change/\emph{vary} or be quite 
\emph{selective}/directional--in the sense of \emph{dynamic fluxes crossing such boundaries}--
if the system is \emph{open}. In the case of an organism that grows and develops it will be  therefore characterized by a \emph{variable topology} that may also depend on the environment, and is thus \emph{context-dependent}, as well. Perhaps one of the simplest example of a system that changes from \emph{closed to open}, and thus has a \emph{variable topology}, is that of a pipe equipped with a functional valve that allows flow in only one direction. On the other hand,
a semi-permeable membrane such as a cellophane, thin-walled 'closed' tube--
that allows water and small molecule fluxes to go through but blocks the transport
of large molecules such as polymers through its pores-- is \emph{selective}
and may be considered as a primitive/'simple' example of an open, 
selective system. Organisms, in general, are \emph{open systems with specific types or patterns of variable topology} which incorporate both the valve and the selectively permeable membrane 
boundaries --albeit much more sophisticated and dynamic than the simple/fixed topology of the cellophane membrane; such variable structures are essential to maintaining  
their stability and also to the control of their internal structural order, of low microscopic entropy. (The formal definition of this important concept of `variable topology'
was introduced in our recent paper (Baianu et al 2007a) in the context of the space-time evolution of organisms, populations and species.)  

 As proposed by Baianu and Poli (2008), an essential feature of boundaries in open systems is that they can be crossed by matter; however, all boundaries may be crossed by either fields
or by quantum wave-particles if the boundaries are sufficiently thin, even in 'closed' systems. The boundaries of closed systems, however, cannot be crossed by molecules or larger particles. \emph{On the contrary, a horizon is something that one cannot reach.} In other words, \emph{a horizon is not a boundary}. This difference between horizon and boundary appears to be useful in distinguishing between systems and their environment (v. \textbf{AN-4.1}). Boundaries may be fixed, clear-cut, or they may be vague/blurred, mobile, varying/variable in time, or again they may be intermediate between these any of these cases, according to how the differentiation is structured. At the beginning of an organism's ontogenetic development, there may be only a slightly asymmetric distribution in perhaps just one direction, but usually still maintaining certain symmetries along other directions or planes. Interestingly, for many multi-cellular organisms, including man, the overall symmetry retained from the beginning of ontogenetic development is bilateral--just one plane of mirror symmetry-- from Planaria to humans. The presence of the head-to-tail asymmetry introduces increasingly marked differences among the various areas of the head, middle, or tail regions as the organism develops. The formation of additional borderline phenomena occurs later as cells divide and differentiate thus causing the organism to grow and develop  

\textbf{v. (AN-4.2.)} 
\textbf{AN4.2}

 Brown and Higgins, 1981a, showed that certain multiple groupoids
equipped with an extra structure called \emph{connections} were
equivalent to another structure called a \emph{crossed complex}
which had already occurred in homotopy theory. such as
\emph{double, or multiple} groupoids (Brown, 2004; 2005). 
For example, the notion of an \emph{atlas} of structures should,
in principle, apply to a lot of interesting, topological and/or
algebraic, structures: groupoids, multiple groupoids, Heyting
algebras, $n$-valued logic algebras and $C^*$-convolution
-algebras. One might incorporate a 3 or 4-valued logic to represent 
genetic dynamic networks in single-cell organisms such as bacteria. Another 
example from the ultra-complex system of the human mind is \emph{synaesthesia}--the case of
extreme communication processes between different types of
`logics' or different levels of `thoughts'/thought processes. The
key point here is \emph{communication}. Hearing has to communicate
to sight/vision in some way; this seems to happen in the human
brain in the audiovisual (neocortex) and in the Wernicke (W)
integrating area in the left-side hemisphere of the brain, that
also communicates with the speech centers or the Broca area, also
in the left hemisphere. Because of this \emph{dual-functional},
quasi-symmetry, or more precisely asymmetry of the human brain, it
may be useful to represent all two-way communication/signalling
pathways in the two brain hemispheres by a \emph{double groupoid}
as an over-simplified groupoid structure that may represent such
quasi-symmetry of the two sides of the human brain. In this case,
the 300 millions or so of neuronal interconnections in the
\emph{corpus callosum} that link up neural network pathways
between the left and the right hemispheres of the brain would be
represented by the geometrical connection in the double groupoid.
The brain's overall \emph{asymmetric} distribution of functions
and neural network structure between the two brain hemispheres may
therefore require a non-commutative, double--groupoid structure
for its relational representation. The potentially interesting
question then arises how one would mathematically represent the
split-brains that have been neurosurgically generated by cutting
just the \emph{corpus callosum}-- some 300 million
interconnections in the human brain (Sperry, 1992). It would seem
that either a crossed complex of two, or several, groupoids, or
indeed a direct product of two groupoids $G_1$ and $G_2$, $G_1
\times G_2$ might provide some of the simplest representations of
the human split-brain. The latter, direct product construction has
a certain kind of built-in commutativity: $(a,b)(c,d) = (ac,bd)$,
which is a form of the interchange law. In fact, from any two
groupoids $G_1$ and $G_2$ one can construct a double groupoid $G_1
\Join G_2$ whose objects are $\ob(G_1)\times \ob(G_2)$. The
internal groupoid `connection' present in the double groupoid
would then represent the remaining basal/`ancient' brain
connections between the two hemispheres, below the corpum callosum
that has been removed by neurosurgery in the split-brain human
patients.

 The remarkable variability observed in such human subjects both
between different subjects and also at different times after the
split-brain (bridge-localized) surgery may very well be accounted
for by the different possible groupoid representations. It may
also be explained by the existence of other, older neural pathways
that remain untouched by the neurosurgeon in the split-brain, and
which re-learn gradually, in time, to at least partially
re-connect the two sides of the human split-brain. The more common
health problem --caused by the senescence of the brain-- could be
approached as a \emph{local-to-global}, super-complex ageing
process represented for example by the \emph{patching} of a
\emph{topological double groupoid atlas} connecting up many local
faulty dynamics in `small' un--repairable regions of the brain
neural network, caused for example by tangles, locally blocked
arterioles and/or capillaries, and also low local oxygen or
nutrient concentrations. The result, as correctly surmised by
Rosen (1987), is a \emph{global}, rather than local, senescence,
super-complex dynamic process.

\textbf{Social Autopoiesis}
Within a social system the autopoiesis of the various components is a necessary
and sufficient condition for realization of  the system itself. In
this respect, the structure of a society as a particular instance
of a social system is determined by the structural framework of
the (autopoietic components) and the sum total of collective
interactive relations. Consequently, the societal framework is
based upon a selection of its component structures in providing a
medium in which these components realize their ontogeny. It is
just through participation alone that an autopoietic system
determines a social system by realizing the relations that are
characteristic of that system. The descriptive and causal notions
are essentially as follows (Maturana and Varela, 1980, Chapter
III):

\begin{itemize}
\item[(1)] Relations of constitution that determine the components
produced constitute the topology in which the autopoiesis is realized.
\item[(2)]
Relations of specificity that determine that the components produced
are the specific ones defined by their participation in autopoiesis.
\item[(3)]
Relations of order that determine that the concatenation of the
components in the relations of specification, constitution and order
are the ones specified by the autopoiesis.
\end{itemize}


\subsection {Propagation and Persistence of Organisms through Space and Time.
Autopoiesis, Survival and Extinction of Species.}

  The autopoietic model of Maturana (1987) claims to explain the
persistence of living systems in time as the consequence of their
structural coupling or \emph{adaptation} as structure determined
systems, and also because of their existence as \emph{molecular}
autopoietic systems with a `closed' network structure. As part of
the autopoietic explanation is the `structural drift', presumably
facilitating evolutionary changes and speciation. One notes that
autopoietic systems may be therefore considered as dynamic
realizations of Rosen's simple \textbf{MR} s.  Similar arguments seem to be
echoed more recently by Dawkins (2003) who claims to explain the
remarkable persistence of biological organisms over geological
timescales as the result of their intrinsic, (super-) complex
adaptive capabilities.

  The point is being often made that it is not the component atoms
that are preserved in organisms (and indeed in `living fosils' for
geological periods of time), but the \emph{structure-function
relational pattern}, or indeed the associated \emph{organismic
categories}, \emph{higher order categories} or \emph{supercategories}. This is a very important point: only the functional organismic structure is `immortal' as it is
being conserved and transmitted from one generation to the next.
Hence the relevance here, and indeed the great importance of the
science of abstract structures and relations, i.e., Mathematics.

 This was the feature that appeared paradoxical or puzzling to
Erwin Schr\"odinger from a quantum theoretical point of view when
he wrote his book ``What is Life?" As individual molecules often
interact through multiple quantum interactions, which are most of
the time causing \emph{irreversible}, molecular or energetic
changes to occur, how can one then explain the hereditary
stability over hundreds of years (\emph{or occasionally, a great
deal longer, NAs}) within  the same genealogy of a family of men?
The answer is that the `actors change but the play does not!'. The
atoms and molecules turn-over, and not infrequently, but the
\emph{structure-function patterns/organismic categories remain
unchanged}/are conserved over long periods of time through
repeated repairs and replacements of the molecular parts that need
repairing, as long as the organism lives. Such stable patterns of
relations are, at least in principle, amenable to logical and
mathematical representation without tearing apart the living
system. In fact, looking at this remarkable persistence of certain
gene subnetworks in time and space from the categorical ontology
and Darwinian viewpoints, the \emph{existence of live `fossils'}
(e.g., a coelacanth found alive in 1923 to have remained unchanged
at great depths in the ocean as a species for 300 million years!)
it is not so difficult to explain; one can attribute the rare
examples of `live fossils' to the lack of `selection pressure in a
very stable niche'. Thus, one sees in such exceptions the lack of
any adaptation apart from those which have already occurred some
300 million years ago. This is by no means the only long lived
species: several species of marine, giant unicellular green algae
with complex morphology from a family called the
\emph{Dasycladales} may have persisted as long as 600 million
years (Goodwin, 1994), and so on. However, the situation of many
other species that emerged through \emph{super-complex
adaptations}--such as the species of \emph{Homo sapiens}--is quite
the opposite, in the sense of marked, super-complex adaptive
changes over much shorter time--scales than that of the
exceptionally `lucky' coelacanths. Clearly, some species, that
were less adaptable, such as the Neanderthals or \emph{Homo
erectus}, became extinct even though many of their functional
genes may be still conserved in \emph{Homo sapiens}, as for
example, through comparison with the more distant chimpanzee
relative. When comparing  the \emph{Homo erectus} fossils with
skeletal remains of modern men one is struck how much closer the
former are to modern man than to either the
\emph{Australopithecus} or the chimpanzee (the last two species
appear to have quite similar skeletons and skulls, and also their
`reconstructed' vocal chords/apparatus would not allow them to
speak). Therefore, if the functional genomes of man and chimpanzee
overlap by about $98 \%$, then the overlap of modern man
functional genome would have to be greater than $99 \%$ with that
of \emph{Homo erectus} of 1 million years ago, if it somehow could
be actually found and measured (but it cannot be, at least not at
this point in time). Thus, one would also wonder if another more
recent hominin than \emph{H. erectus}, such as \emph{Homo
floresiensis}-- which is estimated to have existed between 74,000
and 18,000 years ago on the now Indonesian island of Flores-- may
have been capable of human speech. One may thus consider another
indicator of intelligence such as the size of region 10 of the
dorsomedial prefrontal cortex, which is thought to be associated
with the existence of \emph{self-awareness}; this region 10 is
about the same size in \emph{H. floresiensis} as in modern humans,
despite the much smaller overall size of the brain in the former
(Falk, D. et al., 2005).

 \subsection{Neuro-Groupoids and Cat-Neurons}

 Categorical representations on nerve cells  in the terminology of Ehresmann
and Vanbremeersch (1987,2006) are called `categorical neurons' (or
\emph{cat--neurons} for short). `Consciousness loops' (Edelmann
1989, 1992) and the neuronal workspace of Baars (1988) (see also
Baars and Franklin, 2003) are among an assortment of such local models that
may be consistent with such local categorical representations. Among other notions, there
were proposed several criteria for studying the overall integration of neuronal assemblies 
into an \emph{archetypal core}: the cat--neuron resonates as an
echo that propagates to target concepts through series of
thalamocortical loops in response to the thalamus response to stimuli. 
Mimicking the neuron signalling through synaptic networks, cat--neurons would interact 
according tp certain linking procedures (Baianu, 1972), that could be then studied in the context of
categorical logic, which in its turn may be applied either to semantic
modelling of neural networks (Healy and Caudell, 2006) or possibly the schemata of \emph{adaptive resonance theory} of Grossberg (1999). For such interactive network systems one would expect
\emph{global actions} and \emph{groupoid atlases} to play more instrumental roles as possible realizations of various types of multi--agent systems (Bak et al, 2006). Let one be
aware however, that such models tend to be reductionist in character,
falling somewhere between simple and chaotic (`complex') systems. Although
useful for the industry of higher level automata and robotics, they are unlikely to address at all the ontological problems of the human mind.

As regards to the role played by quantum events in mental processes the situation is different. Although there can be no reasonable doubt that quantum events do occur in
the brain as elsewhere in the material world, there is no substantial,
experimental evidence that quantum events are in any way efficacious or
relevant for those aspects of brain activity that are correlates
of mental activity. Bohm (1990), and Hiley and Pylkk\"annen
(2005) have suggested theories of \emph{active information}
enabling `self' to control brain functions without violating
energy conservation laws. Such ideas are relevant to how quantum
tunneling is instrumental in controlling the engagement of
synaptic exocytosis (Beck and Eccles, 1992) and how the notion of
a `(dendron) mind field' (Eccles, 1986) could alter quantum
transition probabilities as in the case of synaptic vesicular
emission (nevertheless, there are criticisms to this approach as
in Wilson, 1999). 

Attempting to define consciousness runs into somewhat similar problems to those encountered in attempting to define Life, but in many ways far less `tangible'; one can make a long list of important attributes of human consciousness from which one must decide which ones are the essential or primary properties,  and which ones are to be derived from such \emph{primary attributes}
in a rational manner.


 Kant considered that the internal structure of reasoning, or the `pure reason', was
essential to human nature for knowledge of the world but the
inexactness of empirical science amounted to limitations on the
overall comprehension. At the same time,

Kozma et al. (2004) used
network percolation models to analyze phase transitions of
dynamic neural systems such as those embedded within segments of
neuropil. This idea of \emph{neuro-percolation} so provides a
means of passage via transition states within a neurophysiological
hierarchy (viz. levels). But the actual substance of the hierarchy
cannot by itself explain the quality of intention. The
constitution of the latter may be in part consciousness, but
actual neural manifestations, such as for example pain, are
clearly not products of a finite state Turing machine (Searle, 1983).
\bigbreak

AN5-2
section 5
Point (5a) claims that a system should occupy either a macroscopic or a microscopic space-time region, but a system that comes into birth and dies off extremely rapidly may be considered either a short-lived process, or rather, a `resonance' --an instability rather than a system, although it may have significant effects as in the case of `virtual particles', `virtual photons', etc., as in quantum electrodynamics and chromodynamics. Note also that there are many other, different mathematical definitions of systems, ranging from (systems of) coupled differential equations to operator formulations, semigroups, monoids, topological groupoid dynamic systems and dynamic categories. Clearly, the more useful system definitions include algebraic and/or topological structures rather than simple, discrete structure sets, classes or their categories (cf. Baianu, 1970, and Baianu et al., 2006).

 It can be shown that such organizational order must either result in a \emph{stable attractor} or else it should occupy a \emph{stable space-time domain}, which is generally expressed in \emph{closed} systems by the concept of \emph{equilibrium}. On the other hand,
 
 
 Quantum theories (QTs) were developed that are just as elegant mathematically as GR, and they were also physically `validated' through numerous, extremely sensitive and
carefully designed experiments. However, to date,  quantum theories
have not yet been extended, or generalized, to a form capable of recovering the results of Einstein's GR as a quantum field theory over a GR-space-time altered by gravity is not yet available


\emph{\textbf{AUTHORS' AFFILIATIONS:}}

\end{document}
%%%%%
\end{document}
