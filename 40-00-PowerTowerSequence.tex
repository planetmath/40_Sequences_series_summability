\documentclass[12pt]{article}
\usepackage{pmmeta}
\pmcanonicalname{PowerTowerSequence}
\pmcreated{2013-03-22 16:41:58}
\pmmodified{2013-03-22 16:41:58}
\pmowner{pahio}{2872}
\pmmodifier{pahio}{2872}
\pmtitle{power tower sequence}
\pmrecord{8}{38913}
\pmprivacy{1}
\pmauthor{pahio}{2872}
\pmtype{Example}
\pmcomment{trigger rebuild}
\pmclassification{msc}{40-00}
\pmrelated{PowerFunction}
\pmrelated{OrderOfOperations}
\pmrelated{NaturalLogBase}
\pmrelated{FunctionXx}
\pmrelated{SuperexponentiationIsNotElementary}
\pmrelated{ErnstLindelof}
\pmdefines{power tower sequence}

% this is the default PlanetMath preamble.  as your knowledge
% of TeX increases, you will probably want to edit this, but
% it should be fine as is for beginners.

% almost certainly you want these
\usepackage{amssymb}
\usepackage{amsmath}
\usepackage{amsfonts}

% used for TeXing text within eps files
%\usepackage{psfrag}
% need this for including graphics (\includegraphics)
%\usepackage{graphicx}
% for neatly defining theorems and propositions
 \usepackage{amsthm}
% making logically defined graphics
%%%\usepackage{xypic}

% there are many more packages, add them here as you need them

% define commands here

\theoremstyle{definition}
\newtheorem*{thmplain}{Theorem}

\begin{document}
For positive values of $a$, the {\em power tower sequence}
              $$a,\, a^a,\, a^{a^a},\, a^{a^{a^a}},\, \ldots$$
is convergent if and only if
               $$\frac{1}{e^e} \leqq a \leqq e^{\frac{1}{e}},$$
approximately 
                  $$0.065989\leqq a \leqq 1.444667.$$
The limit of the sequence is the least real \PMlinkname{root}{Equation} of the equation
                           $$a^x = x.$$
The proof is found in [1].

\begin{thebibliography}{8}
\bibitem{lindelof}{\sc E. Lindel\"of}: {\em Differentiali- ja integralilasku
ja sen sovellutukset III. Toinen osa.}\, Mercatorin Kirjapaino Osakeyhti\"o, Helsinki (1940).
\end{thebibliography}
%%%%%
%%%%%
\end{document}
