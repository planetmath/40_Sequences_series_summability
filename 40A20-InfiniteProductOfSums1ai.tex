\documentclass[12pt]{article}
\usepackage{pmmeta}
\pmcanonicalname{InfiniteProductOfSums1ai}
\pmcreated{2013-03-22 18:40:01}
\pmmodified{2013-03-22 18:40:01}
\pmowner{pahio}{2872}
\pmmodifier{pahio}{2872}
\pmtitle{infinite product of sums $1\!+\!a_i$}
\pmrecord{6}{41411}
\pmprivacy{1}
\pmauthor{pahio}{2872}
\pmtype{Theorem}
\pmcomment{trigger rebuild}
\pmclassification{msc}{40A20}
\pmclassification{msc}{26E99}
\pmrelated{LimitOfRealNumberSequence}
\pmrelated{DeterminingSeriesConvergence}
\pmrelated{InfiniteProductOfDifferences1A_i}
\pmrelated{AbsoluteConvergenceOfInfiniteProductAndSeries}

% this is the default PlanetMath preamble.  as your knowledge
% of TeX increases, you will probably want to edit this, but
% it should be fine as is for beginners.

% almost certainly you want these
\usepackage{amssymb}
\usepackage{amsmath}
\usepackage{amsfonts}

% used for TeXing text within eps files
%\usepackage{psfrag}
% need this for including graphics (\includegraphics)
%\usepackage{graphicx}
% for neatly defining theorems and propositions
 \usepackage{amsthm}
% making logically defined graphics
%%%\usepackage{xypic}

% there are many more packages, add them here as you need them

% define commands here

\theoremstyle{definition}
\newtheorem*{thmplain}{Theorem}

\begin{document}
\textbf{Lemma.}\, Let the numbers $a_i$ be nonnegative reals.\, The infinite product
\begin{align}
\prod_{i=1}^\infty(1\!+\!a_i) \,=\, (1\!+\!a_1)(1\!+\!a_2)(1\!+\!a_3)\cdots
\end{align}
converges iff the series \;$a_1\!+\!a_2\!+\!a_3\!+\ldots$\; is convergent.\\

{\em Proof.}\, Denote
$$\sum_{i=1}^na_n \;:=\; s_n, \quad \prod_{i=1}^n(1\!+\!a_i) \;:=\; t_n \qquad (n = 1,\,2,\,\ldots).$$
Now\; $\displaystyle 1\!+\!a_i \,\leqq\, 1\!+\!\frac{a_i}{1!}\!+\!\frac{a_i^2}{2!}\!+\ldots \;=\; e^{a_i}$,\, whence
\begin{align}
t_n \;\leqq\; \prod_{i=1}^ne^{a_i} \;=\; e^{s_n}.
\end{align}
We can estimate also downwards:
\begin{align}
t_n \;=\; (1\!+\!a_1)(1\!+\!a_2)\cdots(1\!+\!a_n) \;=\; 
1\!+\!\sum_{i=1}^na_i\!+\ldots+\!a_1a_2\cdots a_n \;>\; \sum_{i=1}^na_i \;=\; s_n
\end{align}
If the series is convergent with sum $S$, then by (2),
$$t_n \;\leqq\; e^{s_n} \;\leqq\; e^S,$$
and since the monotonically nondecreasing sequence \,$\langle t_1,\,t_2,\,t_3,\,\ldots\rangle$\, thus is bounded from above, it converges (cf. limit of nondecreasing sequence).\, So (1) converges.\\

If, on the other hand, the series is divergent, then\, $\displaystyle\lim_{n\to\infty}s_n = \infty$\, and by (3), also\, $\displaystyle\lim_{n\to\infty}t_n = \infty$,\, i.e. the \PMlinkescapetext{product} (1) diverges.



 

%%%%%
%%%%%
\end{document}
