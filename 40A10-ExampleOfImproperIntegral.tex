\documentclass[12pt]{article}
\usepackage{pmmeta}
\pmcanonicalname{ExampleOfImproperIntegral}
\pmcreated{2014-11-07 11:47:42}
\pmmodified{2014-11-07 11:47:42}
\pmowner{pahio}{2872}
\pmmodifier{pahio}{2872}
\pmtitle{example of improper integral}
\pmrecord{6}{41617}
\pmprivacy{1}
\pmauthor{pahio}{2872}
\pmtype{Example}
\pmcomment{trigger rebuild}
\pmclassification{msc}{40A10}
\pmrelated{SubstitutionNotation}

\endmetadata

% this is the default PlanetMath preamble.  as your knowledge
% of TeX increases, you will probably want to edit this, but
% it should be fine as is for beginners.

% almost certainly you want these
\usepackage{amssymb}
\usepackage{amsmath}
\usepackage{amsfonts}

% used for TeXing text within eps files
%\usepackage{psfrag}
% need this for including graphics (\includegraphics)
%\usepackage{graphicx}
% for neatly defining theorems and propositions
%\usepackage{amsthm}
% making logically defined graphics
%%%\usepackage{xypic}

% there are many more packages, add them here as you need them

% define commands here
\newcommand{\sijoitus}[2]%
{\operatornamewithlimits{\Big/}_{\!\!\!#1}^{\,#2}}
\begin{document}
The integrand of 
\begin{align}
I \;=\; \int_0^1\frac{\arctan{x}}{x\sqrt{1\!-\!x^2}}\,dx
\end{align}
is undefined both at the lower and the upper limit.\, However, the value of the improper integral exists and may be found via the more general integral
\begin{align}
I(y) \;=\; \int_0^1\frac{\arctan{xy}}{x\sqrt{1\!-\!x^2}}\,dx.
\end{align}
Denote the integrand of (2) by\, $f(x,\,y)$.\, For any fixed real value $y$, 
$$f(x,\,y) \in O(1) \mbox{\; as \;} x \to 0, \quad
  f(x,\,y) \in O(\frac{1}{\sqrt{1\!-\!x^2}}) \mbox{\; as \;} x \to 1,$$
where the \PMlinkname{Landau big ordo}{formaldefinitionoflandaunotation} notation has been used.\, Accordingly, the integral (2) converges for every $y$.

The inequality
$$\left|\frac{\partial f(x,\,y)}{\partial y}\right| \;=\; \frac{1}{(1\!+\!x^2y^2)\sqrt{1\!-\!x^2}} 
\;\leqq\; \frac{1}{\sqrt{1\!-\!x^2}}$$
and the convergence of the integral
$$\int_0^1\!\frac{dx}{\sqrt{1\!-\!x^2}} \;=\; \frac{\pi}{2}$$
imply that the integral
\begin{align}
\int_0^1\frac{\partial f(x,\,y)}{\partial y}\,dx
\end{align}
\PMlinkid{converges uniformly}{6277} on the whole $y$-axis and equals $I'(y)$.\, For expressing this derivative in a \PMlinkname{closed form}{ExpressibleInClosedForm}, one may utilise the \PMlinkname{changes of variable}{ChangeOfVariableInDefiniteIntegral}
$$x \;:=\; \cos\varphi, \quad \tan\varphi \;:=\; t$$
which yield
\begin{align*}
I'(y) & \;=\; \int_0^1\!\frac{dx}{(1\!+\!x^2y^2)\sqrt{1\!-\!x^2}} 
\;=\; \int_0^{\frac{\pi}{2}}\!\frac{d\varphi}{1\!+\!y^2\cos^2\varphi}\\ 
      & \;=\; \int_0^\infty\!\frac{dt}{1\!+\!y^2\!+\!t^2} 
\;=\; \sijoitus{t\,=0}{\quad\infty}\!\frac{1}{\sqrt{1\!+\!y^2}}\arctan\frac{t}{\sqrt{1\!+\!y^2}}\\   
      & \;=\; \frac{\pi}{2\sqrt{1\!+\!y^2}}.
\end{align*}
Hence,
$$I(y) \;=\; \frac{\pi}{2}\int_0^y\frac{dy}{\sqrt{1\!+\!y^2}} \;=\; \sijoitus{0}{\quad y}\!\ln(y+\sqrt{1\!+\!y^2})$$
and the integral (1) equals \;$I \;=\; I(1) \;=\; \frac{\pi}{2}\ln(1\!+\!\sqrt{2})$,\, i.e.
\begin{align}
\int_0^1\frac{\arctan{x}}{x\sqrt{1\!-\!x^2}}\,dx \;=\; \frac{\pi}{2}\ln(1\!+\!\sqrt{2}).
\end{align}

%%%%%
%%%%%
\end{document}
