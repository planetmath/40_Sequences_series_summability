\documentclass[12pt]{article}
\usepackage{pmmeta}
\pmcanonicalname{ProofOfCompositionLimitLawForUniformConvergence}
\pmcreated{2013-03-22 15:23:08}
\pmmodified{2013-03-22 15:23:08}
\pmowner{stevecheng}{10074}
\pmmodifier{stevecheng}{10074}
\pmtitle{proof of composition limit law for uniform convergence}
\pmrecord{4}{37216}
\pmprivacy{1}
\pmauthor{stevecheng}{10074}
\pmtype{Proof}
\pmcomment{trigger rebuild}
\pmclassification{msc}{40A30}

\endmetadata

% this is the default PlanetMath preamble.  as your knowledge
% of TeX increases, you will probably want to edit this, but
% it should be fine as is for beginners.

% almost certainly you want these
\usepackage{amssymb}
\usepackage{amsmath}
\usepackage{amsfonts}
\usepackage{amsthm}

% used for TeXing text within eps files
%\usepackage{psfrag}
% need this for including graphics (\includegraphics)
%\usepackage{graphicx}
% for neatly defining theorems and propositions
%\usepackage{amsthm}
% making logically defined graphics
%%%\usepackage{xypic}

% there are many more packages, add them here as you need them
\usepackage{enumerate}

% define commands here
\newcommand{\real}{\mathbb{R}}
\newcommand{\rat}{\mathbb{Q}}
\newcommand{\nat}{\mathbb{N}}

\providecommand{\abs}[1]{\lvert#1\rvert}
\providecommand{\absW}[1]{\left\lvert#1\right\rvert}
\providecommand{\absB}[1]{\Bigl\lvert#1\Bigr\rvert}
\providecommand{\norm}[1]{\lVert#1\rVert}
\providecommand{\normW}[1]{\left\lVert#1\right\rVert}
\providecommand{\normB}[1]{\Bigl\lVert#1\Bigr\rVert}
\providecommand{\defnterm}[1]{\emph{#1}}

\providecommand{\clos}[1]{\overline{#1}}

\newtheorem{thm}{Theorem}
\begin{document}
\begin{thm}
Let $X, Y, Z$ be metric spaces, with $X$ compact and $Y$ locally compact.
If $f_n\colon X \to Y$ is a sequence of functions converging uniformly
to a continuous function $f\colon X \to Y$, and $h\colon Y \to Z$
is continuous, then $h\circ f_n$ converge to $h \circ f$ uniformly.

\begin{proof}
Let $K$ denote the compact set $f(X) \subseteq Y$.  By local compactness of $Y$,for each point $y \in K$, there is an open neighbourhood $U_y$ of $y$ such that
$\clos{U_y}$ is compact.  The neighbourhoods $U_y$ cover $K$, so there is a finite
subcover $U_{y_1}, \dotsc, U_{y_n}$ covering $K$.  Let
$U = \bigcup_i U_{y_i} \supseteq K$.  Evidently
$\clos{U} = \bigcup_i \clos{U_{y_i}}$ is compact.

Next, let $V$ be the $\delta_0$-neighbourhood  of $K$
contained in $U$, for some $\delta_0 > 0$.
$\clos{V}$ is compact, since it is contained in $\clos{U}$.

Now let $\epsilon > 0$ be given.
$h$ is uniformly continuous on $\clos{V}$, so
there exists a $\delta > 0$ such that
when $y, y' \in \clos{V}$ and $d(y, y') < \delta$,
we have $d(g(y), g(y')) < \epsilon$.

From the uniform convergence of $f_n$, choose $N$ so that
when $n \geq N$, $d(f_n(x), f(x)) < \min(\delta, \delta_0)$
for all $x \in X$.
Since $f(x) \in K$, it follows that $f_n(x)$ is inside the
$\delta_0$-neighbourhood of $K$, i.e. both $y = f_n(x)$ and $y' = f(x)$
are both in $V$.  Thus $d(g(f_n(x)), g(f(x))) < \epsilon$ when $n \geq N$,
uniformly for all $x \in X$.
\end{proof}
\end{thm}
%%%%%
%%%%%
\end{document}
