\documentclass[12pt]{article}
\usepackage{pmmeta}
\pmcanonicalname{ArithmeticSeries}
\pmcreated{2013-03-22 16:17:58}
\pmmodified{2013-03-22 16:17:58}
\pmowner{georgiosl}{7242}
\pmmodifier{georgiosl}{7242}
\pmtitle{arithmetic series}
\pmrecord{10}{38420}
\pmprivacy{1}
\pmauthor{georgiosl}{7242}
\pmtype{Definition}
\pmcomment{trigger rebuild}
\pmclassification{msc}{40A05}

\endmetadata

% this is the default PlanetMath preamble.  as your knowledge
% of TeX increases, you will probably want to edit this, but
% it should be fine as is for beginners.

% almost certainly you want these
\usepackage{amssymb}
\usepackage{amsmath}
\usepackage{amsfonts}

% used for TeXing text within eps files
%\usepackage{psfrag}
% need this for including graphics (\includegraphics)
%\usepackage{graphicx}
% for neatly defining theorems and propositions
%\usepackage{amsthm}
% making logically defined graphics
%%%\usepackage{xypic}

% there are many more packages, add them here as you need them

% define commands here

\begin{document}
An \emph{arithmetic series} is the series, $\sum_{i=1}^na_i$, in which each real term has the form $ a_i=a_{i-1}+d $ for $i=2,\ldots, n $ where $ d$ is constant. The sum of the sequence is given by the following
$\displaystyle \frac{1}{2}n[2a_1+d(n-1)].$
In order to find the formula above firstly we express the terms of the sequence, $ a_2, \ldots, a_n$ in terms of $ a_1$ and the constant $ d$. In this case we get $ a_2=a_1+d, a_3=a_2+2d,\ldots , a_n=a_1+(n-1)d$. Now we express the sum of the sequence by developing the series forward and we have:
$$S_n=\sum_{i=1}^na_i =a_1+a_1+d+\cdots +a_1+(n-2)d+a_1+(n-1)d$$
Reversely, we develop the series backwards and we get
$$S_n=a_n-d+a_n-2d+\cdots +a_n-(n-1)d$$
\\It is easily seen that by adding the two expressions we get
\begin{eqnarray}
2S_n=n(a_1+a_n)\\
S_n= \frac{1}{2}n(a_1+a_n)
\end{eqnarray}
Hence, by substituting $a_n=a_1+(n-1)d$ we get the first formula. 
%%%%%
%%%%%
\end{document}
