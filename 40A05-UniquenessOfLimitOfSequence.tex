\documentclass[12pt]{article}
\usepackage{pmmeta}
\pmcanonicalname{UniquenessOfLimitOfSequence}
\pmcreated{2013-03-22 19:00:23}
\pmmodified{2013-03-22 19:00:23}
\pmowner{pahio}{2872}
\pmmodifier{pahio}{2872}
\pmtitle{uniqueness of limit of sequence}
\pmrecord{4}{41875}
\pmprivacy{1}
\pmauthor{pahio}{2872}
\pmtype{Theorem}
\pmcomment{trigger rebuild}
\pmclassification{msc}{40A05}

\endmetadata

% this is the default PlanetMath preamble.  as your knowledge
% of TeX increases, you will probably want to edit this, but
% it should be fine as is for beginners.

% almost certainly you want these
\usepackage{amssymb}
\usepackage{amsmath}
\usepackage{amsfonts}

% used for TeXing text within eps files
%\usepackage{psfrag}
% need this for including graphics (\includegraphics)
%\usepackage{graphicx}
% for neatly defining theorems and propositions
 \usepackage{amsthm}
% making logically defined graphics
%%%\usepackage{xypic}

% there are many more packages, add them here as you need them

% define commands here

\theoremstyle{definition}
\newtheorem*{thmplain}{Theorem}

\begin{document}
If a number sequence has a limit, then the limit is uniquely determined.\\

\emph{Proof.}\, For an \PMlinkname{indirect proof}{ReductioAdAbsurdum}, suppose that a sequence
$$a_1,\,a_2,\,a_3,\,\ldots$$
has two distinct limits $a$ and $b$.\, Thus we must have both
$$|a_n\!-\!a| < \frac{|a\!-\!b|}{2} \quad \mbox{for all}\;\; n > \mbox{\,some\;} n_1$$
and
$$|a_n\!-\!b| < \frac{|a\!-\!b|}{2} \quad \mbox{for all}\;\; n > \mbox{\,some\;} n_2$$
But when $n$ exceeds the greater of $n_1$ and $n_2$, we can write
$$|a\!-\!b| \;=\; |a\!-\!a_n\!+\!a_n\!-\!b|\;\leqq\; |a\!-\!a_n|+|a_n\!-\!b| 
\;<\; \frac{|a\!-\!b|}{2}+\frac{|a\!-\!b|}{2} \;=\; |a\!-\!b|.$$
This inequality \PMlinkescapetext{chain contains} an impossibility, whence the antithesis made in the begin is wrong and the assertion is \PMlinkescapetext{right}.

%%%%%
%%%%%
\end{document}
