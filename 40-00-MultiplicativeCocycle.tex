\documentclass[12pt]{article}
\usepackage{pmmeta}
\pmcanonicalname{MultiplicativeCocycle}
\pmcreated{2014-03-19 22:13:54}
\pmmodified{2014-03-19 22:13:54}
\pmowner{Filipe}{28191}
\pmmodifier{Filipe}{28191}
\pmtitle{multiplicative cocycle}
\pmrecord{4}{88074}
\pmprivacy{1}
\pmauthor{Filipe}{28191}
\pmtype{Definition}
\pmsynonym{cocycle; multiplicative linear cocycle}{MultiplicativeCocycle}
\pmrelated{Furstenberg-Kesten theorem}
\pmdefines{multiplicative cocycle}

\endmetadata

% this is the default PlanetMath preamble.  as your knowledge
% of TeX increases, you will probably want to edit this, but
% it should be fine as is for beginners.

% almost certainly you want these
\usepackage{amssymb}
\usepackage{amsmath}
\usepackage{amsfonts}

% need this for including graphics (\includegraphics)
\usepackage{graphicx}
% for neatly defining theorems and propositions
\usepackage{amsthm}

% making logically defined graphics
%\usepackage{xypic}
% used for TeXing text within eps files
%\usepackage{psfrag}

% there are many more packages, add them here as you need them

% define commands here

\begin{document}
Let $f:M\rightarrow M$ be a measurable transformation, and let $\mu$ be an invariant probability measure. Consider $A:M\rightarrow GL(d,\textbf{R})$, a measurable transformation, where GL(d,\textbf{R}) is the space of invertible square matrices of size $d$. We define $A^{-1}:M\rightarrow GL(d,\textbf{R})$ by $A^{-1}(x)=[A(x)]^{-1}$.
Then we define the sequence of functions:
$$\phi^n (x)=A(f^{n-1}(x)) \cdots A(f(x))A(x)$$
$$\phi^{-n}(x)=[\phi^n(f^{-n}(x))]^{-1}$$
for $n\geq 1$ and $x \in M$.

It is easy to verify that:
$$\phi^{m+n}(x)=\phi^n(f^m(x))\phi^m(x)$$
for $n,m\in \textbf{Z}$ and $x \in M$.

The sequence $(\phi^n)_n$ is called a multiplicative cocycle, or just cocycle defined by the transformation $A$.
\end{document}
