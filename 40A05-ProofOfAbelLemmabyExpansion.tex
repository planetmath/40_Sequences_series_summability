\documentclass[12pt]{article}
\usepackage{pmmeta}
\pmcanonicalname{ProofOfAbelLemmabyExpansion}
\pmcreated{2013-03-22 17:28:14}
\pmmodified{2013-03-22 17:28:14}
\pmowner{perucho}{2192}
\pmmodifier{perucho}{2192}
\pmtitle{proof of Abel lemma (by expansion)}
\pmrecord{7}{39856}
\pmprivacy{1}
\pmauthor{perucho}{2192}
\pmtype{Proof}
\pmcomment{trigger rebuild}
\pmclassification{msc}{40A05}

\endmetadata

% this is the default PlanetMath preamble.  as your knowledge
% of TeX increases, you will probably want to edit this, but
% it should be fine as is for beginners.

% almost certainly you want these
\usepackage{amssymb}
\usepackage{amsmath}
\usepackage{amsfonts}

% used for TeXing text within eps files
%\usepackage{psfrag}
% need this for including graphics (\includegraphics)
%\usepackage{graphicx}
% for neatly defining theorems and propositions
%\usepackage{amsthm}
% making logically defined graphics
%%%\usepackage{xypic}

% there are many more packages, add them here as you need them

% define commands here
\newtheorem{theorem}{Theorem}
\newtheorem{defn}{Definition}
\newtheorem{prop}{Proposition}
\newtheorem{lemma}{Lemma}
\newtheorem{cor}{Corollary}

\begin{document}
\section{Abel lemma}
\begin{equation}
\sum_{i=0}^n a_ib_i=\sum_{i=0}^{n-1} A_i(b_i-b_{i+1})+A_nb_n,
\end{equation}
where, $A_i=\sum_{k=0}^i a_k$. Sequences $\{a_i\}$, $\{b_i\}$, $i=0,\dots, n$, are real or complex one.
\section{Proof}
We consider the expansion of the sum
\begin{equation*}
\sum_{i=0}^n A_i(b_i-b_{i+1})
\end{equation*}
on two different forms, namely:
\begin{enumerate}
\item On the short way.
\begin{equation}
\sum_{i=0}^n A_i(b_i-b_{i+1})=\sum_{i=0}^{n-1} A_i(b_i-b_{i+1})+A_nb_n-A_nb_{n+1}.
\end{equation}
\item On the long way.
\end{enumerate}
\begin{equation*}
\sum_{i=0}^n A_i(b_i-b_{i+1})=\sum_{i=0}^n A_ib_i-\sum_{i=0}^n A_ib_{i+1}=
\sum_{i=0}^n A_ib_i-\sum_{i=1}^{n+1} A_{i-1}b_i=
\end{equation*}
\begin{equation}
A_0 b_0+\sum_{i=1}^n (A_{i-1}+a_i)b_i-\sum_{i=1}^n A_{i-1}b_i-A_nb_{n+1}=\sum_{i=0}^n a_ib_i-A_nb_{n+1},
\end{equation}
where a simplification has been performed. Notice that $A_0=a_0$. By equating (2), (3), the last two terms cancel, {\footnote{Without loss of generality, $b_{n+1}$ may be assumed finite. Indeed we don't need $b_{n+1}$, but the proof is a couple lines larger. It is left as an exercise.}} and then, (1) follows. $\Box$



%%%%%
%%%%%
\end{document}
